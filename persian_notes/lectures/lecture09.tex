% lecture09.tex - تغییر پایه
% Chapter 9: Change of Basis

\chapter{تغییر پایه}
\label{ch:changebasis}

\begin{abstract}
یک بردار یکسان در پایه‌های مختلف، مختصات متفاوتی دارد. در این درس یاد می‌گیریم چگونه بین دستگاه‌های مختصات مختلف ترجمه کنیم و این مفهوم چه ارتباطی با ماتریس‌ها دارد.
\end{abstract}

% ============================================
\section{مختصات نسبت به پایه‌های مختلف}
% ============================================

\begin{intuition}
مختصات یک بردار به پایه انتخابی بستگی دارد. اگر زبان متفاوتی صحبت کنید، همان مفهوم را متفاوت بیان می‌کنید.

مثال: بردار $\vv$ که در پایه استاندارد $\twovec{3}{2}$ است، اگر پایه‌ای متفاوت داشته باشیم، ممکن است $\twovec{1}{1}$ باشد!
\end{intuition}

\begin{definition}[مختصات در پایه]
اگر $\mathcal{B} = \{\vb_1, \vb_2\}$ یک پایه باشد، مختصات بردار $\vv$ در این پایه، ضرایب $c_1, c_2$ هستند که:
\[
\vv = c_1 \vb_1 + c_2 \vb_2
\]
نماد: $[\vv]_{\mathcal{B}} = \twovec{c_1}{c_2}$
\end{definition}

\begin{example}
پایه $\mathcal{B} = \left\{ \vb_1 = \twovec{2}{1}, \vb_2 = \twovec{-1}{1} \right\}$

بردار $\vv = \twovec{3}{2}$ را در این پایه بیان کنید.

باید $c_1, c_2$ را پیدا کنیم که:
\[
c_1 \twovec{2}{1} + c_2 \twovec{-1}{1} = \twovec{3}{2}
\]
\[
\system{
2c_1 - c_2 &= 3 \\
c_1 + c_2 &= 2
}
\]
حل: $c_1 = \frac{5}{3}$, $c_2 = \frac{1}{3}$

پس $[\vv]_{\mathcal{B}} = \twovec{5/3}{1/3}$
\end{example}

% ============================================
\section{ماتریس تغییر پایه}
% ============================================

\begin{definition}[ماتریس تغییر پایه]
\vocab{ماتریس تغییر پایه} از پایه $\mathcal{B}$ به پایه استاندارد، ماتریسی است که ستون‌هایش بردارهای پایه $\mathcal{B}$ هستند:
\[
\mP = [\vb_1 \,|\, \vb_2 \,|\, \cdots \,|\, \vb_n]
\]
\end{definition}

\begin{theorem}
\[
\vv = \mP \, [\vv]_{\mathcal{B}}
\]
یعنی: مختصات در پایه $\mathcal{B}$ $\times$ ماتریس تغییر پایه = بردار در پایه استاندارد
\end{theorem}

\begin{theorem}
\[
[\vv]_{\mathcal{B}} = \mP\inv \vv
\]
یعنی: برای تبدیل از پایه استاندارد به پایه $\mathcal{B}$، از معکوس استفاده می‌کنیم.
\end{theorem}

% ============================================
\section{تبدیل تغییر پایه}
% ============================================

\begin{intuition}
ماتریس $\mP$ یک تبدیل هویت است که فقط زبان را عوض می‌کند:
\begin{itemize}
    \item $\mP$: از زبان $\mathcal{B}$ به زبان استاندارد ترجمه می‌کند
    \item $\mP\inv$: از زبان استاندارد به زبان $\mathcal{B}$ ترجمه می‌کند
\end{itemize}
\end{intuition}

\begin{center}
\begin{tikzpicture}[scale=1.2]
    % Standard basis
    \begin{scope}[shift={(0,0)}]
        \draw[grid] (-0.5,-0.5) grid (3.5,2.5);
        \draw[axis] (-0.5,0) -- (3.5,0);
        \draw[axis] (0,-0.5) -- (0,2.5);
        \draw[basis] (0,0) -- (1,0) node[below] {$\vi$};
        \draw[basis] (0,0) -- (0,1) node[left] {$\vj$};
        \draw[vector, blue!70, very thick] (0,0) -- (3,2) node[above] {$\vv$};
        \node at (1.5,-1) {پایه استاندارد};
        \node at (1.5,-1.5) {$\vv = \twovec{3}{2}$};
    \end{scope}

    % Alternative basis
    \begin{scope}[shift={(6,0)}]
        \draw[gray!30] (-0.5,-0.5) -- (1,0.5) -- (3,1.5) -- (1.5,1) -- cycle;
        \draw[axis] (-0.5,0) -- (3.5,0);
        \draw[axis] (0,-0.5) -- (0,2.5);
        \draw[red!70, very thick, -{Stealth}] (0,0) -- (2,1) node[below right] {$\vb_1$};
        \draw[red!70, very thick, -{Stealth}] (0,0) -- (-1,1) node[above left] {$\vb_2$};
        \draw[vector, blue!70, very thick] (0,0) -- (3,2) node[above] {$\vv$};
        \node at (1.5,-1) {پایه $\mathcal{B}$};
        \node at (1.5,-1.5) {$[\vv]_{\mathcal{B}} = \twovec{?}{?}$};
    \end{scope}
\end{tikzpicture}
\end{center}

% ============================================
\section{نمایش تبدیل در پایه‌های مختلف}
% ============================================

\begin{theorem}[تبدیل در پایه جدید]
اگر $\mA$ ماتریس تبدیل $T$ در پایه استاندارد باشد، ماتریس همان تبدیل در پایه $\mathcal{B}$:
\[
[\mA]_{\mathcal{B}} = \mP\inv \mA \mP
\]
\end{theorem}

\begin{intuition}
این فرمول سه مرحله دارد:
\begin{enumerate}
    \item $\mP$: از پایه $\mathcal{B}$ به پایه استاندارد ترجمه کن
    \item $\mA$: تبدیل را در پایه استاندارد اعمال کن
    \item $\mP\inv$: نتیجه را به پایه $\mathcal{B}$ برگردان
\end{enumerate}
\end{intuition}

\begin{center}
\begin{tikzpicture}[node distance=3cm]
    \node (B1) {$[\vv]_{\mathcal{B}}$};
    \node (S1) [right of=B1] {$\vv$};
    \node (S2) [right of=S1] {$T(\vv)$};
    \node (B2) [right of=S2] {$[T(\vv)]_{\mathcal{B}}$};

    \draw[-{Stealth}, thick] (B1) -- (S1) node[midway, above] {$\mP$};
    \draw[-{Stealth}, thick] (S1) -- (S2) node[midway, above] {$\mA$};
    \draw[-{Stealth}, thick] (S2) -- (B2) node[midway, above] {$\mP\inv$};
    \draw[-{Stealth}, thick, blue] (B1) to[bend right=30] node[midway, below] {$\mP\inv\mA\mP$} (B2);
\end{tikzpicture}
\end{center}

% ============================================
\section{اهمیت تغییر پایه}
% ============================================

\begin{intuition}
چرا تغییر پایه مهم است؟

بعضی تبدیلات در پایه‌های خاص \textbf{ساده‌تر} به نظر می‌رسند. مثلاً:
\begin{itemize}
    \item چرخش در پایه استاندارد پیچیده است
    \item اما در پایه‌ای که یک محور روی محور چرخش باشد، ساده می‌شود
\end{itemize}

بهترین پایه برای یک تبدیل؟ \textbf{پایه ویژه} (eigenbasis) - درس بعدی!
\end{intuition}

\begin{practical}
\textbf{کاربرد: ساده‌سازی محاسبات}

فرض کنید می‌خواهید $\mA^{100}$ را محاسبه کنید. اگر در پایه‌ای مناسب، $\mA$ قطری شود:
\[
\mP\inv \mA \mP = \mD \quad \text{(قطری)}
\]
آنگاه:
\[
\mA^{100} = \mP \mD^{100} \mP\inv
\]
و $\mD^{100}$ بسیار ساده محاسبه می‌شود!
\end{practical}

% ============================================
\section{تمرین‌ها}
% ============================================

\begin{exercise}
پایه $\mathcal{B} = \left\{ \twovec{1}{1}, \twovec{1}{-1} \right\}$ داده شده. مختصات بردار $\vv = \twovec{3}{1}$ را در این پایه پیدا کنید.
\end{exercise}

\begin{exercise}
ماتریس تغییر پایه از $\mathcal{B} = \left\{ \twovec{2}{1}, \twovec{1}{1} \right\}$ به پایه استاندارد را بنویسید.
\end{exercise}

\begin{exercise}
اگر $\mA = \twomat{2}{1}{0}{3}$ در پایه استاندارد، ماتریس این تبدیل را در پایه $\mathcal{B} = \left\{ \twovec{1}{0}, \twovec{1}{1} \right\}$ پیدا کنید.
\end{exercise}

\begin{exercise}[چالشی]
نشان دهید که $\det(\mP\inv\mA\mP) = \det(\mA)$.
\end{exercise}

\begin{problem}
دو پایه $\mathcal{B}_1$ و $\mathcal{B}_2$ داده شده. ماتریس تغییر پایه مستقیم از $\mathcal{B}_1$ به $\mathcal{B}_2$ (بدون گذر از پایه استاندارد) را چگونه محاسبه می‌کنید؟
\end{problem}

% lecture06.tex - ضرب داخلی و دوگانگی
% Chapter 6: Dot Products and Duality

\chapter{ضرب داخلی و دوگانگی}
\label{ch:dotproduct}

\begin{abstract}
ضرب داخلی (ضرب نقطه‌ای) یکی از عملیات‌های بنیادین جبر خطی است. در این درس با تعریف، خواص، و مفهوم عمیق دوگانگی آشنا می‌شویم که ارتباط بین بردارها و تابع‌های خطی را نشان می‌دهد.
\end{abstract}

% ============================================
\section{تعریف ضرب داخلی}
% ============================================

\begin{definition}[ضرب داخلی]
\vocab{ضرب داخلی} \lr{(Dot Product)} دو بردار $\vv = \twovec{v_1}{v_2}$ و $\vw = \twovec{w_1}{w_2}$:
\[
\vv \cdot \vw = v_1 w_1 + v_2 w_2
\]
در فضای $n$-بعدی:
\[
\vv \cdot \vw = \sum_{i=1}^{n} v_i w_i
\]
\end{definition}

\begin{important}
ضرب داخلی دو بردار، یک \textbf{عدد} (اسکالر) است، نه بردار!
\end{important}

\begin{example}
\[
\twovec{3}{2} \cdot \twovec{1}{4} = 3 \times 1 + 2 \times 4 = 3 + 8 = 11
\]
\end{example}

% ============================================
\section{تفسیر هندسی}
% ============================================

\begin{theorem}[فرمول هندسی ضرب داخلی]
\[
\vv \cdot \vw = \norm{\vv} \norm{\vw} \cos\theta
\]
که $\theta$ زاویه بین دو بردار است.
\end{theorem}

\begin{intuition}
ضرب داخلی را می‌توان به دو صورت تفسیر کرد:
\begin{enumerate}
    \item طول $\vv$ ضرب در طول تصویر $\vw$ روی $\vv$
    \item طول $\vw$ ضرب در طول تصویر $\vv$ روی $\vw$
\end{enumerate}
\end{intuition}

\begin{center}
\begin{tikzpicture}[scale=1.5]
    % Vectors
    \draw[vector, blue!70, very thick] (0,0) -- (3,0) node[right] {$\vv$};
    \draw[vector, green!60!black, very thick] (0,0) -- (2,1.5) node[above] {$\vw$};

    % Projection
    \draw[dashed, red] (2,1.5) -- (2,0);
    \draw[red, thick] (0,0) -- (2,0) node[midway, below] {تصویر $\vw$};

    % Angle
    \draw (0.7,0) arc (0:36.87:0.7) node[midway, right] {$\theta$};
\end{tikzpicture}
\end{center}

% ============================================
\section{خواص ضرب داخلی}
% ============================================

\begin{theorem}[خواص اصلی]
\begin{enumerate}
    \item \textbf{جابجایی:} $\vv \cdot \vw = \vw \cdot \vv$
    \item \textbf{توزیع‌پذیری:} $\vv \cdot (\vw + \vu) = \vv \cdot \vw + \vv \cdot \vu$
    \item \textbf{همگنی:} $(c\vv) \cdot \vw = c(\vv \cdot \vw)$
    \item \textbf{مثبت‌بودن:} $\vv \cdot \vv \geq 0$ و $\vv \cdot \vv = 0 \Leftrightarrow \vv = \vzero$
\end{enumerate}
\end{theorem}

\begin{definition}[طول بردار]
\[
\norm{\vv} = \sqrt{\vv \cdot \vv} = \sqrt{v_1^2 + v_2^2 + \cdots + v_n^2}
\]
\end{definition}

% ============================================
\section{عمود بودن}
% ============================================

\begin{theorem}[شرط عمود بودن]
دو بردار $\vv$ و $\vw$ بر هم \vocab{عمود} هستند اگر و تنها اگر:
\[
\vv \cdot \vw = 0
\]
\end{theorem}

\begin{intuition}
اگر $\vv \cdot \vw = 0$، یعنی $\cos\theta = 0$، پس $\theta = 90°$.

تصویر هر بردار روی بردار عمود بر آن، صفر است.
\end{intuition}

\begin{example}
بردارهای $\vv = \twovec{3}{2}$ و $\vw = \twovec{2}{-3}$ عمود هستند زیرا:
\[
\vv \cdot \vw = 3 \times 2 + 2 \times (-3) = 6 - 6 = 0
\]
\end{example}

% ============================================
\section{تصویر برداری}
% ============================================

\begin{definition}[تصویر بردار]
تصویر بردار $\vv$ روی بردار $\vw$:
\[
\proj_{\vw}(\vv) = \frac{\vv \cdot \vw}{\vw \cdot \vw} \vw = \frac{\vv \cdot \vw}{\norm{\vw}^2} \vw
\]
\end{definition}

\begin{intuition}
تصویر $\vv$ روی $\vw$ برداری است در راستای $\vw$ که «سایه» $\vv$ روی خط $\vw$ را نشان می‌دهد.
\end{intuition}

\begin{center}
\begin{tikzpicture}[scale=1.5]
    \draw[vector, blue!70, very thick] (0,0) -- (3,0) node[right] {$\vw$};
    \draw[vector, green!60!black, very thick] (0,0) -- (2,2) node[above] {$\vv$};
    \draw[vector, red!70, very thick] (0,0) -- (2,0) node[below] {$\proj_{\vw}(\vv)$};
    \draw[dashed, gray] (2,2) -- (2,0);
\end{tikzpicture}
\end{center}

% ============================================
\section{دوگانگی \lr{(Duality)}}
% ============================================

\begin{intuition}
یک بینش عمیق: هر بردار سطری $1 \times n$ را می‌توان به عنوان یک \textbf{تابع خطی} در نظر گرفت که بردارهای $n$-بعدی را به اعداد می‌برد.

برای مثال، $\bmat{2 & 1}$ یک تابع خطی است:
\[
\bmat{2 & 1} \twovec{x}{y} = 2x + y
\]
\end{intuition}

\begin{theorem}[دوگانگی]
هر تابع خطی $f: \R^n \to \R$ را می‌توان به صورت ضرب داخلی با یک بردار ثابت نوشت:
\[
f(\vx) = \vv \cdot \vx
\]
برای یک بردار یکتای $\vv$.
\end{theorem}

\begin{definition}[بردار دوگان]
بردار $\vv$ که تابع خطی $f$ را نمایش می‌دهد، \vocab{بردار دوگان} آن تابع نامیده می‌شود.
\end{definition}

\begin{practical}
\textbf{کاربرد در یادگیری ماشین:}

در شبکه‌های عصبی، هر نورون یک تابع خطی روی ورودی‌ها محاسبه می‌کند:
\[
\text{خروجی} = w_1 x_1 + w_2 x_2 + \cdots + w_n x_n = \vw \cdot \vx
\]
وزن‌های $\vw$ بردار دوگان آن نورون هستند.
\end{practical}

% ============================================
\section{کاربردهای عملی}
% ============================================

\begin{practical}
\textbf{فیزیک: کار مکانیکی}

کار انجام شده توسط نیروی $\vec{F}$ در طول جابجایی $\vec{d}$:
\[
W = \vec{F} \cdot \vec{d} = \norm{\vec{F}} \norm{\vec{d}} \cos\theta
\]
اگر نیرو عمود بر جهت حرکت باشد، کار صفر است!
\end{practical}

\begin{practical}
\textbf{گرافیک کامپیوتری: محاسبه نور}

شدت نوری که از سطح منعکس می‌شود:
\[
I = \max(0, \vec{n} \cdot \vec{l})
\]
که $\vec{n}$ بردار نرمال سطح و $\vec{l}$ جهت نور است.
\end{practical}

\begin{practical}
\textbf{تشابه متون}

برای مقایسه دو سند متنی، هر سند را به بردار تبدیل می‌کنیم (مثلاً TF-IDF) و تشابه کسینوسی محاسبه می‌کنیم:
\[
\text{similarity} = \frac{\vv \cdot \vw}{\norm{\vv}\norm{\vw}} = \cos\theta
\]
\end{practical}

% ============================================
\section{تمرین‌ها}
% ============================================

\begin{exercise}
ضرب داخلی بردارهای زیر را محاسبه کنید:
\begin{enumerate}[label=(\alph*)]
    \item $\twovec{1}{2} \cdot \twovec{3}{4}$
    \item $\threevec{1}{-1}{2} \cdot \threevec{2}{3}{-1}$
\end{enumerate}
\end{exercise}

\begin{exercise}
آیا بردارهای $\va = \twovec{4}{3}$ و $\vb = \twovec{-3}{4}$ عمود هستند؟
\end{exercise}

\begin{exercise}
تصویر بردار $\vv = \twovec{3}{4}$ روی بردار $\vw = \twovec{1}{0}$ را پیدا کنید.
\end{exercise}

\begin{exercise}
زاویه بین بردارهای $\va = \twovec{1}{1}$ و $\vb = \twovec{1}{0}$ را محاسبه کنید.
\end{exercise}

\begin{exercise}[چالشی]
نشان دهید که برای هر دو بردار $\vv$ و $\vw$:
\[
\norm{\vv + \vw}^2 = \norm{\vv}^2 + 2(\vv \cdot \vw) + \norm{\vw}^2
\]
\end{exercise}

\begin{problem}
نیرویی با بزرگی $10$ نیوتن تحت زاویه $60°$ با سطح افقی به جسمی وارد می‌شود. اگر جسم $5$ متر در راستای افقی حرکت کند، چقدر کار انجام شده؟
\end{problem}

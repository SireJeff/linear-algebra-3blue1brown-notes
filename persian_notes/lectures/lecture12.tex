% lecture12.tex - فضاهای برداری انتزاعی
% Chapter 12: Abstract Vector Spaces

\chapter{فضاهای برداری انتزاعی}
\label{ch:abstract}

\begin{abstract}
در این درس پایانی، مفهوم بردار را فراتر از پیکان‌های هندسی گسترش می‌دهیم. خواهیم دید که توابع، چندجمله‌ای‌ها، و حتی موسیقی می‌توانند «بردار» باشند - به شرطی که قوانین جبر خطی را رعایت کنند.
\end{abstract}

% ============================================
\section{انگیزه: بردار چیست؟}
% ============================================

\begin{intuition}
تا کنون بردار را به عنوان پیکان یا لیست اعداد شناختیم. اما ریاضیدان‌ها سؤال عمیق‌تری می‌پرسند:

\textbf{«چه چیزهایی مثل بردار رفتار می‌کنند؟»}

پاسخ: هر چیزی که بتوان آن را جمع کرد و در اسکالر ضرب کرد!
\end{intuition}

\begin{example}
توابع را می‌توان جمع کرد: $(f + g)(x) = f(x) + g(x)$

توابع را می‌توان در عدد ضرب کرد: $(cf)(x) = c \cdot f(x)$

پس توابع می‌توانند «بردار» باشند!
\end{example}

% ============================================
\section{تعریف رسمی فضای برداری}
% ============================================

\begin{definition}[فضای برداری]
یک \vocab{فضای برداری} روی میدان $\R$ مجموعه‌ای $V$ با دو عملیات است:
\begin{itemize}
    \item \textbf{جمع:} $+ : V \times V \to V$
    \item \textbf{ضرب اسکالری:} $\cdot : \R \times V \to V$
\end{itemize}
که اصول زیر (آکسیوم‌ها) را برآورده کند.
\end{definition}

\begin{theorem}[آکسیوم‌های فضای برداری]
برای همه $\vu, \vv, \vw \in V$ و $a, b \in \R$:

\textbf{آکسیوم‌های جمع:}
\begin{enumerate}
    \item $\vu + \vv = \vv + \vu$ (جابجایی)
    \item $(\vu + \vv) + \vw = \vu + (\vv + \vw)$ (شرکت‌پذیری)
    \item وجود عنصر خنثی: $\exists \vzero : \vv + \vzero = \vv$
    \item وجود قرینه: $\forall \vv, \exists (-\vv) : \vv + (-\vv) = \vzero$
\end{enumerate}

\textbf{آکسیوم‌های ضرب اسکالری:}
\begin{enumerate}[resume]
    \item $a(b\vv) = (ab)\vv$
    \item $1 \cdot \vv = \vv$
\end{enumerate}

\textbf{آکسیوم‌های توزیع:}
\begin{enumerate}[resume]
    \item $a(\vu + \vv) = a\vu + a\vv$
    \item $(a + b)\vv = a\vv + b\vv$
\end{enumerate}
\end{theorem}

% ============================================
\section{مثال‌های فضای برداری}
% ============================================

\subsection{$\R^n$ - فضای استاندارد}

\begin{example}
$\R^n = \{(x_1, x_2, \ldots, x_n) \mid x_i \in \R\}$ با جمع و ضرب معمولی.

این همان فضایی است که تا کنون با آن کار کردیم.
\end{example}

\subsection{فضای چندجمله‌ای‌ها}

\begin{example}
$\mathcal{P}_n$ = مجموعه چندجمله‌ای‌های با درجه حداکثر $n$:
\[
p(x) = a_0 + a_1 x + a_2 x^2 + \cdots + a_n x^n
\]

جمع: $(p + q)(x) = p(x) + q(x)$

ضرب اسکالری: $(cp)(x) = c \cdot p(x)$

بردار صفر: $p(x) = 0$

\textbf{پایه:} $\{1, x, x^2, \ldots, x^n\}$ - بعد فضا: $n + 1$
\end{example}

\subsection{فضای توابع}

\begin{example}
$C[a,b]$ = مجموعه توابع پیوسته روی بازه $[a,b]$

این فضا \textbf{بی‌نهایت بعدی} است!
\end{example}

\subsection{فضای ماتریس‌ها}

\begin{example}
$M_{m \times n}$ = مجموعه ماتریس‌های $m \times n$

جمع: جمع درایه‌ای

ضرب اسکالری: ضرب همه درایه‌ها در اسکالر

بعد: $m \times n$
\end{example}

% ============================================
\section{تبدیلات خطی انتزاعی}
% ============================================

\begin{definition}[تبدیل خطی بین فضاها]
تابع $T: V \to W$ یک \vocab{تبدیل خطی} است اگر:
\begin{enumerate}
    \item $T(\vu + \vv) = T(\vu) + T(\vv)$
    \item $T(c\vv) = c \cdot T(\vv)$
\end{enumerate}
\end{definition}

\begin{example}
\textbf{مشتق‌گیری:} $D: \mathcal{P}_n \to \mathcal{P}_{n-1}$ با $D(p) = p'$

خطی است زیرا $(f + g)' = f' + g'$ و $(cf)' = cf'$.
\end{example}

\begin{example}
\textbf{انتگرال معین:} $I: C[0,1] \to \R$ با $I(f) = \int_0^1 f(x) dx$

خطی است زیرا $\int(f+g) = \int f + \int g$ و $\int cf = c\int f$.
\end{example}

% ============================================
\section{توابع به عنوان بردار بی‌نهایت بعدی}
% ============================================

\begin{intuition}
یک تابع $f: [0, 2\pi] \to \R$ را می‌توان به عنوان بردار بی‌نهایت بعدی در نظر گرفت:
\begin{itemize}
    \item هر نقطه $x$ یک «مؤلفه» است
    \item مقدار $f(x)$ مقدار آن مؤلفه است
\end{itemize}

ضرب داخلی توابع:
\[
\langle f, g \rangle = \int_0^{2\pi} f(x) g(x) dx
\]
\end{intuition}

\begin{practical}
\textbf{سری فوریه}

توابع سینوسی و کسینوسی یک «پایه» برای توابع تناوبی هستند:
\[
f(x) = \frac{a_0}{2} + \sum_{n=1}^{\infty} (a_n \cos nx + b_n \sin nx)
\]

ضرایب $a_n, b_n$ مثل «مختصات» تابع در این پایه هستند!
\end{practical}

% ============================================
\section{چرا انتزاع مهم است؟}
% ============================================

\begin{summary}
\textbf{قدرت انتزاع:}

وقتی چیزی یک فضای برداری است، \textbf{همه ابزارهای جبر خطی} قابل استفاده‌اند:
\begin{itemize}
    \item استقلال و وابستگی خطی
    \item پایه و بعد
    \item تبدیلات خطی و ماتریس‌ها
    \item مقادیر ویژه و بردارهای ویژه
    \item تصویر و فضای پوچ
\end{itemize}

یک قضیه در جبر خطی = قضیه‌ای برای همه این فضاها!
\end{summary}

\begin{practical}
\textbf{مکانیک کوانتومی}

حالت‌های کوانتومی یک فضای برداری (فضای هیلبرت) تشکیل می‌دهند. عملگرهای فیزیکی تبدیلات خطی هستند. مقادیر اندازه‌گیری = مقادیر ویژه!
\end{practical}

\begin{practical}
\textbf{پردازش سیگنال}

سیگنال‌های صوتی بردارهایی در فضای توابع هستند. تبدیل فوریه تغییر پایه است. فیلترها تبدیلات خطی هستند.
\end{practical}

\begin{practical}
\textbf{یادگیری ماشین}

داده‌ها بردارهایی در فضای ویژگی‌ها هستند. مدل‌های خطی تبدیلات خطی هستند. کاهش بعد = یافتن پایه بهتر.
\end{practical}

% ============================================
\section{نگاه به آینده}
% ============================================

\begin{remark}
این دوره مقدمه‌ای بر جبر خطی بود. موضوعات پیشرفته‌تر:
\begin{itemize}
    \item \textbf{فرم‌های درجه دوم} و دسته‌بندی مقاطع مخروطی
    \item \textbf{تجزیه مقدار تکین (SVD)} - ابزار قدرتمند علم داده
    \item \textbf{جبر خطی عددی} - الگوریتم‌های کارآمد
    \item \textbf{فضاهای هیلبرت} - جبر خطی بی‌نهایت بعدی
    \item \textbf{نظریه نمایش} - گروه‌ها و جبر خطی
\end{itemize}
\end{remark}

% ============================================
\section{تمرین‌ها}
% ============================================

\begin{exercise}
نشان دهید که مجموعه ماتریس‌های $2 \times 2$ متقارن یک فضای برداری است. بعد آن چیست؟
\end{exercise}

\begin{exercise}
آیا مجموعه چندجمله‌ای‌هایی که $p(0) = 1$ یک فضای برداری است؟ چرا؟
\end{exercise}

\begin{exercise}
نشان دهید که مشتق‌گیری $D: \mathcal{P}_3 \to \mathcal{P}_2$ یک تبدیل خطی است. ماتریس آن را در پایه استاندارد بنویسید.
\end{exercise}

\begin{exercise}
ثابت کنید که در هر فضای برداری: $0 \cdot \vv = \vzero$
\end{exercise}

\begin{exercise}[چالشی]
فضای جواب‌های معادله دیفرانسیل $y'' + y = 0$ را در نظر بگیرید. نشان دهید این یک فضای برداری است و پایه‌ای برای آن پیدا کنید.
\end{exercise}

\begin{problem}
برای تبدیل خطی $T: \mathcal{P}_2 \to \mathcal{P}_2$ با $T(p) = p + p'$:
\begin{enumerate}[label=(\alph*)]
    \item ماتریس $T$ را در پایه $\{1, x, x^2\}$ بنویسید
    \item مقادیر ویژه را پیدا کنید
    \item چندجمله‌ای‌های ویژه (بردارهای ویژه) را پیدا کنید
\end{enumerate}
\end{problem}

% ============================================
\section{جمع‌بندی دوره}
% ============================================

\begin{summary}
\textbf{ذات جبر خطی}

از بردارها و ماتریس‌ها شروع کردیم و به فضاهای انتزاعی رسیدیم. ایده‌های کلیدی:

\begin{enumerate}
    \item \textbf{بردار:} چیزی که جمع و ضرب اسکالری دارد
    \item \textbf{تبدیل خطی:} تابعی که خطوط را حفظ می‌کند
    \item \textbf{ماتریس:} نمایش عددی تبدیل خطی
    \item \textbf{دترمینان:} ضریب تغییر حجم
    \item \textbf{مقادیر ویژه:} جهت‌های خاص که فقط مقیاس می‌شوند
    \item \textbf{انتزاع:} همه این مفاهیم فراتر از پیکان‌ها کار می‌کنند
\end{enumerate}

جبر خطی زبان مشترک ریاضیات، فیزیک، مهندسی، و علوم کامپیوتر است.
\end{summary}

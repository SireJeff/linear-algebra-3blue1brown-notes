% lecture08.tex - قاعده کرامر
% Chapter 8: Cramer's Rule

\chapter{قاعده کرامر}
\label{ch:cramer}

\begin{abstract}
قاعده کرامر روشی زیبا برای حل دستگاه معادلات خطی با استفاده از دترمینان است. در این درس با تفسیر هندسی این قاعده و شرایط کاربرد آن آشنا می‌شویم.
\end{abstract}

% ============================================
\section{بیان قاعده کرامر}
% ============================================

\begin{theorem}[قاعده کرامر]
برای دستگاه $\mA\vx = \vb$ با ماتریس $n \times n$ و $\det(\mA) \neq 0$:
\[
x_i = \frac{\det(\mA_i)}{\det(\mA)}
\]
که $\mA_i$ ماتریسی است که ستون $i$ام $\mA$ با بردار $\vb$ جایگزین شده.
\end{theorem}

\begin{example}
دستگاه $2 \times 2$:
\[
\system{
ax + by &= e \\
cx + dy &= f
}
\]
جواب:
\[
x = \frac{\det\twomat{e}{b}{f}{d}}{\det\twomat{a}{b}{c}{d}} = \frac{ed - bf}{ad - bc}
\]
\[
y = \frac{\det\twomat{a}{e}{c}{f}}{\det\twomat{a}{b}{c}{d}} = \frac{af - ec}{ad - bc}
\]
\end{example}

% ============================================
\section{تفسیر هندسی}
% ============================================

\begin{intuition}
در دو بعد، معادله $\mA\vx = \vb$ می‌پرسد:

«چه ترکیب خطی از ستون‌های $\mA$ برابر $\vb$ است؟»

اگر $\mA = [\va_1 \,|\, \va_2]$ و $\vx = \twovec{x}{y}$:
\[
x\va_1 + y\va_2 = \vb
\]
\end{intuition}

\begin{center}
\begin{tikzpicture}[scale=1.3]
    % Parallelogram
    \draw[vector, red!70, very thick] (0,0) -- (2,0.5) node[right] {$\va_1$};
    \draw[vector, blue!70, very thick] (0,0) -- (0.5,2) node[above] {$\va_2$};
    \draw[vector, green!60!black, ultra thick] (0,0) -- (1.5,1.5) node[above right] {$\vb$};

    % Parallelogram formed by b and a2
    \fill[orange!20] (0,0) -- (1.5,1.5) -- (2,3.5) -- (0.5,2) -- cycle;

    % Parallelogram formed by a1 and a2
    \fill[gray!20] (0,0) -- (2,0.5) -- (2.5,2.5) -- (0.5,2) -- cycle;

    \node at (1.2,2.5) {\small مساحت نارنجی};
    \node at (1.8,1) {\small مساحت خاکستری};
\end{tikzpicture}
\end{center}

\begin{theorem}[تفسیر مساحتی]
\[
x = \frac{\text{مساحت متوازی‌الاضلاع}(\vb, \va_2)}{\text{مساحت متوازی‌الاضلاع}(\va_1, \va_2)}
\]
\[
y = \frac{\text{مساحت متوازی‌الاضلاع}(\va_1, \vb)}{\text{مساحت متوازی‌الاضلاع}(\va_1, \va_2)}
\]
\end{theorem}

\begin{intuition}
چرا این کار می‌کند؟

متوازی‌الاضلاع $(\vb, \va_2)$ را در نظر بگیرید. چون $\vb = x\va_1 + y\va_2$:
\[
\text{مساحت}(\vb, \va_2) = \text{مساحت}(x\va_1 + y\va_2, \va_2)
\]
چون $\va_2 \times \va_2 = 0$:
\[
= \text{مساحت}(x\va_1, \va_2) = x \cdot \text{مساحت}(\va_1, \va_2)
\]
پس:
\[
x = \frac{\text{مساحت}(\vb, \va_2)}{\text{مساحت}(\va_1, \va_2)}
\]
\end{intuition}

% ============================================
\section{مثال محاسباتی}
% ============================================

\begin{example}
حل دستگاه:
\[
\system{
3x + 2y &= 7 \\
x + 4y &= 9
}
\]

\textbf{مرحله ۱:} محاسبه $\det(\mA)$:
\[
\det\twomat{3}{2}{1}{4} = 12 - 2 = 10
\]

\textbf{مرحله ۲:} محاسبه $x$:
\[
x = \frac{\det\twomat{7}{2}{9}{4}}{\det(\mA)} = \frac{28 - 18}{10} = \frac{10}{10} = 1
\]

\textbf{مرحله ۳:} محاسبه $y$:
\[
y = \frac{\det\twomat{3}{7}{1}{9}}{\det(\mA)} = \frac{27 - 7}{10} = \frac{20}{10} = 2
\]

\textbf{بررسی:} $3(1) + 2(2) = 7$ ✓ و $1(1) + 4(2) = 9$ ✓
\end{example}

% ============================================
\section{سه بعد و بالاتر}
% ============================================

\begin{example}
دستگاه $3 \times 3$:
\[
\system{
2x + y - z &= 3 \\
x - y + 2z &= 1 \\
3x + 2y + z &= 4
}
\]

ماتریس ضرایب:
\[
\mA = \threemat{2}{1}{-1}{1}{-1}{2}{3}{2}{1}
\]

\[
x = \frac{\det\threemat{3}{1}{-1}{1}{-1}{2}{4}{2}{1}}{\det(\mA)}, \quad
y = \frac{\det\threemat{2}{3}{-1}{1}{1}{2}{3}{4}{1}}{\det(\mA)}, \quad
z = \frac{\det\threemat{2}{1}{3}{1}{-1}{1}{3}{2}{4}}{\det(\mA)}
\]
\end{example}

\begin{intuition}
در سه بعد، به جای نسبت مساحت‌ها، نسبت \textbf{حجم‌ها} را داریم.

$x$ = نسبت حجم متوازی‌السطوح $(\vb, \va_2, \va_3)$ به حجم $(\va_1, \va_2, \va_3)$
\end{intuition}

% ============================================
\section{محدودیت‌ها و کاربردها}
% ============================================

\begin{warning}
قاعده کرامر:
\begin{itemize}
    \item فقط برای دستگاه‌های مربعی ($n$ معادله، $n$ مجهول)
    \item فقط وقتی $\det(\mA) \neq 0$ (جواب یکتا)
    \item برای $n$ بزرگ، محاسباتی \textbf{بسیار پرهزینه} است
\end{itemize}
\end{warning}

\begin{remark}
در عمل برای حل دستگاه‌های بزرگ از روش‌هایی مثل حذف گاوسی استفاده می‌شود که کارآمدتر هستند. اما قاعده کرامر:
\begin{itemize}
    \item درک نظری عمیق‌تری می‌دهد
    \item برای فرمول‌های تحلیلی مفید است
    \item در اثبات قضایا کاربرد دارد
\end{itemize}
\end{remark}

\begin{practical}
\textbf{کاربرد: یافتن تقاطع خطوط}

دو خط $a_1x + b_1y = c_1$ و $a_2x + b_2y = c_2$ در نقطه زیر تقاطع دارند:
\[
x = \frac{c_1 b_2 - c_2 b_1}{a_1 b_2 - a_2 b_1}, \quad y = \frac{a_1 c_2 - a_2 c_1}{a_1 b_2 - a_2 b_1}
\]
این فرمول مستقیم از قاعده کرامر می‌آید.
\end{practical}

% ============================================
\section{تمرین‌ها}
% ============================================

\begin{exercise}
دستگاه زیر را با قاعده کرامر حل کنید:
\[
\system{
2x + 3y &= 8 \\
4x - y &= 2
}
\]
\end{exercise}

\begin{exercise}
دستگاه زیر را با قاعده کرامر حل کنید:
\[
\system{
x + y + z &= 6 \\
2x - y + z &= 3 \\
x + 2y - z &= 2
}
\]
\end{exercise}

\begin{exercise}
نقطه تقاطع دو خط $2x + 3y = 7$ و $x - y = 1$ را با قاعده کرامر پیدا کنید.
\end{exercise}

\begin{exercise}
تفسیر هندسی قاعده کرامر را برای حالتی که $\det(\mA) = 0$ توضیح دهید.
\end{exercise}

\begin{exercise}[چالشی]
نشان دهید که فرمول کرامر با ضرب $\mA\inv \vb$ سازگار است.
\end{exercise}

\begin{problem}
یک مثلث با رأس‌های $A(1,1)$، $B(4,2)$، $C(2,5)$ داده شده. با استفاده از قاعده کرامر، مختصات مرکز ثقل را پیدا کنید.
\end{problem}

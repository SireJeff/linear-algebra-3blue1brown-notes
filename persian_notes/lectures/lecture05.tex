% lecture05.tex - ماتریس معکوس، فضای ستونی و فضای پوچ
% Chapter 5: Inverse Matrices, Column Space, and Null Space

\chapter{ماتریس معکوس، فضای ستونی و فضای پوچ}
\label{ch:inverse}

\begin{abstract}
در این درس با مفهوم ماتریس معکوس، حل دستگاه معادلات خطی، و فضاهای مهم مرتبط با ماتریس (فضای ستونی، فضای پوچ، و رتبه) آشنا می‌شویم.
\end{abstract}

% ============================================
\section{دستگاه معادلات خطی}
% ============================================

\begin{definition}[دستگاه معادلات به فرم ماتریسی]
دستگاه معادلات خطی:
\[
\system{
a_{11}x_1 + a_{12}x_2 + \cdots + a_{1n}x_n &= b_1 \\
a_{21}x_1 + a_{22}x_2 + \cdots + a_{2n}x_n &= b_2 \\
&\vdots \\
a_{m1}x_1 + a_{m2}x_2 + \cdots + a_{mn}x_n &= b_m
}
\]
را می‌توان به صورت $\mA\vx = \vb$ نوشت.
\end{definition}

\begin{intuition}
معادله $\mA\vx = \vb$ را اینگونه تفسیر کنید:
\begin{center}
«چه برداری $\vx$ پس از اعمال تبدیل $\mA$ به $\vb$ می‌رسد؟»
\end{center}
یا به عبارتی: تبدیل $\mA$ را «برعکس» کنید تا از $\vb$ به $\vx$ برسید.
\end{intuition}

% ============================================
\section{ماتریس معکوس}
% ============================================

\begin{definition}[ماتریس معکوس]
ماتریس $\mA\inv$ \vocab{معکوس} ماتریس $\mA$ است اگر:
\[
\mA\inv \mA = \mA \mA\inv = \mI
\]
که $\mI$ ماتریس همانی است.
\end{definition}

\begin{intuition}
اگر $\mA$ یک تبدیل باشد، $\mA\inv$ تبدیلی است که اثر $\mA$ را خنثی می‌کند:
\begin{itemize}
    \item اگر $\mA$ چرخش $90°$ باشد، $\mA\inv$ چرخش $-90°$ است
    \item اگر $\mA$ مقیاس ۲ برابر باشد، $\mA\inv$ مقیاس $\frac{1}{2}$ است
\end{itemize}
\end{intuition}

\begin{theorem}[حل دستگاه با ماتریس معکوس]
اگر $\mA$ معکوس‌پذیر باشد، جواب یکتای $\mA\vx = \vb$:
\[
\vx = \mA\inv \vb
\]
\end{theorem}

\subsection{فرمول معکوس ماتریس $2 \times 2$}

\begin{theorem}
اگر $\mA = \twomat{a}{b}{c}{d}$ و $\det(\mA) \neq 0$:
\[
\mA\inv = \frac{1}{ad-bc} \twomat{d}{-b}{-c}{a}
\]
\end{theorem}

\begin{example}
\[
\mA = \twomat{3}{1}{2}{4}, \quad \det(\mA) = 12 - 2 = 10
\]
\[
\mA\inv = \frac{1}{10} \twomat{4}{-1}{-2}{3} = \twomat{0.4}{-0.1}{-0.2}{0.3}
\]
\end{example}

% ============================================
\section{شرط وجود معکوس}
% ============================================

\begin{theorem}
ماتریس $\mA$ معکوس‌پذیر است اگر و تنها اگر $\det(\mA) \neq 0$.
\end{theorem}

\begin{intuition}
اگر $\det(\mA) = 0$، تبدیل $\mA$ فضا را فشرده می‌کند (مثلاً صفحه به خط). چنین تبدیلی برگشت‌پذیر نیست - اطلاعات از دست رفته قابل بازیابی نیست.

مثل این است که چند عدد را جمع کنید: از حاصل‌جمع نمی‌توانید اعداد اصلی را پیدا کنید.
\end{intuition}

% ============================================
\section{فضای ستونی \lr{(Column Space)}}
% ============================================

\begin{definition}[فضای ستونی]
\vocab{فضای ستونی} ماتریس $\mA$، که با $\Col(\mA)$ نمایش داده می‌شود، فضای پوشش ستون‌های $\mA$ است:
\[
\Col(\mA) = \spn\{\text{ستون‌های } \mA\}
\]
\end{definition}

\begin{intuition}
فضای ستونی = مجموعه همه خروجی‌های ممکن تبدیل $\mA$

سؤال: «معادله $\mA\vx = \vb$ جواب دارد؟» معادل است با «آیا $\vb$ در فضای ستونی $\mA$ است؟»
\end{intuition}

\begin{example}
\[
\mA = \twomat{1}{3}{2}{6}
\]
ستون دوم = ۳ برابر ستون اول، پس:
\[
\Col(\mA) = \spn\left\{\twovec{1}{2}\right\} = \text{یک خط از مبدأ}
\]
\end{example}

% ============================================
\section{فضای پوچ \lr{(Null Space)}}
% ============================================

\begin{definition}[فضای پوچ]
\vocab{فضای پوچ} (یا هسته) ماتریس $\mA$، مجموعه تمام بردارهایی است که $\mA$ آنها را به صفر می‌برد:
\[
\Null(\mA) = \{\vx \mid \mA\vx = \vzero\}
\]
\end{definition}

\begin{intuition}
فضای پوچ = بردارهایی که تبدیل $\mA$ آنها را «له» می‌کند

اگر $\det(\mA) \neq 0$: فقط بردار صفر له می‌شود، پس $\Null(\mA) = \{\vzero\}$

اگر $\det(\mA) = 0$: یک خط یا صفحه کامل به مبدأ فشرده می‌شود
\end{intuition}

\begin{example}
برای ماتریس $\mA = \twomat{1}{2}{2}{4}$:

حل $\mA\vx = \vzero$:
\[
\twomat{1}{2}{2}{4}\twovec{x}{y} = \twovec{0}{0}
\]
معادله: $x + 2y = 0$، پس $x = -2y$

فضای پوچ: $\Null(\mA) = \left\{ t\twovec{-2}{1} \mid t \in \R \right\}$ (یک خط)
\end{example}

% ============================================
\section{رتبه \lr{(Rank)}}
% ============================================

\begin{definition}[رتبه]
\vocab{رتبه} ماتریس $\mA$ برابر است با:
\begin{itemize}
    \item بعد فضای ستونی
    \item تعداد ستون‌های مستقل خطی
    \item تعداد سطرهای مستقل خطی
\end{itemize}
\[
\rank(\mA) = \dim(\Col(\mA))
\]
\end{definition}

\begin{theorem}[قضیه رتبه-پوچی]
برای ماتریس $m \times n$:
\[
\rank(\mA) + \dim(\Null(\mA)) = n
\]
(تعداد ستون‌ها)
\end{theorem}

\begin{intuition}
رتبه = ابعادی که تبدیل حفظ می‌کند

$\dim(\Null(\mA))$ = ابعادی که از دست می‌رود

جمع این دو = ابعاد فضای ورودی
\end{intuition}

% ============================================
\section{ماتریس‌های غیرمربعی}
% ============================================

\begin{definition}[تبدیل بین ابعاد مختلف]
ماتریس $m \times n$ تبدیلی از $\R^n$ به $\R^m$ را نمایش می‌دهد:
\begin{itemize}
    \item $m > n$: تبدیل از بعد کمتر به بعد بیشتر
    \item $m < n$: تبدیل از بعد بیشتر به بعد کمتر
\end{itemize}
\end{definition}

\begin{example}
ماتریس $2 \times 3$:
\[
\mA = \twomat{1}{0}{2}{0}{1}{-1}
\]
تبدیلی از $\Rthree$ به $\Rtwo$ است. فضای سه‌بعدی را روی صفحه «پروژه» می‌کند.
\end{example}

% ============================================
\section{حالت‌های مختلف دستگاه معادلات}
% ============================================

\begin{theorem}[تحلیل جواب دستگاه $\mA\vx = \vb$]
\begin{enumerate}
    \item \textbf{جواب یکتا:} اگر $\det(\mA) \neq 0$
    \item \textbf{بی‌نهایت جواب:} اگر $\det(\mA) = 0$ و $\vb \in \Col(\mA)$
    \item \textbf{بدون جواب:} اگر $\vb \notin \Col(\mA)$
\end{enumerate}
\end{theorem}

\begin{center}
\begin{tikzpicture}[scale=0.9]
    % Case 1: Unique solution
    \begin{scope}[shift={(0,0)}]
        \fill[blue!10] (-1.5,-1.5) rectangle (1.5,1.5);
        \draw[axis] (-1.5,0) -- (1.5,0);
        \draw[axis] (0,-1.5) -- (0,1.5);
        \fill[red] (0.8,0.6) circle (2pt) node[above right] {$\vx$};
        \node at (0,-2) {جواب یکتا};
        \node at (0,-2.5) {\small $\det \neq 0$};
    \end{scope}

    % Case 2: Infinite solutions
    \begin{scope}[shift={(5,0)}]
        \draw[blue!30, very thick] (-1.5,-0.75) -- (1.5,0.75);
        \draw[axis] (-1.5,0) -- (1.5,0);
        \draw[axis] (0,-1.5) -- (0,1.5);
        \node at (0,-2) {بی‌نهایت جواب};
        \node at (0,-2.5) {\small $\det = 0$, $\vb \in \Col$};
    \end{scope}

    % Case 3: No solution
    \begin{scope}[shift={(10,0)}]
        \draw[blue!30, very thick] (-1.5,-0.75) -- (1.5,0.75);
        \draw[axis] (-1.5,0) -- (1.5,0);
        \draw[axis] (0,-1.5) -- (0,1.5);
        \fill[red] (0.5,1.2) circle (2pt) node[right] {$\vb$};
        \node[red] at (0.5,0.8) {$\times$};
        \node at (0,-2) {بدون جواب};
        \node at (0,-2.5) {\small $\vb \notin \Col$};
    \end{scope}
\end{tikzpicture}
\end{center}

% ============================================
\section{تمرین‌ها}
% ============================================

\begin{exercise}
معکوس ماتریس زیر را محاسبه کنید:
\[
\mA = \twomat{2}{5}{1}{3}
\]
\end{exercise}

\begin{exercise}
دستگاه زیر را با استفاده از ماتریس معکوس حل کنید:
\[
\system{
2x + 3y &= 7 \\
x + 2y &= 4
}
\]
\end{exercise}

\begin{exercise}
فضای پوچ ماتریس زیر را پیدا کنید:
\[
\mA = \twomat{1}{-2}{-3}{6}
\]
\end{exercise}

\begin{exercise}
رتبه ماتریس زیر را تعیین کنید:
\[
\mA = \threemat{1}{2}{3}{2}{4}{6}{1}{1}{1}
\]
\end{exercise}

\begin{exercise}[چالشی]
نشان دهید که $(\mA\mB)\inv = \mB\inv\mA\inv$ (اگر معکوس‌ها وجود داشته باشند).
\end{exercise}

\begin{problem}
برای چه مقادیر $k$، ماتریس زیر معکوس‌پذیر نیست؟
\[
\mA = \twomat{k}{2}{3}{k}
\]
\end{problem}

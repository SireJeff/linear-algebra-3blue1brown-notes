% lecture03.tex - تبدیلات خطی و ماتریس‌ها
% Chapter 3: Linear Transformations and Matrices

\chapter{تبدیلات خطی و ماتریس‌ها}
\label{ch:transformations}

\begin{abstract}
تبدیلات خطی قلب جبر خطی هستند. در این درس می‌آموزیم که چگونه هر تبدیل خطی را می‌توان با یک ماتریس نمایش داد، و ضرب ماتریسی چگونه با ترکیب تبدیلات مرتبط است.
\end{abstract}

% ============================================
\section{تبدیل چیست؟}
% ============================================

\begin{definition}[تبدیل]
\vocab{تبدیل} \lr{(Transformation)} تابعی است که بردارها را به بردارهای دیگر می‌برد:
\[
T: \Rtwo \to \Rtwo
\]
یعنی هر بردار ورودی را به یک بردار خروجی نگاشت می‌کند.
\end{definition}

\begin{intuition}
به جای «تابع» از واژه «تبدیل» استفاده می‌کنیم چون می‌خواهیم به \textbf{حرکت} فکر کنیم. تصور کنید هر بردار در فضا به جایی جدید «منتقل» می‌شود.

شبکه خطوط مختصات را تصور کنید. یک تبدیل این شبکه را می‌کشد، می‌فشارد، می‌چرخاند، یا به شکل دیگری تغییر می‌دهد.
\end{intuition}

% ============================================
\section{تبدیل خطی}
% ============================================

\begin{definition}[تبدیل خطی]
تبدیل $T$ یک \vocab{تبدیل خطی} است اگر و تنها اگر دو شرط زیر برقرار باشد:
\begin{enumerate}
    \item \textbf{جمع‌پذیری:} $T(\vv + \vw) = T(\vv) + T(\vw)$
    \item \textbf{همگنی:} $T(c\vv) = c \cdot T(\vv)$
\end{enumerate}
برای هر بردارهای $\vv, \vw$ و هر اسکالر $c$.
\end{definition}

\begin{intuition}
یک تبدیل خطی را می‌توان با دو ویژگی هندسی شناخت:
\begin{enumerate}
    \item \textbf{خطوط راست، راست می‌مانند} (خم نمی‌شوند)
    \item \textbf{مبدأ در جای خود می‌ماند}
\end{enumerate}

اگر خطوط شبکه بعد از تبدیل همچنان موازی و با فاصله یکنواخت باقی بمانند، تبدیل خطی است.
\end{intuition}

\begin{center}
\begin{tikzpicture}[scale=0.8]
    % Original grid
    \begin{scope}[shift={(0,0)}]
        \draw[grid] (-2,-2) grid (2,2);
        \draw[axis] (-2,0) -- (2,0);
        \draw[axis] (0,-2) -- (0,2);
        \draw[basis, -{Stealth[length=3mm]}] (0,0) -- (1,0) node[below] {$\vi$};
        \draw[basis, -{Stealth[length=3mm]}] (0,0) -- (0,1) node[left] {$\vj$};
        \node at (0,-3) {قبل از تبدیل};
    \end{scope}

    % Arrow
    \draw[-{Stealth[length=5mm]}, very thick] (3,0) -- (5,0) node[midway, above] {$T$};

    % Transformed grid (shear)
    \begin{scope}[shift={(8,0)}]
        \draw[blue!30, thin] (-2,-2) -- (0,2) -- (4,2) -- (2,-2) -- cycle;
        \foreach \y in {-1,0,1} {
            \draw[blue!30, thin] ({-2+\y},-2) -- ({\y+2},2);
        }
        \foreach \x in {-2,-1,0,1,2} {
            \draw[blue!30, thin] ({\x-1},-2) -- ({\x+1},2);
        }
        \draw[axis] (-2,0) -- (4,0);
        \draw[axis] (0,-2) -- (2,2);
        \draw[transformed, -{Stealth[length=3mm]}, very thick] (0,0) -- (1,0) node[below] {$T(\vi)$};
        \draw[transformed, -{Stealth[length=3mm]}, very thick] (0,0) -- (1,1) node[above left] {$T(\vj)$};
        \node at (1,-3) {بعد از تبدیل};
    \end{scope}
\end{tikzpicture}
\end{center}

% ============================================
\section{ماتریس یک تبدیل خطی}
% ============================================

\begin{theorem}[قضیه اساسی]
هر تبدیل خطی $T: \Rtwo \to \Rtwo$ کاملاً با دانستن اینکه $T$ بردارهای پایه $\vi$ و $\vj$ را به کجا می‌برد، مشخص می‌شود.
\end{theorem}

\begin{proof}
هر بردار $\vv = \twovec{x}{y}$ را می‌توان نوشت: $\vv = x\vi + y\vj$

با استفاده از خطی بودن:
\[
T(\vv) = T(x\vi + y\vj) = xT(\vi) + yT(\vj)
\]
پس اگر $T(\vi)$ و $T(\vj)$ را بدانیم، می‌توانیم $T(\vv)$ را برای هر $\vv$ محاسبه کنیم.
\end{proof}

\begin{definition}[ماتریس تبدیل]
اگر $T(\vi) = \twovec{a}{c}$ و $T(\vj) = \twovec{b}{d}$ باشد، \vocab{ماتریس تبدیل} $T$ برابر است با:
\[
\mA = \twomat{a}{b}{c}{d}
\]
ستون اول محل فرود $\vi$ و ستون دوم محل فرود $\vj$ است.
\end{definition}

\begin{important}
\textbf{قانون طلایی:} ستون‌های ماتریس = محل فرود بردارهای پایه
\end{important}

% ============================================
\section{ضرب ماتریس در بردار}
% ============================================

\begin{definition}[ضرب ماتریس در بردار]
\[
\twomat{a}{b}{c}{d} \twovec{x}{y} = x\twovec{a}{c} + y\twovec{b}{d} = \twovec{ax + by}{cx + dy}
\]
\end{definition}

\begin{intuition}
ضرب ماتریس در بردار یعنی:
\begin{enumerate}
    \item مؤلفه $x$ بردار ورودی را در ستون اول ضرب کن
    \item مؤلفه $y$ بردار ورودی را در ستون دوم ضرب کن
    \item نتایج را جمع کن
\end{enumerate}
این دقیقاً همان ترکیب خطی بردارهای پایه جدید است!
\end{intuition}

\begin{example}
تبدیل چرخش $90°$ پادساعتگرد:
\[
T(\vi) = \twovec{0}{1}, \quad T(\vj) = \twovec{-1}{0}
\]
پس ماتریس چرخش:
\[
\mR = \twomat{0}{-1}{1}{0}
\]
بررسی: $\mR\twovec{1}{0} = \twovec{0}{1}$ ✓
\end{example}

% ============================================
\section{مثال‌های تبدیلات مهم}
% ============================================

\subsection{چرخش \lr{(Rotation)}}

\begin{definition}[ماتریس چرخش]
چرخش به اندازه زاویه $\theta$ پادساعتگرد:
\[
\mR_\theta = \twomat{\cos\theta}{-\sin\theta}{\sin\theta}{\cos\theta}
\]
\end{definition}

\begin{example}
چرخش $45°$:
\[
\mR_{45°} = \twomat{\frac{\sqrt{2}}{2}}{-\frac{\sqrt{2}}{2}}{\frac{\sqrt{2}}{2}}{\frac{\sqrt{2}}{2}}
\]
\end{example}

\subsection{مقیاس‌گذاری \lr{(Scaling)}}

\begin{definition}[ماتریس مقیاس]
مقیاس‌گذاری با ضرایب $s_x$ در راستای $x$ و $s_y$ در راستای $y$:
\[
\mS = \twomat{s_x}{0}{0}{s_y}
\]
\end{definition}

\begin{practical}
\textbf{کاربرد در گرافیک کامپیوتری:} برای بزرگ یا کوچک کردن تصویر:
\begin{itemize}
    \item $s_x = s_y = 2$: تصویر دو برابر بزرگ می‌شود
    \item $s_x = 1, s_y = 0.5$: تصویر در راستای عمودی فشرده می‌شود
\end{itemize}
\end{practical}

\subsection{برش \lr{(Shear)}}

\begin{definition}[ماتریس برش افقی]
\[
\mH = \twomat{1}{k}{0}{1}
\]
که در آن $k$ مقدار برش است.
\end{definition}

\begin{center}
\begin{tikzpicture}[scale=1]
    % Original square
    \begin{scope}[shift={(0,0)}]
        \draw[fill=blue!20] (0,0) -- (1,0) -- (1,1) -- (0,1) -- cycle;
        \draw[basis] (0,0) -- (1,0) node[midway, below] {$\vi$};
        \draw[basis] (0,0) -- (0,1) node[midway, left] {$\vj$};
        \node at (0.5,-0.7) {اصلی};
    \end{scope}

    % Sheared
    \begin{scope}[shift={(4,0)}]
        \draw[fill=blue!20] (0,0) -- (1,0) -- (2,1) -- (1,1) -- cycle;
        \draw[transformed, very thick] (0,0) -- (1,0) node[midway, below] {$T(\vi)$};
        \draw[transformed, very thick] (0,0) -- (1,1) node[midway, left] {$T(\vj)$};
        \node at (1,-0.7) {برش با $k=1$};
    \end{scope}
\end{tikzpicture}
\end{center}

\subsection{انعکاس \lr{(Reflection)}}

\begin{example}
انعکاس نسبت به محور $x$:
\[
\twomat{1}{0}{0}{-1}
\]

انعکاس نسبت به خط $y = x$:
\[
\twomat{0}{1}{1}{0}
\]
\end{example}

% ============================================
\section{ضرب ماتریس‌ها = ترکیب تبدیلات}
% ============================================

\begin{theorem}[ترکیب تبدیلات]
اگر $\mA$ ماتریس تبدیل $T_1$ و $\mB$ ماتریس تبدیل $T_2$ باشد، آنگاه:
\[
\mB \cdot \mA = \text{ماتریس تبدیل } (T_2 \circ T_1)
\]
یعنی اول $T_1$ و سپس $T_2$ اعمال شود.
\end{theorem}

\begin{warning}
ترتیب مهم است! $\mA\mB \neq \mB\mA$ در حالت کلی.

اول چرخش و بعد برش $\neq$ اول برش و بعد چرخش
\end{warning}

\begin{definition}[ضرب دو ماتریس]
\[
\twomat{a}{b}{c}{d} \twomat{e}{f}{g}{h} = \twomat{ae+bg}{af+bh}{ce+dg}{cf+dh}
\]
\end{definition}

\begin{intuition}
برای محاسبه ستون‌های $\mB\mA$:
\begin{itemize}
    \item ستون اول: $\mB$ ضرب در ستون اول $\mA$ = محل فرود $\vi$ پس از هر دو تبدیل
    \item ستون دوم: $\mB$ ضرب در ستون دوم $\mA$ = محل فرود $\vj$ پس از هر دو تبدیل
\end{itemize}
\end{intuition}

\begin{example}
چرخش $90°$ و سپس برش:
\[
\underbrace{\twomat{1}{1}{0}{1}}_{\text{برش}} \underbrace{\twomat{0}{-1}{1}{0}}_{\text{چرخش}} = \twomat{1}{-1}{1}{0}
\]

بررسی: $\vi \to \twovec{0}{1} \to \twovec{1}{1}$ ✓
\end{example}

% ============================================
\section{کاربردهای عملی}
% ============================================

\begin{practical}
\textbf{گرافیک کامپیوتری و بازی‌های ویدیویی}

هر شیء در بازی با مجموعه‌ای از بردارها (رأس‌ها) توصیف می‌شود. برای حرکت، چرخش، یا تغییر اندازه شیء، کافی است ماتریس مناسب را در همه رأس‌ها ضرب کنیم.

یک انیمیشن = دنباله‌ای از ضرب ماتریس‌ها
\end{practical}

\begin{practical}
\textbf{پردازش تصویر}

فیلترهای تصویر مثل تار کردن، تیز کردن، و تشخیص لبه با ضرب ماتریسی پیاده‌سازی می‌شوند.
\end{practical}

% ============================================
\section{تمرین‌ها}
% ============================================

\begin{exercise}
ماتریس تبدیلی که $\vi$ را به $\twovec{2}{1}$ و $\vj$ را به $\twovec{-1}{3}$ می‌برد، بنویسید.
\end{exercise}

\begin{exercise}
حاصل ضرب زیر را محاسبه کنید:
\[
\twomat{1}{2}{3}{4} \twovec{5}{6}
\]
\end{exercise}

\begin{exercise}
ماتریس چرخش $180°$ را بنویسید و نشان دهید که برابر است با $-\mI$.
\end{exercise}

\begin{exercise}
دو ماتریس زیر را در هم ضرب کنید (به هر دو ترتیب) و نشان دهید که $\mA\mB \neq \mB\mA$:
\[
\mA = \twomat{1}{2}{0}{1}, \quad \mB = \twomat{0}{1}{1}{0}
\]
\end{exercise}

\begin{exercise}[چالشی]
ماتریسی پیدا کنید که انعکاس نسبت به خط $y = 2x$ باشد.

\textit{راهنمایی: این خط با محور $x$ زاویه $\arctan(2)$ می‌سازد.}
\end{exercise}

\begin{problem}
نشان دهید که ترکیب دو چرخش با زاویه‌های $\alpha$ و $\beta$ برابر است با چرخش به زاویه $\alpha + \beta$.
\end{problem}

\begin{problem}
اگر $\mA^2 = \mA$ (یعنی $\mA$ یک ماتریس تصویر باشد)، چه تفسیر هندسی‌ای دارد؟
\end{problem}

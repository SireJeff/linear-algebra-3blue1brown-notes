% lecture07.tex - ضرب خارجی و کاربردها
% Chapter 7: Cross Products and Applications

\chapter{ضرب خارجی و کاربردها}
\label{ch:crossproduct}

\begin{abstract}
ضرب خارجی (ضرب برداری) عملیاتی است که از دو بردار در فضای سه‌بعدی، برداری عمود بر هر دو می‌سازد. در این درس تعریف، خواص، و ارتباط عمیق ضرب خارجی با دترمینان را بررسی می‌کنیم.
\end{abstract}

% ============================================
\section{ضرب خارجی در دو بعد}
% ============================================

\begin{definition}[ضرب خارجی دوبعدی]
برای دو بردار در $\Rtwo$:
\[
\vv \times \vw = v_1 w_2 - v_2 w_1 = \det\twomat{v_1}{w_1}{v_2}{w_2}
\]
نتیجه یک \textbf{عدد} است (نه بردار).
\end{definition}

\begin{intuition}
ضرب خارجی دوبعدی = مساحت متوازی‌الاضلاع ساخته شده از دو بردار (با علامت)

علامت مثبت: $\vw$ در سمت چپ $\vv$ قرار دارد
علامت منفی: $\vw$ در سمت راست $\vv$ قرار دارد
\end{intuition}

\begin{center}
\begin{tikzpicture}[scale=1.2]
    % Parallelogram
    \fill[blue!20] (0,0) -- (2,0.5) -- (3,2) -- (1,1.5) -- cycle;
    \draw[vector, red!70, very thick] (0,0) -- (2,0.5) node[midway, below] {$\vv$};
    \draw[vector, green!60!black, very thick] (0,0) -- (1,1.5) node[midway, left] {$\vw$};
    \draw[dashed] (2,0.5) -- (3,2);
    \draw[dashed] (1,1.5) -- (3,2);
    \node at (1.5,1) {مساحت $= |\vv \times \vw|$};
\end{tikzpicture}
\end{center}

% ============================================
\section{ضرب خارجی در سه بعد}
% ============================================

\begin{definition}[ضرب خارجی سه‌بعدی]
برای $\vv = \threevec{v_1}{v_2}{v_3}$ و $\vw = \threevec{w_1}{w_2}{w_3}$:
\[
\vv \times \vw = \threevec{v_2 w_3 - v_3 w_2}{v_3 w_1 - v_1 w_3}{v_1 w_2 - v_2 w_1}
\]
\end{definition}

\begin{theorem}[فرمول دترمینان]
\[
\vv \times \vw = \det\threemat{\vi}{\vj}{\vk}{v_1}{v_2}{v_3}{w_1}{w_2}{w_3}
\]
(بسط نمادین نسبت به سطر اول)
\end{theorem}

\begin{intuition}
ضرب خارجی $\vv \times \vw$:
\begin{itemize}
    \item \textbf{جهت:} عمود بر هر دو بردار (قاعده دست راست)
    \item \textbf{اندازه:} مساحت متوازی‌الاضلاع ساخته شده از $\vv$ و $\vw$
\end{itemize}
\end{intuition}

\begin{center}
\begin{tikzpicture}[scale=1]
    % 3D coordinate hint
    \draw[axis] (0,0) -- (3,0) node[right] {$x$};
    \draw[axis] (0,0) -- (0,3) node[above] {$z$};
    \draw[axis] (0,0) -- (-1,-0.7) node[below left] {$y$};

    % Vectors
    \draw[vector, red!70, very thick] (0,0) -- (2,-0.5) node[right] {$\vv$};
    \draw[vector, green!60!black, very thick] (0,0) -- (-0.5,0) node[left] {$\vw$};
    \draw[vector, blue!70, ultra thick] (0,0) -- (0,2) node[above] {$\vv \times \vw$};

    % Parallelogram base
    \fill[gray!20] (0,0) -- (2,-0.5) -- (1.5,-0.5) -- (-0.5,0) -- cycle;
\end{tikzpicture}
\end{center}

% ============================================
\section{خواص ضرب خارجی}
% ============================================

\begin{theorem}[خواص اصلی]
\begin{enumerate}
    \item \textbf{پادجابجایی:} $\vv \times \vw = -(\vw \times \vv)$
    \item \textbf{توزیع‌پذیری:} $\vv \times (\vw + \vu) = \vv \times \vw + \vv \times \vu$
    \item \textbf{همگنی:} $(c\vv) \times \vw = c(\vv \times \vw)$
    \item \textbf{عمود بودن:} $\vv \times \vw \perp \vv$ و $\vv \times \vw \perp \vw$
    \item \textbf{صفر شدن:} $\vv \times \vv = \vzero$
\end{enumerate}
\end{theorem}

\begin{warning}
ضرب خارجی \textbf{جابجایی ندارد}!
\[
\vv \times \vw \neq \vw \times \vv
\]
در واقع برعکس هم هستند.
\end{warning}

\begin{theorem}[فرمول اندازه]
\[
\norm{\vv \times \vw} = \norm{\vv} \norm{\vw} \sin\theta
\]
که $\theta$ زاویه بین دو بردار است.
\end{theorem}

% ============================================
\section{ضرب خارجی از دیدگاه تبدیلات خطی}
% ============================================

\begin{intuition}
نگاه عمیق‌تر: ضرب خارجی را می‌توان از طریق دوگانگی تعریف کرد.

تابع $f(\vx) = \det[\vv \,|\, \vw \,|\, \vx]$ یک تابع خطی از $\vx$ است. طبق دوگانگی، باید برداری $\vp$ وجود داشته باشد که:
\[
f(\vx) = \vp \cdot \vx
\]
این $\vp$ همان $\vv \times \vw$ است!
\end{intuition}

\begin{theorem}
\[
\det[\vv \,|\, \vw \,|\, \vx] = (\vv \times \vw) \cdot \vx
\]
\end{theorem}

% ============================================
\section{ضرب‌های پایه}
% ============================================

\begin{theorem}[ضرب خارجی بردارهای پایه]
\begin{align*}
\vi \times \vj &= \vk & \vj \times \vi &= -\vk \\
\vj \times \vk &= \vi & \vk \times \vj &= -\vi \\
\vk \times \vi &= \vj & \vi \times \vk &= -\vj
\end{align*}
\end{theorem}

\begin{example}
\[
\threevec{2}{3}{4} \times \threevec{5}{6}{7}
\]
\begin{align*}
&= (3 \times 7 - 4 \times 6)\vi - (2 \times 7 - 4 \times 5)\vj + (2 \times 6 - 3 \times 5)\vk \\
&= (21-24)\vi - (14-20)\vj + (12-15)\vk \\
&= -3\vi + 6\vj - 3\vk = \threevec{-3}{6}{-3}
\end{align*}
\end{example}

% ============================================
\section{کاربردها}
% ============================================

\begin{practical}
\textbf{فیزیک: گشتاور}

گشتاور نیروی $\vec{F}$ حول نقطه‌ای به فاصله $\vec{r}$:
\[
\vec{\tau} = \vec{r} \times \vec{F}
\]
جهت گشتاور عمود بر صفحه‌ای است که نیرو و بازو در آن قرار دارند.
\end{practical}

\begin{practical}
\textbf{فیزیک: نیروی لورنتس}

نیروی وارد بر ذره باردار متحرک در میدان مغناطیسی:
\[
\vec{F} = q\vec{v} \times \vec{B}
\]
\end{practical}

\begin{practical}
\textbf{گرافیک کامپیوتری: بردار نرمال}

برای یافتن بردار عمود بر یک سطح مثلثی با رأس‌های $A$، $B$، $C$:
\[
\vec{n} = (\vec{B} - \vec{A}) \times (\vec{C} - \vec{A})
\]
\end{practical}

\begin{practical}
\textbf{محاسبه مساحت مثلث}

مساحت مثلث با رأس‌های $A$، $B$، $C$:
\[
\text{مساحت} = \frac{1}{2}\norm{(\vec{B}-\vec{A}) \times (\vec{C}-\vec{A})}
\]
\end{practical}

% ============================================
\section{ضرب سه‌گانه}
% ============================================

\begin{definition}[ضرب سه‌گانه اسکالر]
\[
\va \cdot (\vb \times \vc) = \det[\va \,|\, \vb \,|\, \vc]
\]
نتیجه = حجم متوازی‌السطوح ساخته شده از سه بردار (با علامت)
\end{definition}

\begin{theorem}[خاصیت دوری]
\[
\va \cdot (\vb \times \vc) = \vb \cdot (\vc \times \va) = \vc \cdot (\va \times \vb)
\]
\end{theorem}

% ============================================
\section{تمرین‌ها}
% ============================================

\begin{exercise}
ضرب خارجی بردارهای زیر را محاسبه کنید:
\[
\va = \threevec{1}{2}{3}, \quad \vb = \threevec{4}{5}{6}
\]
\end{exercise}

\begin{exercise}
مساحت متوازی‌الاضلاع با اضلاع $\va = \threevec{1}{0}{0}$ و $\vb = \threevec{1}{1}{0}$ را پیدا کنید.
\end{exercise}

\begin{exercise}
نشان دهید که $\va \times \vb$ بر هر دو بردار $\va$ و $\vb$ عمود است.
\end{exercise}

\begin{exercise}
بردار نرمال به صفحه‌ای که از نقاط $(1,0,0)$، $(0,1,0)$، $(0,0,1)$ می‌گذرد را پیدا کنید.
\end{exercise}

\begin{exercise}[چالشی]
نشان دهید که:
\[
\norm{\va \times \vb}^2 + (\va \cdot \vb)^2 = \norm{\va}^2 \norm{\vb}^2
\]
(این رابطه لاگرانژ نام دارد)
\end{exercise}

\begin{problem}
حجم متوازی‌السطوح با اضلاع:
\[
\va = \threevec{1}{1}{0}, \quad \vb = \threevec{0}{1}{1}, \quad \vc = \threevec{1}{0}{1}
\]
را محاسبه کنید.
\end{problem}

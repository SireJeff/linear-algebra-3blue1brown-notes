% lecture04.tex - تبدیلات در سه بعد و دترمینان
% Chapter 4: 3D Transformations and Determinants

\chapter{تبدیلات سه‌بعدی و دترمینان}
\label{ch:determinant}

\begin{abstract}
دترمینان عددی است که میزان «کشیدگی» یا «فشردگی» فضا توسط یک تبدیل خطی را اندازه می‌گیرد. در این درس معنای هندسی دترمینان و نحوه محاسبه آن را می‌آموزیم.
\end{abstract}

% ============================================
\section{تبدیلات در فضای سه‌بعدی}
% ============================================

\begin{definition}[ماتریس تبدیل سه‌بعدی]
یک تبدیل خطی در $\Rthree$ با ماتریس $3 \times 3$ نمایش داده می‌شود:
\[
\mA = \threemat{a_{11}}{a_{12}}{a_{13}}{a_{21}}{a_{22}}{a_{23}}{a_{31}}{a_{32}}{a_{33}}
\]
ستون‌ها محل فرود $\vi$، $\vj$، و $\vk$ هستند.
\end{definition}

\begin{intuition}
همان منطق دوبعدی در سه بعد هم کار می‌کند. شبکه سه‌بعدی را تصور کنید که کشیده، فشرده، چرخیده، یا برش می‌خورد. خطوط همچنان راست می‌مانند و مبدأ ثابت است.
\end{intuition}

% ============================================
\section{دترمینان: معنای هندسی}
% ============================================

\begin{definition}[دترمینان - تعریف هندسی]
\vocab{دترمینان} یک ماتریس، ضریبی است که نشان می‌دهد تبدیل متناظر، مساحت (در دو بعد) یا حجم (در سه بعد) را چند برابر می‌کند.

اگر $\mA$ ماتریس تبدیل $T$ باشد:
\[
\det(\mA) = \frac{\text{مساحت/حجم بعد از تبدیل}}{\text{مساحت/حجم قبل از تبدیل}}
\]
\end{definition}

\begin{center}
\begin{tikzpicture}[scale=1.2]
    % Original unit square
    \begin{scope}[shift={(0,0)}]
        \fill[blue!20] (0,0) -- (1,0) -- (1,1) -- (0,1) -- cycle;
        \draw[thick] (0,0) -- (1,0) -- (1,1) -- (0,1) -- cycle;
        \draw[basis] (0,0) -- (1,0);
        \draw[basis] (0,0) -- (0,1);
        \node at (0.5,0.5) {$1$};
        \node at (0.5,-0.5) {مساحت = $1$};
    \end{scope}

    \draw[-{Stealth}, very thick] (1.5,0.5) -- (2.5,0.5) node[midway, above] {$T$};

    % Transformed parallelogram
    \begin{scope}[shift={(3,0)}]
        \fill[green!20] (0,0) -- (2,0) -- (3,1.5) -- (1,1.5) -- cycle;
        \draw[thick] (0,0) -- (2,0) -- (3,1.5) -- (1,1.5) -- cycle;
        \draw[transformed, very thick] (0,0) -- (2,0);
        \draw[transformed, very thick] (0,0) -- (1,1.5);
        \node at (1.5,0.75) {$3$};
        \node at (1.5,-0.5) {مساحت = $3$};
    \end{scope}
\end{tikzpicture}
\end{center}

\begin{important}
دترمینان = ضریب تغییر مساحت/حجم

اگر $\det(\mA) = 3$، هر شکل پس از تبدیل، $3$ برابر بزرگ‌تر می‌شود.
\end{important}

% ============================================
\section{علامت دترمینان}
% ============================================

\begin{theorem}[معنای علامت دترمینان]
\begin{itemize}
    \item $\det(\mA) > 0$: جهت‌گیری فضا حفظ می‌شود
    \item $\det(\mA) < 0$: جهت‌گیری فضا معکوس می‌شود (مثل آینه)
    \item $\det(\mA) = 0$: فضا به بعد پایین‌تر فشرده می‌شود
\end{itemize}
\end{theorem}

\begin{intuition}
در دو بعد، اگر $\vj$ نسبت به $\vi$ در سمت چپ باشد، جهت‌گیری «مثبت» است. اگر تبدیل این رابطه را عوض کند (مثلاً انعکاس)، دترمینان منفی می‌شود.

در سه بعد، قاعده دست راست: اگر انگشتان از $\vi$ به $\vj$ بچرخند و شست به $\vk$ اشاره کند، جهت‌گیری مثبت است.
\end{intuition}

% ============================================
\section{محاسبه دترمینان}
% ============================================

\subsection{دترمینان ماتریس $2 \times 2$}

\begin{definition}
\[
\det\twomat{a}{b}{c}{d} = ad - bc
\]
\end{definition}

\begin{example}
\[
\det\twomat{3}{1}{0}{2} = 3 \times 2 - 1 \times 0 = 6
\]
مساحت هر شکل ۶ برابر می‌شود.
\end{example}

\subsection{دترمینان ماتریس $3 \times 3$}

\begin{definition}[قاعده ساروس یا بسط]
\[
\det\threemat{a}{b}{c}{d}{e}{f}{g}{h}{i} = a(ei-fh) - b(di-fg) + c(dh-eg)
\]
\end{definition}

\begin{intuition}
دترمینان $3 \times 3$ برابر است با حجم متوازی‌السطوح ساخته شده از سه بردار ستونی ماتریس (با علامت).
\end{intuition}

\begin{example}
\[
\det\threemat{1}{2}{3}{4}{5}{6}{7}{8}{9}
\]
\begin{align*}
&= 1(5 \times 9 - 6 \times 8) - 2(4 \times 9 - 6 \times 7) + 3(4 \times 8 - 5 \times 7) \\
&= 1(45-48) - 2(36-42) + 3(32-35) \\
&= 1(-3) - 2(-6) + 3(-3) \\
&= -3 + 12 - 9 = 0
\end{align*}
دترمینان صفر یعنی سه ستون در یک صفحه هستند!
\end{example}

% ============================================
\section{خواص دترمینان}
% ============================================

\begin{theorem}[خواص اصلی دترمینان]
\begin{enumerate}
    \item $\det(\mI) = 1$
    \item $\det(\mA\mB) = \det(\mA) \cdot \det(\mB)$
    \item $\det(\mA\trans) = \det(\mA)$
    \item $\det(c\mA) = c^n \det(\mA)$ برای ماتریس $n \times n$
    \item اگر یک سطر/ستون صفر باشد: $\det(\mA) = 0$
    \item اگر دو سطر/ستون برابر باشند: $\det(\mA) = 0$
\end{enumerate}
\end{theorem}

\begin{intuition}
خاصیت $\det(\mA\mB) = \det(\mA) \cdot \det(\mB)$ بسیار مهم است:

اگر $\mA$ مساحت را ۳ برابر کند و $\mB$ مساحت را ۲ برابر کند، ترکیب آنها مساحت را $3 \times 2 = 6$ برابر می‌کند.
\end{intuition}

% ============================================
\section{دترمینان صفر: فشردگی فضا}
% ============================================

\begin{theorem}
$\det(\mA) = 0$ اگر و تنها اگر:
\begin{itemize}
    \item ستون‌های $\mA$ وابسته خطی باشند
    \item تبدیل متناظر، فضا را به بعد پایین‌تر ببرد
\end{itemize}
\end{theorem}

\begin{center}
\begin{tikzpicture}[scale=1]
    % 2D to 1D
    \begin{scope}[shift={(0,0)}]
        \fill[blue!20] (0,0) -- (1,0) -- (1,1) -- (0,1) -- cycle;
        \node at (0.5,-0.5) {$\Rtwo$};
    \end{scope}

    \draw[-{Stealth}, very thick] (1.5,0.5) -- (3,0.5) node[midway, above] {$\det = 0$};

    \begin{scope}[shift={(3.5,0)}]
        \draw[blue!70, ultra thick] (-0.5,0.5) -- (2,0.5);
        \node at (0.75,-0.5) {یک خط};
    \end{scope}
\end{tikzpicture}
\end{center}

\begin{practical}
\textbf{تشخیص وابستگی خطی:}

می‌خواهید بدانید آیا سه بردار در فضا مستقل خطی هستند؟ آنها را ستون‌های یک ماتریس قرار دهید. اگر $\det \neq 0$، مستقل هستند.
\end{practical}

% ============================================
\section{مثال‌های کاربردی}
% ============================================

\begin{practical}
\textbf{محاسبه مساحت مثلث با رأس‌های مشخص}

اگر رأس‌های مثلث $(x_1, y_1)$، $(x_2, y_2)$، $(x_3, y_3)$ باشند:
\[
\text{مساحت} = \frac{1}{2} \left| \det\threemat{x_1}{y_1}{1}{x_2}{y_2}{1}{x_3}{y_3}{1} \right|
\]
\end{practical}

\begin{practical}
\textbf{فیزیک: گشتاور و نیرو}

دترمینان در محاسبه ضرب خارجی بردارها استفاده می‌شود که در محاسبه گشتاور، تکانه زاویه‌ای، و میدان‌های الکترومغناطیسی کاربرد دارد.
\end{practical}

% ============================================
\section{تمرین‌ها}
% ============================================

\begin{exercise}
دترمینان ماتریس‌های زیر را محاسبه کنید:
\begin{enumerate}[label=(\alph*)]
    \item $\twomat{3}{7}{1}{4}$
    \item $\twomat{2}{6}{1}{3}$
    \item $\twomat{-1}{2}{3}{-6}$
\end{enumerate}
\end{exercise}

\begin{exercise}
اگر $\det(\mA) = 5$ و $\det(\mB) = -2$، مقدار $\det(\mA\mB)$ چیست؟
\end{exercise}

\begin{exercise}
مساحت متوازی‌الاضلاع با رأس‌های $(0,0)$، $(3,1)$، $(1,4)$، $(4,5)$ را محاسبه کنید.
\end{exercise}

\begin{exercise}
دترمینان ماتریس چرخش $\theta$ را محاسبه کنید و نتیجه را تفسیر کنید.
\end{exercise}

\begin{exercise}[چالشی]
ثابت کنید که برای هر ماتریس $\mA$:
\[
\det(\mA\inv) = \frac{1}{\det(\mA)}
\]
(فرض کنید $\mA$ معکوس‌پذیر است)
\end{exercise}

\begin{problem}
حجم متوازی‌السطوح ساخته شده از بردارهای زیر را محاسبه کنید:
\[
\va = \threevec{1}{0}{2}, \quad \vb = \threevec{0}{3}{1}, \quad \vc = \threevec{2}{1}{0}
\]
\end{problem}

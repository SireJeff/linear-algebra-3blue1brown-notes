% lecture01.tex - مقدمه‌ای بر بردارها
% Chapter 1: Introduction to Vectors

\chapter{مقدمه‌ای بر بردارها}
\label{ch:vectors}

\begin{abstract}
بردار، سنگ بنای اصلی جبر خطی است. در این درس با سه دیدگاه متفاوت به بردار آشنا می‌شویم: دیدگاه فیزیک، دیدگاه علوم کامپیوتر، و دیدگاه ریاضیات. همچنین عملیات‌های پایه‌ای جمع برداری و ضرب اسکالری را یاد می‌گیریم.
\end{abstract}

% ============================================
\section{سه دیدگاه به بردار}
% ============================================

مفهوم بردار بسته به رشته تحصیلی، معانی متفاوتی دارد. درک این سه دیدگاه به ما کمک می‌کند تا تصویر کاملی از بردارها داشته باشیم.

% -------------------------------------------
\subsection{دیدگاه فیزیک}
% -------------------------------------------

\begin{definition}[بردار از دیدگاه فیزیک]
بردار یک \vocab{پیکان} در فضا است که با دو ویژگی تعریف می‌شود:
\begin{enumerate}
    \item \textbf{طول} (اندازه یا بزرگی)
    \item \textbf{جهت}
\end{enumerate}
تا زمانی که این دو ویژگی ثابت بماند، بردار را می‌توان در فضا جابجا کرد و همچنان همان بردار باقی می‌ماند.
\end{definition}

\begin{intuition}
یک پیکان روی کاغذ رسم کنید. حالا آن را به نقطه دیگری منتقل کنید بدون اینکه بچرخانید یا اندازه‌اش را عوض کنید. از نظر فیزیکدان‌ها، این همان بردار اولی است!

به عنوان مثال، نیروی گرانش روی یک سیب، همیشه رو به پایین است با اندازه‌ای ثابت - مهم نیست سیب کجای اتاق باشد.
\end{intuition}

\begin{center}
\begin{tikzpicture}[scale=1.2]
    % Draw multiple copies of same vector
    \draw[vector, blue!70, very thick] (0,0) -- (2,1) node[midway, above] {$\vv$};
    \draw[vector, blue!70, very thick] (3,1) -- (5,2) node[midway, above] {$\vv$};
    \draw[vector, blue!70, very thick] (-1,2) -- (1,3) node[midway, above] {$\vv$};

    % Note
    \node at (2,-0.5) {\small همه اینها یک بردار هستند};
\end{tikzpicture}
\end{center}

% -------------------------------------------
\subsection{دیدگاه علوم کامپیوتر}
% -------------------------------------------

\begin{definition}[بردار از دیدگاه علوم کامپیوتر]
بردار یک \vocab{لیست مرتب از اعداد} است. برای مثال:
\[
\vv = \twovec{3}{-2}
\]
در این دیدگاه، «بردار» تقریباً مترادف با «لیست» است.
\end{definition}

\begin{practical}
\textbf{مثال: مدل‌سازی قیمت مسکن}

فرض کنید می‌خواهید خانه‌ها را بر اساس دو ویژگی مدل کنید:
\begin{itemize}
    \item متراژ (بر حسب متر مربع)
    \item قیمت (بر حسب میلیون تومان)
\end{itemize}

هر خانه یک بردار دوبعدی است:
\[
\text{خانه}_1 = \twovec{85}{2500}, \quad
\text{خانه}_2 = \twovec{120}{4200}, \quad
\text{خانه}_3 = \twovec{65}{1800}
\]

\textbf{نکته مهم:} ترتیب اعداد اهمیت دارد! $\twovec{85}{2500}$ با $\twovec{2500}{85}$ فرق دارد.
\end{practical}

% -------------------------------------------
\subsection{دیدگاه ریاضیات}
% -------------------------------------------

\begin{definition}[بردار از دیدگاه ریاضیات]
ریاضیدان بردار را به صورت انتزاعی تعریف می‌کند: بردار هر چیزی است که بتوان دو عملیات زیر را روی آن انجام داد:
\begin{enumerate}
    \item \textbf{جمع دو بردار}
    \item \textbf{ضرب بردار در یک عدد (اسکالر)}
\end{enumerate}
جزئیات این تعریف را در درس آخر (فضاهای برداری انتزاعی) بررسی خواهیم کرد.
\end{definition}

\begin{important}
در سراسر این دوره، ما بردار را به صورت یک \textbf{پیکان در دستگاه مختصات} تصور می‌کنیم که \textbf{دُم آن روی مبدأ} قرار دارد. این دیدگاه هندسی به ما کمک می‌کند مفاهیم را بهتر درک کنیم.
\end{important}

% ============================================
\section{دستگاه مختصات و نمایش بردار}
% ============================================

\begin{definition}[دستگاه مختصات دکارتی دوبعدی]
دستگاه مختصات شامل:
\begin{itemize}
    \item محور افقی: \vocab{محور $x$}
    \item محور عمودی: \vocab{محور $y$}
    \item نقطه تقاطع: \vocab{مبدأ} \lr{(Origin)}
\end{itemize}
\end{definition}

\begin{center}
\begin{tikzpicture}[scale=1.5]
    % Grid
    \draw[grid] (-0.5,-0.5) grid (4.5,3.5);

    % Axes
    \draw[axis, thick] (-0.5,0) -- (4.5,0) node[right] {$x$};
    \draw[axis, thick] (0,-0.5) -- (0,3.5) node[above] {$y$};

    % Origin label
    \node[below left] at (0,0) {O};

    % Tick marks
    \foreach \x in {1,2,3,4} {
        \draw (\x,0.1) -- (\x,-0.1) node[below] {\small $\x$};
    }
    \foreach \y in {1,2,3} {
        \draw (0.1,\y) -- (-0.1,\y) node[left] {\small $\y$};
    }

    % Vector
    \draw[vector, blue!70, ultra thick] (0,0) -- (3,2);
    \node[above right, blue!70!black] at (3,2) {$\vv = \twovec{3}{2}$};

    % Dashed lines to show coordinates
    \draw[dashed, red!70] (3,0) -- (3,2);
    \draw[dashed, red!70] (0,2) -- (3,2);

    % Coordinate labels
    \node[below, red!70!black] at (3,0) {$3$};
    \node[left, red!70!black] at (0,2) {$2$};
\end{tikzpicture}
\end{center}

\begin{definition}[مختصات بردار]
\vocab{مختصات} یک بردار، زوجی از اعداد است که نحوه رسیدن از مبدأ به نوک بردار را توصیف می‌کند:
\begin{itemize}
    \item عدد اول: چقدر در راستای محور $x$ حرکت کنیم (راست مثبت، چپ منفی)
    \item عدد دوم: چقدر در راستای محور $y$ حرکت کنیم (بالا مثبت، پایین منفی)
\end{itemize}
\end{definition}

\begin{intuition}
مختصات $\twovec{3}{2}$ یعنی:
\begin{enumerate}
    \item از مبدأ، ۳ واحد به راست برو
    \item سپس ۲ واحد به بالا برو
    \item نوک بردار همین‌جاست!
\end{enumerate}
\end{intuition}

% ============================================
\section{فضای سه‌بعدی}
% ============================================

\begin{definition}[بردار سه‌بعدی]
در فضای سه‌بعدی، یک محور سوم به نام \vocab{محور $z$} اضافه می‌شود که بر هر دو محور $x$ و $y$ عمود است. هر بردار با سه عدد مشخص می‌شود:
\[
\vv = \threevec{a}{b}{c}
\]
\end{definition}

\begin{center}
\begin{tikzpicture}[scale=1.2]
    % 3D axes
    \draw[axis, thick] (0,0) -- (3,0) node[right] {$x$};
    \draw[axis, thick] (0,0) -- (0,3) node[above] {$z$};
    \draw[axis, thick] (0,0) -- (-1.5,-1) node[below left] {$y$};

    % Vector
    \draw[vector, blue!70, ultra thick] (0,0) -- (2,1.5);
    \draw[vector, green!60!black, ultra thick] (0,0) -- (-0.8,-0.5);

    % Labels
    \node[above right, blue!70!black] at (2,1.5) {$\threevec{2}{0}{1.5}$};
\end{tikzpicture}
\end{center}

% ============================================
\section{جمع برداری}
% ============================================

\begin{definition}[جمع دو بردار - روش هندسی]
برای جمع دو بردار $\va$ و $\vb$:
\begin{enumerate}
    \item بردار $\va$ را رسم کنید
    \item بردار $\vb$ را طوری جابجا کنید که دُم آن روی نوک $\va$ قرار گیرد
    \item بردار حاصل‌جمع از مبدأ (دُم $\va$) تا نوک $\vb$ کشیده می‌شود
\end{enumerate}
\end{definition}

\begin{center}
\begin{tikzpicture}[scale=1.3]
    % Grid
    \draw[grid] (-0.5,-0.5) grid (5.5,3.5);

    % Axes
    \draw[axis] (-0.5,0) -- (5.5,0);
    \draw[axis] (0,-0.5) -- (0,3.5);

    % Vector a
    \draw[vector, red!70, very thick] (0,0) -- (2,1) node[midway, below] {$\va$};

    % Vector b (translated)
    \draw[vector, blue!70, very thick] (2,1) -- (4,3) node[midway, right] {$\vb$};

    % Sum vector
    \draw[vector, purple!70, ultra thick] (0,0) -- (4,3) node[midway, above left] {$\va + \vb$};

    % Original b (dashed)
    \draw[vector, blue!40, dashed] (0,0) -- (2,2) node[midway, left] {\small $\vb$};
\end{tikzpicture}
\end{center}

\begin{definition}[جمع دو بردار - روش جبری]
اگر $\va = \twovec{a_1}{a_2}$ و $\vb = \twovec{b_1}{b_2}$ باشند:
\[
\va + \vb = \twovec{a_1 + b_1}{a_2 + b_2}
\]
یعنی مؤلفه‌های متناظر را با هم جمع می‌کنیم.
\end{definition}

\begin{example}
\[
\twovec{1}{2} + \twovec{3}{-1} = \twovec{1+3}{2+(-1)} = \twovec{4}{1}
\]
\end{example}

\begin{intuition}
جمع برداری را می‌توان به صورت \textbf{پیمودن مسیر} تفسیر کرد:
\begin{itemize}
    \item اگر ابتدا بردار $\va$ را طی کنید
    \item سپس بردار $\vb$ را طی کنید
    \item اثر نهایی مثل این است که بردار $\va + \vb$ را مستقیم طی کرده باشید
\end{itemize}

مثل راه رفتن: اگر ۲ قدم به راست و سپس ۵ قدم به راست بروید، مثل این است که ۷ قدم به راست رفته باشید.
\end{intuition}

\begin{practical}
\textbf{مثال: سرعت هواپیما در باد}

یک هواپیما با سرعت $\twovec{500}{0}$ کیلومتر بر ساعت به سمت شرق پرواز می‌کند (جهت مثبت $x$). باد با سرعت $\twovec{0}{50}$ کیلومتر بر ساعت به سمت شمال می‌وزد (جهت مثبت $y$).

سرعت واقعی هواپیما نسبت به زمین:
\[
\vv_{\text{واقعی}} = \twovec{500}{0} + \twovec{0}{50} = \twovec{500}{50}
\]

هواپیما هم به شرق و هم کمی به شمال حرکت می‌کند.
\end{practical}

% ============================================
\section{ضرب اسکالری (ضرب عددی)}
% ============================================

\begin{definition}[ضرب اسکالر در بردار]
ضرب یک عدد (اسکالر) $c$ در بردار $\vv$، بردار را به نسبت $c$ \vocab{مقیاس} می‌کند:
\[
c \cdot \twovec{v_1}{v_2} = \twovec{c \cdot v_1}{c \cdot v_2}
\]
\end{definition}

\begin{center}
\begin{tikzpicture}[scale=1]
    % Original vector
    \draw[vector, blue!70, very thick] (0,0) -- (1.5,1) node[right] {$\vv$};

    % Scaled vectors
    \draw[vector, green!60!black, very thick] (3,0) -- (6,2) node[right] {$2\vv$};
    \draw[vector, orange!70, very thick] (8,0) -- (8.75,0.5) node[right] {$\frac{1}{2}\vv$};
    \draw[vector, red!70, very thick] (11,0) -- (8.3,-1.8) node[below] {$-1.8\vv$};

    % Labels below
    \node at (0.75,-0.5) {\small اصلی};
    \node at (4.5,-0.5) {\small کشیده};
    \node at (8.4,-0.5) {\small فشرده};
    \node at (9.7,-2.3) {\small معکوس و کشیده};
\end{tikzpicture}
\end{center}

\begin{important}
اثرات ضرب اسکالری:
\begin{itemize}
    \item $c > 1$: بردار \textbf{کشیده} می‌شود
    \item $0 < c < 1$: بردار \textbf{فشرده} می‌شود
    \item $c < 0$: بردار \textbf{معکوس} می‌شود و سپس مقیاس می‌شود
    \item $c = 0$: بردار صفر می‌شود
    \item $c = 1$: بردار بدون تغییر می‌ماند
\end{itemize}
\end{important}

\begin{warning}
اگر $c < 0$ باشد، بردار علاوه بر تغییر اندازه، \textbf{جهتش نیز معکوس} می‌شود!
\end{warning}

\begin{example}
اگر $\vv = \twovec{4}{-2}$ باشد:
\begin{align*}
2\vv &= \twovec{8}{-4} & &\text{(دو برابر شده)} \\
-\vv &= \twovec{-4}{2} & &\text{(معکوس شده)} \\
\frac{1}{2}\vv &= \twovec{2}{-1} & &\text{(نصف شده)}
\end{align*}
\end{example}

\begin{definition}[اسکالر]
در جبر خطی، به اعدادی که در بردارها ضرب می‌شوند \vocab{اسکالر} گفته می‌شود. این نام از فعل «مقیاس کردن» \lr{(to scale)} گرفته شده است. واژه «اسکالر» تقریباً مترادف «عدد» است.
\end{definition}

% ============================================
\section{بردارهای پایه}
% ============================================

\begin{definition}[بردارهای پایه استاندارد در $\Rtwo$]
دو بردار پایه استاندارد عبارتند از:
\[
\vi = \twovec{1}{0} \quad \text{و} \quad \vj = \twovec{0}{1}
\]
\end{definition}

\begin{center}
\begin{tikzpicture}[scale=2]
    % Grid
    \draw[grid] (-0.5,-0.5) grid (2.5,2.5);

    % Axes
    \draw[axis] (-0.5,0) -- (2.5,0) node[right] {$x$};
    \draw[axis] (0,-0.5) -- (0,2.5) node[above] {$y$};

    % Basis vectors
    \draw[basis, -{Stealth[length=4mm]}] (0,0) -- (1,0) node[below right] {$\vi$};
    \draw[basis, -{Stealth[length=4mm]}] (0,0) -- (0,1) node[above left] {$\vj$};

    % A general vector
    \draw[vector, blue!70, very thick] (0,0) -- (2,1.5);
    \node[above right, blue!70!black] at (2,1.5) {$\vv = 2\vi + 1.5\vj$};

    % Components
    \draw[dashed, orange] (0,0) -- (2,0) node[midway, below] {$2\vi$};
    \draw[dashed, orange] (2,0) -- (2,1.5) node[midway, right] {$1.5\vj$};
\end{tikzpicture}
\end{center}

\begin{theorem}
هر بردار در $\Rtwo$ را می‌توان به صورت ترکیب خطی از بردارهای پایه نوشت:
\[
\twovec{a}{b} = a\vi + b\vj = a\twovec{1}{0} + b\twovec{0}{1}
\]
\end{theorem}

\begin{intuition}
وقتی می‌نویسیم $\twovec{3}{2}$، در واقع داریم می‌گوییم:
\begin{center}
«۳ تا از $\vi$ بردار و ۲ تا از $\vj$ را با هم جمع کن»
\end{center}
یعنی:
\[
\twovec{3}{2} = 3\vi + 2\vj = 3\twovec{1}{0} + 2\twovec{0}{1}
\]
\end{intuition}

% ============================================
\section{ارتباط دیدگاه‌ها}
% ============================================

\begin{summary}
\textbf{قدرت جبر خطی} در توانایی ترجمه بین دیدگاه‌های مختلف است:
\begin{itemize}
    \item \textbf{تحلیل‌گر داده:} می‌تواند لیست‌های طولانی اعداد را به صورت بردار در فضا تجسم کند
    \item \textbf{فیزیکدان:} می‌تواند حرکت و نیروها را با اعداد توصیف کند
    \item \textbf{برنامه‌نویس گرافیک:} می‌تواند تبدیلات هندسی را با ماتریس‌ها پیاده‌سازی کند
\end{itemize}
\end{summary}

% ============================================
\section{تمرین‌ها}
% ============================================

\begin{exercise}
بردارهای زیر را در دستگاه مختصات رسم کنید:
\[
\va = \twovec{3}{1}, \quad \vb = \twovec{-2}{4}, \quad \vc = \twovec{0}{-3}
\]
\end{exercise}

\begin{exercise}
حاصل جمع و تفاضل بردارهای زیر را محاسبه کنید:
\[
\va = \twovec{2}{5}, \quad \vb = \twovec{-1}{3}
\]
\begin{enumerate}[label=(\alph*)}]
    \item $\va + \vb$
    \item $\va - \vb$
    \item $2\va + 3\vb$
\end{enumerate}
\end{exercise}

\begin{exercise}
اگر $\vv = \twovec{4}{-2}$ باشد، بردارهای زیر را محاسبه و رسم کنید:
\begin{enumerate}[label=(\alph*)]
    \item $2\vv$
    \item $-\vv$
    \item $\frac{1}{2}\vv$
    \item $-2.5\vv$
\end{enumerate}
\end{exercise}

\begin{exercise}[کاربردی]
یک کشتی با سرعت $30$ کیلومتر بر ساعت به سمت شمال حرکت می‌کند. جریان آب با سرعت $10$ کیلومتر بر ساعت به سمت شرق است. سرعت واقعی کشتی نسبت به ساحل چیست؟
\end{exercise}

\begin{exercise}
نشان دهید که برای هر بردار $\vv$:
\[
\vv + (-\vv) = \vzero
\]
که در آن $\vzero = \twovec{0}{0}$ بردار صفر است.
\end{exercise}

\begin{exercise}
ثابت کنید که جمع برداری خاصیت جابجایی دارد:
\[
\va + \vb = \vb + \va
\]
\end{exercise}

\begin{exercise}[چالشی]
سه نقطه $A(1,2)$، $B(4,6)$ و $C(7,2)$ داده شده‌اند. نشان دهید که این سه نقطه یک مثلث متساوی‌الساقین تشکیل می‌دهند.

\textit{راهنمایی: طول بردار $\twovec{a}{b}$ برابر است با $\sqrt{a^2 + b^2}$.}
\end{exercise}

\begin{problem}
در فضای سه‌بعدی، بردار $\vv = \threevec{2}{3}{-1}$ داده شده است. بردارهای زیر را محاسبه کنید:
\begin{enumerate}[label=(\alph*)]
    \item $3\vv$
    \item $\vv + \threevec{1}{-1}{2}$
    \item $-\frac{1}{2}\vv$
\end{enumerate}
\end{problem}

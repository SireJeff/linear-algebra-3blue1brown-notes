% lecture11.tex - محاسبه مقادیر ویژه و پایه ویژه
% Chapter 11: Computing Eigenvalues and Eigenbasis

\chapter{محاسبه مقادیر ویژه و پایه ویژه}
\label{ch:eigenbasis}

\begin{abstract}
در این درس روش‌های سریع‌تر برای محاسبه مقادیر ویژه ماتریس‌های $2 \times 2$ را یاد می‌گیریم و با مفهوم پایه ویژه و قطری‌سازی آشنا می‌شویم.
\end{abstract}

% ============================================
\section{ترفند سریع برای ماتریس $2 \times 2$}
% ============================================

\begin{theorem}[فرمول سریع]
برای ماتریس $\mA = \twomat{a}{b}{c}{d}$:

\textbf{حاصل‌جمع مقادیر ویژه:}
\[
\lambda_1 + \lambda_2 = a + d = \tr(\mA)
\]

\textbf{حاصل‌ضرب مقادیر ویژه:}
\[
\lambda_1 \cdot \lambda_2 = ad - bc = \det(\mA)
\]
\end{theorem}

\begin{intuition}
با دانستن $m = \lambda_1 + \lambda_2$ و $p = \lambda_1 \lambda_2$، می‌توانید $\lambda_{1,2}$ را پیدا کنید:
\[
\lambda = \frac{m}{2} \pm \sqrt{\left(\frac{m}{2}\right)^2 - p}
\]
یا: دو عدد پیدا کنید که مجموعشان $m$ و حاصل‌ضربشان $p$ باشد.
\end{intuition}

\begin{example}
$\mA = \twomat{3}{1}{4}{1}$

اثر: $m = 3 + 1 = 4$

دترمینان: $p = 3 - 4 = -1$

مقادیر ویژه: دو عددی که $x + y = 4$ و $xy = -1$:
\[
\lambda = 2 \pm \sqrt{4 - (-1)} = 2 \pm \sqrt{5}
\]
\end{example}

% ============================================
\section{پایه ویژه \lr{(Eigenbasis)}}
% ============================================

\begin{definition}[پایه ویژه]
اگر بردارهای ویژه یک ماتریس بتوانند یک \vocab{پایه} تشکیل دهند (یعنی $n$ بردار ویژه مستقل خطی داشته باشیم)، این پایه را \vocab{پایه ویژه} می‌نامیم.
\end{definition}

\begin{theorem}
در پایه ویژه، ماتریس تبدیل \vocab{قطری} می‌شود:
\[
[\mA]_{\text{پایه ویژه}} = \mD = \threemat{\lambda_1}{0}{0}{0}{\lambda_2}{0}{0}{0}{\ddots}
\]
\end{theorem}

\begin{intuition}
چرا قطری؟

در پایه ویژه، هر بردار پایه فقط مقیاس می‌شود:
\begin{align*}
\mA \vv_1 &= \lambda_1 \vv_1 \to \text{ستون اول: } \threevec{\lambda_1}{0}{0} \\
\mA \vv_2 &= \lambda_2 \vv_2 \to \text{ستون دوم: } \threevec{0}{\lambda_2}{0}
\end{align*}
\end{intuition}

% ============================================
\section{قطری‌سازی}
% ============================================

\begin{definition}[قطری‌سازی]
ماتریس $\mA$ \vocab{قطری‌پذیر} است اگر:
\[
\mA = \mP \mD \mP\inv
\]
که $\mD$ قطری و $\mP$ ماتریس بردارهای ویژه است.
\end{definition}

\begin{theorem}[شرط قطری‌پذیری]
ماتریس $n \times n$ قطری‌پذیر است اگر و تنها اگر $n$ بردار ویژه مستقل خطی داشته باشد.
\end{theorem}

\begin{example}
$\mA = \twomat{3}{1}{0}{2}$ با مقادیر ویژه $\lambda_1 = 3$, $\lambda_2 = 2$

بردارهای ویژه: $\vv_1 = \twovec{1}{0}$, $\vv_2 = \twovec{-1}{1}$

\[
\mP = \twomat{1}{-1}{0}{1}, \quad \mD = \twomat{3}{0}{0}{2}
\]

بررسی: $\mP\mD\mP\inv = \mA$ ✓
\end{example}

% ============================================
\section{توان ماتریس با قطری‌سازی}
% ============================================

\begin{theorem}
اگر $\mA = \mP\mD\mP\inv$:
\[
\mA^n = \mP \mD^n \mP\inv
\]
و $\mD^n$ بسیار ساده است:
\[
\mD^n = \threemat{\lambda_1^n}{0}{0}{0}{\lambda_2^n}{0}{0}{0}{\ddots}
\]
\end{theorem}

\begin{example}
محاسبه $\mA^{100}$ برای مثال قبل:
\[
\mA^{100} = \mP \twomat{3^{100}}{0}{0}{2^{100}} \mP\inv
\]
بدون قطری‌سازی، باید ۱۰۰ ضرب ماتریسی انجام می‌دادیم!
\end{example}

\begin{practical}
\textbf{کاربرد: اعداد فیبوناچی}

دنباله فیبوناچی: $F_{n+1} = F_n + F_{n-1}$

ماتریس:
\[
\twomat{1}{1}{1}{0}^n = \twomat{F_{n+1}}{F_n}{F_n}{F_{n-1}}
\]

با قطری‌سازی، فرمول بسته پیدا می‌کنیم:
\[
F_n = \frac{1}{\sqrt{5}}\left[\left(\frac{1+\sqrt{5}}{2}\right)^n - \left(\frac{1-\sqrt{5}}{2}\right)^n\right]
\]
\end{practical}

% ============================================
\section{ماتریس‌های غیرقطری‌پذیر}
% ============================================

\begin{example}
ماتریس برش $\mA = \twomat{1}{1}{0}{1}$:

چندجمله‌ای مشخصه: $(1-\lambda)^2 = 0$

مقدار ویژه: $\lambda = 1$ (مضاعف)

بردار ویژه: فقط $\twovec{1}{0}$ (یک بعدی)

این ماتریس قطری‌پذیر نیست!
\end{example}

\begin{warning}
مقدار ویژه تکراری لزوماً مشکل‌ساز نیست. مشکل زمانی است که تعداد بردارهای ویژه مستقل کمتر از تعداد تکرار مقدار ویژه باشد.
\end{warning}

% ============================================
\section{ماتریس‌های متقارن}
% ============================================

\begin{theorem}[قضیه طیفی]
ماتریس متقارن ($\mA = \mA\trans$):
\begin{enumerate}
    \item مقادیر ویژه حقیقی دارد
    \item بردارهای ویژه متناظر با مقادیر ویژه متمایز، عمود هستند
    \item همیشه قطری‌پذیر است (با پایه متعامد)
\end{enumerate}
\end{theorem}

\begin{intuition}
ماتریس‌های متقارن «خوش‌رفتار» هستند. آنها همیشه قطری می‌شوند و بردارهای ویژه‌شان عمود هستند - مثل محورهای اصلی یک بیضی.
\end{intuition}

% ============================================
\section{تمرین‌ها}
% ============================================

\begin{exercise}
با ترفند سریع، مقادیر ویژه ماتریس زیر را پیدا کنید:
\[
\mA = \twomat{5}{2}{2}{2}
\]
\end{exercise}

\begin{exercise}
ماتریس $\mA = \twomat{2}{1}{1}{2}$ را قطری کنید.
\end{exercise}

\begin{exercise}
با استفاده از قطری‌سازی، $\mA^{10}$ را محاسبه کنید:
\[
\mA = \twomat{1}{1}{0}{2}
\]
\end{exercise}

\begin{exercise}
آیا ماتریس زیر قطری‌پذیر است؟
\[
\mA = \twomat{2}{1}{0}{2}
\]
\end{exercise}

\begin{exercise}[چالشی]
نشان دهید که $\mA$ و $\mA\trans$ مقادیر ویژه یکسانی دارند.
\end{exercise}

\begin{problem}
جمعیت خرگوش‌ها و روباه‌ها با مدل زیر توصیف می‌شود:
\[
\twovec{R_{n+1}}{F_{n+1}} = \twomat{1.1}{-0.4}{0.2}{0.8}\twovec{R_n}{F_n}
\]
رفتار بلندمدت جمعیت را تحلیل کنید.
\end{problem}

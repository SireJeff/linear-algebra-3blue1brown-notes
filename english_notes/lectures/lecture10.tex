% lecture10.tex - Eigenvectors and Eigenvalues
% Chapter 10: Eigenvectors and Eigenvalues

\chapter{Eigenvectors and Eigenvalues}
\label{ch:eigen}

\begin{abstract}
Eigenvectors are special directions that a linear transformation only scales without changing their direction. This concept is one of the most important ideas in linear algebra with broad applications in physics, engineering, and data science.
\end{abstract}

% ============================================
\section{Motivation: Special Vectors}
% ============================================

\begin{intuition}
When you apply a linear transformation, most vectors get knocked off their original direction. But some special vectors only get stretched or compressed and \textbf{stay on the same line}.

These special vectors are called \textbf{eigenvectors}.
\end{intuition}

\begin{center}
\begin{tikzpicture}[scale=1.2]
    % Before transformation
    \begin{scope}[shift={(0,0)}]
        \draw[grid] (-2,-2) grid (2,2);
        \draw[axis] (-2,0) -- (2,0);
        \draw[axis] (0,-2) -- (0,2);
        \draw[vector, blue!70, very thick] (0,0) -- (1,0.5) node[right] {$\vv_1$};
        \draw[vector, red!70, very thick] (0,0) -- (0.5,1) node[above] {$\vv_2$};
        \draw[vector, green!60!black, very thick] (0,0) -- (-1,1) node[above left] {eigen!};
        \node at (0,-2.5) {Before transformation};
    \end{scope}

    \draw[-{Stealth}, very thick] (2.5,0) -- (4,0) node[midway, above] {$T$};

    % After transformation
    \begin{scope}[shift={(6.5,0)}]
        \draw[blue!20] (-2,-1) -- (0,2) -- (4,2) -- (2,-1) -- cycle;
        \draw[axis] (-2,0) -- (4,0);
        \draw[axis] (0,-2) -- (0,2);
        \draw[vector, blue!70, very thick] (0,0) -- (1.5,1) node[right] {$T(\vv_1)$};
        \draw[vector, red!70, very thick] (0,0) -- (1.5,1.5) node[above] {$T(\vv_2)$};
        \draw[vector, green!60!black, very thick] (0,0) -- (-2,2) node[above left] {$2 \times$ eigen!};
        \node at (1,-2.5) {After transformation};
    \end{scope}
\end{tikzpicture}
\end{center}

% ============================================
\section{Formal Definition}
% ============================================

\begin{definition}[Eigenvector and Eigenvalue]
A nonzero vector $\vv$ is an \vocab{eigenvector} of matrix $\mA$ if:
\[
\mA\vv = \lambda\vv
\]
for some scalar $\lambda$. This scalar is called the corresponding \vocab{eigenvalue}.
\end{definition}

\begin{intuition}
$\mA\vv = \lambda\vv$ means:
\begin{center}
``Transformation $\mA$ acts on vector $\vv$ as just a scalar multiplication''
\end{center}
Vector $\vv$ stays on the same line, just scaled by $\lambda$.
\end{intuition}

\begin{example}
For matrix $\mA = \twomat{3}{1}{0}{2}$:

Vector $\vv = \twovec{1}{0}$ is an eigenvector:
\[
\mA\vv = \twomat{3}{1}{0}{2}\twovec{1}{0} = \twovec{3}{0} = 3\twovec{1}{0} = 3\vv
\]
Corresponding eigenvalue: $\lambda = 3$
\end{example}

% ============================================
\section{Finding Eigenvalues}
% ============================================

\begin{theorem}[Characteristic Equation]
$\lambda$ is an eigenvalue of $\mA$ if and only if:
\[
\det(\mA - \lambda\mI) = 0
\]
\end{theorem}

\begin{proof}
$\mA\vv = \lambda\vv$ can be written as:
\[
\mA\vv - \lambda\vv = \vzero \implies (\mA - \lambda\mI)\vv = \vzero
\]
This equation has a nonzero solution if and only if $\mA - \lambda\mI$ is not invertible, i.e., $\det(\mA - \lambda\mI) = 0$.
\end{proof}

\begin{definition}[Characteristic Polynomial]
$\det(\mA - \lambda\mI)$ is a polynomial in $\lambda$ called the \vocab{characteristic polynomial}. Its roots are the eigenvalues.
\end{definition}

\begin{example}
For $\mA = \twomat{3}{1}{0}{2}$:
\[
\det(\mA - \lambda\mI) = \det\twomat{3-\lambda}{1}{0}{2-\lambda} = (3-\lambda)(2-\lambda) - 0 = 0
\]
Roots: $\lambda_1 = 3$, $\lambda_2 = 2$
\end{example}

% ============================================
\section{Finding Eigenvectors}
% ============================================

\begin{theorem}
For each eigenvalue $\lambda$, the corresponding eigenvectors are found by solving the homogeneous system:
\[
(\mA - \lambda\mI)\vv = \vzero
\]
\end{theorem}

\begin{example}
Continuing the previous example with $\lambda = 2$:
\[
(\mA - 2\mI)\vv = \twomat{1}{1}{0}{0}\twovec{v_1}{v_2} = \vzero
\]
Equation: $v_1 + v_2 = 0$, so $v_1 = -v_2$

Eigenvector: $\vv = t\twovec{-1}{1}$ for any $t \neq 0$
\end{example}

% ============================================
\section{Eigenspace}
% ============================================

\begin{definition}[Eigenspace]
The \vocab{eigenspace} corresponding to eigenvalue $\lambda$:
\[
E_\lambda = \Null(\mA - \lambda\mI) = \{\vv \mid \mA\vv = \lambda\vv\}
\]
\end{definition}

\begin{intuition}
The eigenspace contains all vectors that transformation $\mA$ only scales by factor $\lambda$. This space is always a vector subspace.
\end{intuition}

% ============================================
\section{Geometric Interpretation}
% ============================================

\begin{practical}
\textbf{3D Rotation}

For a rotation in $\Rthree$, the eigenvector with $\lambda = 1$ is the \textbf{axis of rotation}! This vector stays fixed.

Describing rotation with axis and angle is much simpler than a $3 \times 3$ matrix.
\end{practical}

\begin{example}
Shear matrix: $\mA = \twomat{1}{1}{0}{1}$

Eigenvalue: $\lambda = 1$ (repeated)

Eigenvector: $\twovec{1}{0}$ (only one eigen-direction)

Interpretation: Shear keeps the $x$-axis fixed.
\end{example}

% ============================================
\section{Special Cases}
% ============================================

\begin{theorem}[2D Rotation]
The rotation matrix $\mR_\theta = \twomat{\cos\theta}{-\sin\theta}{\sin\theta}{\cos\theta}$ for $\theta \neq 0, \pi$:

Eigenvalues: $\lambda = \cos\theta \pm i\sin\theta$ (complex!)

No real vector stays in place - all vectors rotate.
\end{theorem}

\begin{warning}
Eigenvalues can be \textbf{complex} even for real matrices! This happens in rotations.
\end{warning}

% ============================================
\section{Applications}
% ============================================

\begin{practical}
\textbf{Google PageRank}

Consider web pages as vectors and links as a matrix. The dominant eigenvector (with the largest eigenvalue) shows the relative importance of pages.
\end{practical}

\begin{practical}
\textbf{Principal Component Analysis (PCA)}

In machine learning, eigenvectors of the covariance matrix show the principal directions of variation in the data.
\end{practical}

\begin{practical}
\textbf{Quantum Mechanics}

Eigenvalues of operators = possible measurement outcomes

Eigenvectors = stable states of the system
\end{practical}

% ============================================
\section{Exercises}
% ============================================

\begin{exercise}
Find the eigenvalues and eigenvectors of the following matrix:
\[
\mA = \twomat{4}{1}{2}{3}
\]
\end{exercise}

\begin{exercise}
Show that the eigenvalues of a diagonal matrix are its diagonal entries.
\end{exercise}

\begin{exercise}
Find the eigenvalues of the $90°$ rotation matrix.
\end{exercise}

\begin{exercise}
If $\lambda$ is an eigenvalue of $\mA$, show that $\lambda^2$ is an eigenvalue of $\mA^2$.
\end{exercise}

\begin{exercise}[Challenge]
Prove that the trace of a matrix (sum of diagonal entries) equals the sum of eigenvalues.
\end{exercise}

\begin{problem}
Population matrix:
\[
\mL = \twomat{0}{4}{0.5}{0}
\]
Find the dominant eigenvalue and interpret it.
\end{problem}


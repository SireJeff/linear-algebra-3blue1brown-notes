% lecture09.tex - Change of Basis
% Chapter 9: Change of Basis

\chapter{Change of Basis}
\label{ch:changebasis}

\begin{abstract}
The same vector has different coordinates in different bases. In this chapter, we learn how to translate between different coordinate systems and how this concept relates to matrices.
\end{abstract}

% ============================================
\section{Coordinates Relative to Different Bases}
% ============================================

\begin{intuition}
The coordinates of a vector depend on the chosen basis. If you speak a different language, you express the same concept differently.

Example: A vector $\vv$ that is $\twovec{3}{2}$ in the standard basis might be $\twovec{1}{1}$ in a different basis!
\end{intuition}

\begin{definition}[Coordinates in a Basis]
If $\mathcal{B} = \{\vb_1, \vb_2\}$ is a basis, the coordinates of vector $\vv$ in this basis are the coefficients $c_1, c_2$ such that:
\[
\vv = c_1 \vb_1 + c_2 \vb_2
\]
Notation: $[\vv]_{\mathcal{B}} = \twovec{c_1}{c_2}$
\end{definition}

\begin{example}
Basis $\mathcal{B} = \left\{ \vb_1 = \twovec{2}{1}, \vb_2 = \twovec{-1}{1} \right\}$

Express vector $\vv = \twovec{3}{2}$ in this basis.

We need to find $c_1, c_2$ such that:
\[
c_1 \twovec{2}{1} + c_2 \twovec{-1}{1} = \twovec{3}{2}
\]
\[
\system{
2c_1 - c_2 &= 3 \\
c_1 + c_2 &= 2
}
\]
Solution: $c_1 = \frac{5}{3}$, $c_2 = \frac{1}{3}$

So $[\vv]_{\mathcal{B}} = \twovec{5/3}{1/3}$
\end{example}

% ============================================
\section{Change of Basis Matrix}
% ============================================

\begin{definition}[Change of Basis Matrix]
The \vocab{change of basis matrix} from basis $\mathcal{B}$ to the standard basis is a matrix whose columns are the basis vectors of $\mathcal{B}$:
\[
\mP = [\vb_1 \,|\, \vb_2 \,|\, \cdots \,|\, \vb_n]
\]
\end{definition}

\begin{theorem}
\[
\vv = \mP \, [\vv]_{\mathcal{B}}
\]
That is: coordinates in basis $\mathcal{B}$ $\times$ change of basis matrix = vector in standard basis
\end{theorem}

\begin{theorem}
\[
[\vv]_{\mathcal{B}} = \mP\inv \vv
\]
That is: to convert from standard basis to basis $\mathcal{B}$, we use the inverse.
\end{theorem}

% ============================================
\section{Change of Basis Transformation}
% ============================================

\begin{intuition}
Matrix $\mP$ is an identity transformation that only changes the language:
\begin{itemize}
    \item $\mP$: translates from language $\mathcal{B}$ to standard language
    \item $\mP\inv$: translates from standard language to language $\mathcal{B}$
\end{itemize}
\end{intuition}

\begin{center}
\begin{tikzpicture}[scale=1.2]
    % Standard basis
    \begin{scope}[shift={(0,0)}]
        \draw[grid] (-0.5,-0.5) grid (3.5,2.5);
        \draw[axis] (-0.5,0) -- (3.5,0);
        \draw[axis] (0,-0.5) -- (0,2.5);
        \draw[basis] (0,0) -- (1,0) node[below] {$\vi$};
        \draw[basis] (0,0) -- (0,1) node[left] {$\vj$};
        \draw[vector, blue!70, very thick] (0,0) -- (3,2) node[above] {$\vv$};
        \node at (1.5,-1) {Standard basis};
        \node at (1.5,-1.5) {$\vv = \twovec{3}{2}$};
    \end{scope}

    % Alternative basis
    \begin{scope}[shift={(6,0)}]
        \draw[gray!30] (-0.5,-0.5) -- (1,0.5) -- (3,1.5) -- (1.5,1) -- cycle;
        \draw[axis] (-0.5,0) -- (3.5,0);
        \draw[axis] (0,-0.5) -- (0,2.5);
        \draw[red!70, very thick, -{Stealth}] (0,0) -- (2,1) node[below right] {$\vb_1$};
        \draw[red!70, very thick, -{Stealth}] (0,0) -- (-1,1) node[above left] {$\vb_2$};
        \draw[vector, blue!70, very thick] (0,0) -- (3,2) node[above] {$\vv$};
        \node at (1.5,-1) {Basis $\mathcal{B}$};
        \node at (1.5,-1.5) {$[\vv]_{\mathcal{B}} = \twovec{?}{?}$};
    \end{scope}
\end{tikzpicture}
\end{center}

% ============================================
\section{Representing Transformations in Different Bases}
% ============================================

\begin{theorem}[Transformation in New Basis]
If $\mA$ is the matrix of transformation $T$ in the standard basis, the matrix of the same transformation in basis $\mathcal{B}$ is:
\[
[\mA]_{\mathcal{B}} = \mP\inv \mA \mP
\]
\end{theorem}

\begin{intuition}
This formula has three steps:
\begin{enumerate}
    \item $\mP$: translate from basis $\mathcal{B}$ to standard basis
    \item $\mA$: apply the transformation in standard basis
    \item $\mP\inv$: translate the result back to basis $\mathcal{B}$
\end{enumerate}
\end{intuition}

\begin{center}
\begin{tikzpicture}[node distance=3cm]
    \node (B1) {$[\vv]_{\mathcal{B}}$};
    \node (S1) [right of=B1] {$\vv$};
    \node (S2) [right of=S1] {$T(\vv)$};
    \node (B2) [right of=S2] {$[T(\vv)]_{\mathcal{B}}$};

    \draw[-{Stealth}, thick] (B1) -- (S1) node[midway, above] {$\mP$};
    \draw[-{Stealth}, thick] (S1) -- (S2) node[midway, above] {$\mA$};
    \draw[-{Stealth}, thick] (S2) -- (B2) node[midway, above] {$\mP\inv$};
    \draw[-{Stealth}, thick, blue] (B1) to[bend right=30] node[midway, below] {$\mP\inv\mA\mP$} (B2);
\end{tikzpicture}
\end{center}

% ============================================
\section{Importance of Change of Basis}
% ============================================

\begin{intuition}
Why is change of basis important?

Some transformations look \textbf{simpler} in certain bases. For example:
\begin{itemize}
    \item Rotation in the standard basis is complicated
    \item But in a basis where one axis lies on the axis of rotation, it becomes simpler
\end{itemize}

The best basis for a transformation? The \textbf{eigenbasis} - next chapter!
\end{intuition}

\begin{practical}
\textbf{Application: Simplifying Computations}

Suppose you want to compute $\mA^{100}$. If in a suitable basis, $\mA$ becomes diagonal:
\[
\mP\inv \mA \mP = \mD \quad \text{(diagonal)}
\]
Then:
\[
\mA^{100} = \mP \mD^{100} \mP\inv
\]
And $\mD^{100}$ is very easy to compute!
\end{practical}

% ============================================
\section{Exercises}
% ============================================

\begin{exercise}
Given basis $\mathcal{B} = \left\{ \twovec{1}{1}, \twovec{1}{-1} \right\}$, find the coordinates of vector $\vv = \twovec{3}{1}$ in this basis.
\end{exercise}

\begin{exercise}
Write the change of basis matrix from $\mathcal{B} = \left\{ \twovec{2}{1}, \twovec{1}{1} \right\}$ to the standard basis.
\end{exercise}

\begin{exercise}
If $\mA = \twomat{2}{1}{0}{3}$ in the standard basis, find the matrix of this transformation in basis $\mathcal{B} = \left\{ \twovec{1}{0}, \twovec{1}{1} \right\}$.
\end{exercise}

\begin{exercise}[Challenge]
Show that $\det(\mP\inv\mA\mP) = \det(\mA)$.
\end{exercise}

\begin{problem}
Given two bases $\mathcal{B}_1$ and $\mathcal{B}_2$. How do you compute the change of basis matrix directly from $\mathcal{B}_1$ to $\mathcal{B}_2$ (without going through the standard basis)?
\end{problem}


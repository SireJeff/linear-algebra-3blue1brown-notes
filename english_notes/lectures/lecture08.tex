% lecture08.tex - Cramer's Rule
% Chapter 8: Cramer's Rule

\chapter{Cramer's Rule}
\label{ch:cramer}

\begin{abstract}
Cramer's rule is an elegant method for solving systems of linear equations using determinants. In this chapter, we learn the geometric interpretation of this rule and its conditions for use.
\end{abstract}

% ============================================
\section{Statement of Cramer's Rule}
% ============================================

\begin{theorem}[Cramer's Rule]
For the system $\mA\vx = \vb$ with an $n \times n$ matrix and $\det(\mA) \neq 0$:
\[
x_i = \frac{\det(\mA_i)}{\det(\mA)}
\]
where $\mA_i$ is the matrix with column $i$ of $\mA$ replaced by vector $\vb$.
\end{theorem}

\begin{example}
For a $2 \times 2$ system:
\[
\system{
ax + by &= e \\
cx + dy &= f
}
\]
Solution:
\[
x = \frac{\det\twomat{e}{b}{f}{d}}{\det\twomat{a}{b}{c}{d}} = \frac{ed - bf}{ad - bc}
\]
\[
y = \frac{\det\twomat{a}{e}{c}{f}}{\det\twomat{a}{b}{c}{d}} = \frac{af - ec}{ad - bc}
\]
\end{example}

% ============================================
\section{Geometric Interpretation}
% ============================================

\begin{intuition}
In two dimensions, the equation $\mA\vx = \vb$ asks:

``What linear combination of the columns of $\mA$ equals $\vb$?''

If $\mA = [\va_1 \,|\, \va_2]$ and $\vx = \twovec{x}{y}$:
\[
x\va_1 + y\va_2 = \vb
\]
\end{intuition}

\begin{center}
\begin{tikzpicture}[scale=1.3]
    % Parallelogram
    \draw[vector, red!70, very thick] (0,0) -- (2,0.5) node[right] {$\va_1$};
    \draw[vector, blue!70, very thick] (0,0) -- (0.5,2) node[above] {$\va_2$};
    \draw[vector, green!60!black, ultra thick] (0,0) -- (1.5,1.5) node[above right] {$\vb$};

    % Parallelogram formed by b and a2
    \fill[orange!20] (0,0) -- (1.5,1.5) -- (2,3.5) -- (0.5,2) -- cycle;

    % Parallelogram formed by a1 and a2
    \fill[gray!20] (0,0) -- (2,0.5) -- (2.5,2.5) -- (0.5,2) -- cycle;

    \node at (1.2,2.5) {\small orange area};
    \node at (1.8,1) {\small gray area};
\end{tikzpicture}
\end{center}

\begin{theorem}[Area Interpretation]
\[
x = \frac{\text{Area of parallelogram}(\vb, \va_2)}{\text{Area of parallelogram}(\va_1, \va_2)}
\]
\[
y = \frac{\text{Area of parallelogram}(\va_1, \vb)}{\text{Area of parallelogram}(\va_1, \va_2)}
\]
\end{theorem}

\begin{intuition}
Why does this work?

Consider the parallelogram $(\vb, \va_2)$. Since $\vb = x\va_1 + y\va_2$:
\[
\text{Area}(\vb, \va_2) = \text{Area}(x\va_1 + y\va_2, \va_2)
\]
Since $\va_2 \times \va_2 = 0$:
\[
= \text{Area}(x\va_1, \va_2) = x \cdot \text{Area}(\va_1, \va_2)
\]
Therefore:
\[
x = \frac{\text{Area}(\vb, \va_2)}{\text{Area}(\va_1, \va_2)}
\]
\end{intuition}

% ============================================
\section{Computational Example}
% ============================================

\begin{example}
Solve the system:
\[
\system{
3x + 2y &= 7 \\
x + 4y &= 9
}
\]

\textbf{Step 1:} Compute $\det(\mA)$:
\[
\det\twomat{3}{2}{1}{4} = 12 - 2 = 10
\]

\textbf{Step 2:} Compute $x$:
\[
x = \frac{\det\twomat{7}{2}{9}{4}}{\det(\mA)} = \frac{28 - 18}{10} = \frac{10}{10} = 1
\]

\textbf{Step 3:} Compute $y$:
\[
y = \frac{\det\twomat{3}{7}{1}{9}}{\det(\mA)} = \frac{27 - 7}{10} = \frac{20}{10} = 2
\]

\textbf{Check:} $3(1) + 2(2) = 7$ \checkmark\ and $1(1) + 4(2) = 9$ \checkmark
\end{example}

% ============================================
\section{Three Dimensions and Higher}
% ============================================

\begin{example}
A $3 \times 3$ system:
\[
\system{
2x + y - z &= 3 \\
x - y + 2z &= 1 \\
3x + 2y + z &= 4
}
\]

Coefficient matrix:
\[
\mA = \threemat{2}{1}{-1}{1}{-1}{2}{3}{2}{1}
\]

\[
x = \frac{\det\threemat{3}{1}{-1}{1}{-1}{2}{4}{2}{1}}{\det(\mA)}, \quad
y = \frac{\det\threemat{2}{3}{-1}{1}{1}{2}{3}{4}{1}}{\det(\mA)}, \quad
z = \frac{\det\threemat{2}{1}{3}{1}{-1}{1}{3}{2}{4}}{\det(\mA)}
\]
\end{example}

\begin{intuition}
In three dimensions, instead of ratios of areas, we have ratios of \textbf{volumes}.

$x$ = ratio of volume of parallelepiped $(\vb, \va_2, \va_3)$ to volume of $(\va_1, \va_2, \va_3)$
\end{intuition}

% ============================================
\section{Limitations and Applications}
% ============================================

\begin{warning}
Cramer's rule:
\begin{itemize}
    \item Only works for square systems ($n$ equations, $n$ unknowns)
    \item Only when $\det(\mA) \neq 0$ (unique solution)
    \item For large $n$, computationally \textbf{very expensive}
\end{itemize}
\end{warning}

\begin{remark}
In practice, for large systems, methods like Gaussian elimination are used which are more efficient. However, Cramer's rule:
\begin{itemize}
    \item Provides deeper theoretical understanding
    \item Is useful for analytical formulas
    \item Is used in theorem proofs
\end{itemize}
\end{remark}

\begin{practical}
\textbf{Application: Finding Line Intersections}

Two lines $a_1x + b_1y = c_1$ and $a_2x + b_2y = c_2$ intersect at:
\[
x = \frac{c_1 b_2 - c_2 b_1}{a_1 b_2 - a_2 b_1}, \quad y = \frac{a_1 c_2 - a_2 c_1}{a_1 b_2 - a_2 b_1}
\]
This formula comes directly from Cramer's rule.
\end{practical}

% ============================================
\section{Exercises}
% ============================================

\begin{exercise}
Solve using Cramer's rule:
\[
\system{
2x + 3y &= 8 \\
4x - y &= 2
}
\]
\end{exercise}

\begin{exercise}
Solve using Cramer's rule:
\[
\system{
x + y + z &= 6 \\
2x - y + z &= 3 \\
x + 2y - z &= 2
}
\]
\end{exercise}

\begin{exercise}
Find the intersection point of lines $2x + 3y = 7$ and $x - y = 1$ using Cramer's rule.
\end{exercise}

\begin{exercise}
Explain the geometric interpretation of Cramer's rule when $\det(\mA) = 0$.
\end{exercise}

\begin{exercise}[Challenge]
Show that the Cramer formula is consistent with computing $\mA\inv \vb$.
\end{exercise}

\begin{problem}
A triangle with vertices $A(1,1)$, $B(4,2)$, $C(2,5)$ is given. Using Cramer's rule, find the coordinates of the centroid.
\end{problem}

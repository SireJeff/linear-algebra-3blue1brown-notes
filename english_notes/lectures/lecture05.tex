% lecture05.tex - Inverse Matrices, Column Space, and Null Space
% Chapter 5: Inverse Matrices, Column Space, and Null Space

\chapter{Inverse Matrices, Column Space, and Null Space}
\label{ch:inverse}

\begin{abstract}
In this chapter, we learn about inverse matrices, solving systems of linear equations, and important spaces associated with matrices (column space, null space, and rank).
\end{abstract}

% ============================================
\section{Systems of Linear Equations}
% ============================================

\begin{definition}[System of Equations in Matrix Form]
A system of linear equations:
\[
\system{
a_{11}x_1 + a_{12}x_2 + \cdots + a_{1n}x_n &= b_1 \\
a_{21}x_1 + a_{22}x_2 + \cdots + a_{2n}x_n &= b_2 \\
&\vdots \\
a_{m1}x_1 + a_{m2}x_2 + \cdots + a_{mn}x_n &= b_m
}
\]
can be written as $\mA\vx = \vb$.
\end{definition}

\begin{intuition}
Interpret the equation $\mA\vx = \vb$ as:
\begin{center}
``What vector $\vx$ lands on $\vb$ after applying transformation $\mA$?''
\end{center}
In other words: ``reverse'' the transformation $\mA$ to get from $\vb$ to $\vx$.
\end{intuition}

% ============================================
\section{Inverse Matrix}
% ============================================

\begin{definition}[Inverse Matrix]
The matrix $\mA\inv$ is the \vocab{inverse} of matrix $\mA$ if:
\[
\mA\inv \mA = \mA \mA\inv = \mI
\]
where $\mI$ is the identity matrix.
\end{definition}

\begin{intuition}
If $\mA$ is a transformation, $\mA\inv$ is the transformation that undoes the effect of $\mA$:
\begin{itemize}
    \item If $\mA$ is rotation by $90°$, then $\mA\inv$ is rotation by $-90°$
    \item If $\mA$ scales by 2, then $\mA\inv$ scales by $\frac{1}{2}$
\end{itemize}
\end{intuition}

\begin{theorem}[Solving Systems with the Inverse]
If $\mA$ is invertible, the unique solution to $\mA\vx = \vb$ is:
\[
\vx = \mA\inv \vb
\]
\end{theorem}

\subsection{Formula for $2 \times 2$ Inverse}

\begin{theorem}
If $\mA = \twomat{a}{b}{c}{d}$ and $\det(\mA) \neq 0$:
\[
\mA\inv = \frac{1}{ad-bc} \twomat{d}{-b}{-c}{a}
\]
\end{theorem}

\begin{example}
\[
\mA = \twomat{3}{1}{2}{4}, \quad \det(\mA) = 12 - 2 = 10
\]
\[
\mA\inv = \frac{1}{10} \twomat{4}{-1}{-2}{3} = \twomat{0.4}{-0.1}{-0.2}{0.3}
\]
\end{example}

% ============================================
\section{When Does the Inverse Exist?}
% ============================================

\begin{theorem}
Matrix $\mA$ is invertible if and only if $\det(\mA) \neq 0$.
\end{theorem}

\begin{intuition}
If $\det(\mA) = 0$, the transformation $\mA$ collapses space (e.g., plane to line). Such a transformation is not reversible---information is lost.

It's like adding several numbers together: you can't recover the original numbers from just the sum.
\end{intuition}

% ============================================
\section{Column Space}
% ============================================

\begin{definition}[Column Space]
The \vocab{column space} of matrix $\mA$, denoted $\Col(\mA)$, is the span of the columns of $\mA$:
\[
\Col(\mA) = \spn\{\text{columns of } \mA\}
\]
\end{definition}

\begin{intuition}
Column space = set of all possible outputs of transformation $\mA$

Question: ``Does $\mA\vx = \vb$ have a solution?'' is equivalent to ``Is $\vb$ in the column space of $\mA$?''
\end{intuition}

\begin{example}
\[
\mA = \twomat{1}{3}{2}{6}
\]
The second column = 3 times the first column, so:
\[
\Col(\mA) = \spn\left\{\twovec{1}{2}\right\} = \text{a line through the origin}
\]
\end{example}

% ============================================
\section{Null Space}
% ============================================

\begin{definition}[Null Space]
The \vocab{null space} (or kernel) of matrix $\mA$ is the set of all vectors that $\mA$ sends to zero:
\[
\Null(\mA) = \{\vx \mid \mA\vx = \vzero\}
\]
\end{definition}

\begin{intuition}
Null space = vectors that transformation $\mA$ ``squishes'' to zero

If $\det(\mA) \neq 0$: only the zero vector gets squished, so $\Null(\mA) = \{\vzero\}$

If $\det(\mA) = 0$: an entire line or plane gets compressed to the origin
\end{intuition}

\begin{example}
For matrix $\mA = \twomat{1}{2}{2}{4}$:

Solve $\mA\vx = \vzero$:
\[
\twomat{1}{2}{2}{4}\twovec{x}{y} = \twovec{0}{0}
\]
Equation: $x + 2y = 0$, so $x = -2y$

Null space: $\Null(\mA) = \left\{ t\twovec{-2}{1} \mid t \in \R \right\}$ (a line)
\end{example}

% ============================================
\section{Rank}
% ============================================

\begin{definition}[Rank]
The \vocab{rank} of matrix $\mA$ equals:
\begin{itemize}
    \item The dimension of the column space
    \item The number of linearly independent columns
    \item The number of linearly independent rows
\end{itemize}
\[
\rank(\mA) = \dim(\Col(\mA))
\]
\end{definition}

\begin{theorem}[Rank-Nullity Theorem]
For an $m \times n$ matrix:
\[
\rank(\mA) + \dim(\Null(\mA)) = n
\]
(the number of columns)
\end{theorem}

\begin{intuition}
Rank = dimensions that the transformation preserves

$\dim(\Null(\mA))$ = dimensions that are lost

Their sum = dimension of the input space
\end{intuition}

% ============================================
\section{Non-Square Matrices}
% ============================================

\begin{definition}[Transformations Between Dimensions]
An $m \times n$ matrix represents a transformation from $\R^n$ to $\R^m$:
\begin{itemize}
    \item $m > n$: transformation from lower to higher dimension
    \item $m < n$: transformation from higher to lower dimension
\end{itemize}
\end{definition}

\begin{example}
A $2 \times 3$ matrix:
\[
\mA = \twomat{1}{0}{2}{0}{1}{-1}
\]
is a transformation from $\Rthree$ to $\Rtwo$. It ``projects'' 3D space onto a plane.
\end{example}

% ============================================
\section{Different Cases for Systems of Equations}
% ============================================

\begin{theorem}[Analysis of $\mA\vx = \vb$]
\begin{enumerate}
    \item \textbf{Unique solution:} If $\det(\mA) \neq 0$
    \item \textbf{Infinitely many solutions:} If $\det(\mA) = 0$ and $\vb \in \Col(\mA)$
    \item \textbf{No solution:} If $\vb \notin \Col(\mA)$
\end{enumerate}
\end{theorem}

\begin{center}
\begin{tikzpicture}[scale=0.9]
    % Case 1: Unique solution
    \begin{scope}[shift={(0,0)}]
        \fill[blue!10] (-1.5,-1.5) rectangle (1.5,1.5);
        \draw[axis] (-1.5,0) -- (1.5,0);
        \draw[axis] (0,-1.5) -- (0,1.5);
        \fill[red] (0.8,0.6) circle (2pt) node[above right] {$\vx$};
        \node at (0,-2) {Unique solution};
        \node at (0,-2.5) {\small $\det \neq 0$};
    \end{scope}

    % Case 2: Infinite solutions
    \begin{scope}[shift={(5,0)}]
        \draw[blue!30, very thick] (-1.5,-0.75) -- (1.5,0.75);
        \draw[axis] (-1.5,0) -- (1.5,0);
        \draw[axis] (0,-1.5) -- (0,1.5);
        \node at (0,-2) {Infinite solutions};
        \node at (0,-2.5) {\small $\det = 0$, $\vb \in \Col$};
    \end{scope}

    % Case 3: No solution
    \begin{scope}[shift={(10,0)}]
        \draw[blue!30, very thick] (-1.5,-0.75) -- (1.5,0.75);
        \draw[axis] (-1.5,0) -- (1.5,0);
        \draw[axis] (0,-1.5) -- (0,1.5);
        \fill[red] (0.5,1.2) circle (2pt) node[right] {$\vb$};
        \node[red] at (0.5,0.8) {$\times$};
        \node at (0,-2) {No solution};
        \node at (0,-2.5) {\small $\vb \notin \Col$};
    \end{scope}
\end{tikzpicture}
\end{center}

% ============================================
\section{Exercises}
% ============================================

\begin{exercise}
Compute the inverse of:
\[
\mA = \twomat{2}{5}{1}{3}
\]
\end{exercise}

\begin{exercise}
Solve the following system using the inverse matrix:
\[
\system{
2x + 3y &= 7 \\
x + 2y &= 4
}
\]
\end{exercise}

\begin{exercise}
Find the null space of:
\[
\mA = \twomat{1}{-2}{-3}{6}
\]
\end{exercise}

\begin{exercise}
Determine the rank of:
\[
\mA = \threemat{1}{2}{3}{2}{4}{6}{1}{1}{1}
\]
\end{exercise}

\begin{exercise}[Challenge]
Show that $(\mA\mB)\inv = \mB\inv\mA\inv$ (if the inverses exist).
\end{exercise}

\begin{problem}
For what values of $k$ is the following matrix not invertible?
\[
\mA = \twomat{k}{2}{3}{k}
\]
\end{problem}

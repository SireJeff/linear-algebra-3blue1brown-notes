% lecture03.tex - Linear Transformations and Matrices
% Chapter 3: Linear Transformations and Matrices

\chapter{Linear Transformations and Matrices}
\label{ch:transformations}

\begin{abstract}
Linear transformations are the heart of linear algebra. In this chapter, we learn how any linear transformation can be represented by a matrix, and how matrix multiplication relates to composing transformations.
\end{abstract}

% ============================================
\section{What is a Transformation?}
% ============================================

\begin{definition}[Transformation]
A \vocab{transformation} is a function that takes vectors to other vectors:
\[
T: \Rtwo \to \Rtwo
\]
Each input vector is mapped to an output vector.
\end{definition}

\begin{intuition}
We use the word ``transformation'' instead of ``function'' because we want to think about \textbf{movement}. Imagine every vector in space being ``moved'' to a new location.

Picture the grid lines of a coordinate system. A transformation stretches, compresses, rotates, or otherwise changes this grid.
\end{intuition}

% ============================================
\section{Linear Transformation}
% ============================================

\begin{definition}[Linear Transformation]
A transformation $T$ is a \vocab{linear transformation} if and only if these two conditions hold:
\begin{enumerate}
    \item \textbf{Additivity:} $T(\vv + \vw) = T(\vv) + T(\vw)$
    \item \textbf{Homogeneity:} $T(c\vv) = c \cdot T(\vv)$
\end{enumerate}
for any vectors $\vv, \vw$ and any scalar $c$.
\end{definition}

\begin{intuition}
A linear transformation can be recognized by two geometric properties:
\begin{enumerate}
    \item \textbf{Straight lines remain straight} (they don't curve)
    \item \textbf{The origin stays fixed}
\end{enumerate}

If grid lines remain parallel and evenly spaced after the transformation, it's linear.
\end{intuition}

\begin{center}
\begin{tikzpicture}[scale=0.8]
    % Original grid
    \begin{scope}[shift={(0,0)}]
        \draw[grid] (-2,-2) grid (2,2);
        \draw[axis] (-2,0) -- (2,0);
        \draw[axis] (0,-2) -- (0,2);
        \draw[basis, -{Stealth[length=3mm]}] (0,0) -- (1,0) node[below] {$\vi$};
        \draw[basis, -{Stealth[length=3mm]}] (0,0) -- (0,1) node[left] {$\vj$};
        \node at (0,-3) {Before transformation};
    \end{scope}

    % Arrow
    \draw[-{Stealth[length=5mm]}, very thick] (3,0) -- (5,0) node[midway, above] {$T$};

    % Transformed grid (shear)
    \begin{scope}[shift={(8,0)}]
        \draw[blue!30, thin] (-2,-2) -- (0,2) -- (4,2) -- (2,-2) -- cycle;
        \foreach \y in {-1,0,1} {
            \draw[blue!30, thin] ({-2+\y},-2) -- ({\y+2},2);
        }
        \foreach \x in {-2,-1,0,1,2} {
            \draw[blue!30, thin] ({\x-1},-2) -- ({\x+1},2);
        }
        \draw[axis] (-2,0) -- (4,0);
        \draw[axis] (0,-2) -- (2,2);
        \draw[transformed, -{Stealth[length=3mm]}, very thick] (0,0) -- (1,0) node[below] {$T(\vi)$};
        \draw[transformed, -{Stealth[length=3mm]}, very thick] (0,0) -- (1,1) node[above left] {$T(\vj)$};
        \node at (1,-3) {After transformation};
    \end{scope}
\end{tikzpicture}
\end{center}

% ============================================
\section{The Matrix of a Linear Transformation}
% ============================================

\begin{theorem}[Fundamental Theorem]
Every linear transformation $T: \Rtwo \to \Rtwo$ is completely determined by knowing where $T$ sends the basis vectors $\vi$ and $\vj$.
\end{theorem}

\begin{proof}
Any vector $\vv = \twovec{x}{y}$ can be written as: $\vv = x\vi + y\vj$

Using linearity:
\[
T(\vv) = T(x\vi + y\vj) = xT(\vi) + yT(\vj)
\]
So if we know $T(\vi)$ and $T(\vj)$, we can compute $T(\vv)$ for any $\vv$.
\end{proof}

\begin{definition}[Transformation Matrix]
If $T(\vi) = \twovec{a}{c}$ and $T(\vj) = \twovec{b}{d}$, the \vocab{transformation matrix} of $T$ is:
\[
\mA = \twomat{a}{b}{c}{d}
\]
The first column is where $\vi$ lands, and the second column is where $\vj$ lands.
\end{definition}

\begin{important}
\textbf{Golden Rule:} The columns of a matrix = where the basis vectors land
\end{important}

% ============================================
\section{Matrix-Vector Multiplication}
% ============================================

\begin{definition}[Matrix Times Vector]
\[
\twomat{a}{b}{c}{d} \twovec{x}{y} = x\twovec{a}{c} + y\twovec{b}{d} = \twovec{ax + by}{cx + dy}
\]
\end{definition}

\begin{intuition}
Multiplying a matrix by a vector means:
\begin{enumerate}
    \item Multiply the $x$ component of the input by the first column
    \item Multiply the $y$ component of the input by the second column
    \item Add the results
\end{enumerate}
This is exactly the linear combination of the new basis vectors!
\end{intuition}

\begin{example}
Rotation by $90°$ counterclockwise:
\[
T(\vi) = \twovec{0}{1}, \quad T(\vj) = \twovec{-1}{0}
\]
So the rotation matrix is:
\[
\mR = \twomat{0}{-1}{1}{0}
\]
Check: $\mR\twovec{1}{0} = \twovec{0}{1}$ \checkmark
\end{example}

% ============================================
\section{Examples of Important Transformations}
% ============================================

\subsection{Rotation}

\begin{definition}[Rotation Matrix]
Rotation by angle $\theta$ counterclockwise:
\[
\mR_\theta = \twomat{\cos\theta}{-\sin\theta}{\sin\theta}{\cos\theta}
\]
\end{definition}

\begin{example}
Rotation by $45°$:
\[
\mR_{45°} = \twomat{\frac{\sqrt{2}}{2}}{-\frac{\sqrt{2}}{2}}{\frac{\sqrt{2}}{2}}{\frac{\sqrt{2}}{2}}
\]
\end{example}

\subsection{Scaling}

\begin{definition}[Scaling Matrix]
Scaling by factors $s_x$ along $x$ and $s_y$ along $y$:
\[
\mS = \twomat{s_x}{0}{0}{s_y}
\]
\end{definition}

\begin{practical}
\textbf{Application in Computer Graphics:} To enlarge or shrink an image:
\begin{itemize}
    \item $s_x = s_y = 2$: Image doubles in size
    \item $s_x = 1, s_y = 0.5$: Image is compressed vertically
\end{itemize}
\end{practical}

\subsection{Shear}

\begin{definition}[Horizontal Shear Matrix]
\[
\mH = \twomat{1}{k}{0}{1}
\]
where $k$ is the shear amount.
\end{definition}

\begin{center}
\begin{tikzpicture}[scale=1]
    % Original square
    \begin{scope}[shift={(0,0)}]
        \draw[fill=blue!20] (0,0) -- (1,0) -- (1,1) -- (0,1) -- cycle;
        \draw[basis] (0,0) -- (1,0) node[midway, below] {$\vi$};
        \draw[basis] (0,0) -- (0,1) node[midway, left] {$\vj$};
        \node at (0.5,-0.7) {Original};
    \end{scope}

    % Sheared
    \begin{scope}[shift={(4,0)}]
        \draw[fill=blue!20] (0,0) -- (1,0) -- (2,1) -- (1,1) -- cycle;
        \draw[transformed, very thick] (0,0) -- (1,0) node[midway, below] {$T(\vi)$};
        \draw[transformed, very thick] (0,0) -- (1,1) node[midway, left] {$T(\vj)$};
        \node at (1,-0.7) {Shear with $k=1$};
    \end{scope}
\end{tikzpicture}
\end{center}

\subsection{Reflection}

\begin{example}
Reflection across the $x$-axis:
\[
\twomat{1}{0}{0}{-1}
\]

Reflection across the line $y = x$:
\[
\twomat{0}{1}{1}{0}
\]
\end{example}

% ============================================
\section{Matrix Multiplication = Composing Transformations}
% ============================================

\begin{theorem}[Composition of Transformations]
If $\mA$ is the matrix of transformation $T_1$ and $\mB$ is the matrix of transformation $T_2$, then:
\[
\mB \cdot \mA = \text{matrix of } (T_2 \circ T_1)
\]
meaning first apply $T_1$, then apply $T_2$.
\end{theorem}

\begin{warning}
Order matters! $\mA\mB \neq \mB\mA$ in general.

Rotation then shear $\neq$ Shear then rotation
\end{warning}

\begin{definition}[Matrix Multiplication]
\[
\twomat{a}{b}{c}{d} \twomat{e}{f}{g}{h} = \twomat{ae+bg}{af+bh}{ce+dg}{cf+dh}
\]
\end{definition}

\begin{intuition}
To compute the columns of $\mB\mA$:
\begin{itemize}
    \item First column: $\mB$ times the first column of $\mA$ = where $\vi$ lands after both transformations
    \item Second column: $\mB$ times the second column of $\mA$ = where $\vj$ lands after both transformations
\end{itemize}
\end{intuition}

\begin{example}
Rotation by $90°$ followed by shear:
\[
\underbrace{\twomat{1}{1}{0}{1}}_{\text{shear}} \underbrace{\twomat{0}{-1}{1}{0}}_{\text{rotation}} = \twomat{1}{-1}{1}{0}
\]

Check: $\vi \to \twovec{0}{1} \to \twovec{1}{1}$ \checkmark
\end{example}

% ============================================
\section{Practical Applications}
% ============================================

\begin{practical}
\textbf{Computer Graphics and Video Games}

Every object in a game is described by a set of vectors (vertices). To move, rotate, or resize an object, we just multiply all vertices by the appropriate matrix.

An animation = a sequence of matrix multiplications
\end{practical}

\begin{practical}
\textbf{Image Processing}

Image filters like blur, sharpen, and edge detection are implemented using matrix multiplication.
\end{practical}

% ============================================
\section{Exercises}
% ============================================

\begin{exercise}
Write the matrix of the transformation that takes $\vi$ to $\twovec{2}{1}$ and $\vj$ to $\twovec{-1}{3}$.
\end{exercise}

\begin{exercise}
Compute the following product:
\[
\twomat{1}{2}{3}{4} \twovec{5}{6}
\]
\end{exercise}

\begin{exercise}
Write the matrix for rotation by $180°$ and show that it equals $-\mI$.
\end{exercise}

\begin{exercise}
Multiply the following two matrices (in both orders) and show that $\mA\mB \neq \mB\mA$:
\[
\mA = \twomat{1}{2}{0}{1}, \quad \mB = \twomat{0}{1}{1}{0}
\]
\end{exercise}

\begin{exercise}[Challenge]
Find the matrix for reflection across the line $y = 2x$.

\textit{Hint: This line makes an angle of $\arctan(2)$ with the $x$-axis.}
\end{exercise}

\begin{problem}
Show that the composition of two rotations by angles $\alpha$ and $\beta$ equals a rotation by angle $\alpha + \beta$.
\end{problem}

\begin{problem}
If $\mA^2 = \mA$ (i.e., $\mA$ is a projection matrix), what is the geometric interpretation?
\end{problem}

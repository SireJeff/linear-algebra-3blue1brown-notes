% lecture07.tex - Cross Products and Applications
% Chapter 7: Cross Products and Applications

\chapter{Cross Products and Applications}
\label{ch:crossproduct}

\begin{abstract}
The cross product (vector product) is an operation that creates a vector perpendicular to two vectors in three-dimensional space. In this chapter, we study its definition, properties, and deep connection to determinants.
\end{abstract}

% ============================================
\section{Cross Product in Two Dimensions}
% ============================================

\begin{definition}[2D Cross Product]
For two vectors in $\Rtwo$:
\[
\vv \times \vw = v_1 w_2 - v_2 w_1 = \det\twomat{v_1}{w_1}{v_2}{w_2}
\]
The result is a \textbf{number} (not a vector).
\end{definition}

\begin{intuition}
2D cross product = signed area of the parallelogram formed by the two vectors

Positive sign: $\vw$ is to the left of $\vv$\\
Negative sign: $\vw$ is to the right of $\vv$
\end{intuition}

\begin{center}
\begin{tikzpicture}[scale=1.2]
    % Parallelogram
    \fill[blue!20] (0,0) -- (2,0.5) -- (3,2) -- (1,1.5) -- cycle;
    \draw[vector, red!70, very thick] (0,0) -- (2,0.5) node[midway, below] {$\vv$};
    \draw[vector, green!60!black, very thick] (0,0) -- (1,1.5) node[midway, left] {$\vw$};
    \draw[dashed] (2,0.5) -- (3,2);
    \draw[dashed] (1,1.5) -- (3,2);
    \node at (1.5,1) {Area $= |\vv \times \vw|$};
\end{tikzpicture}
\end{center}

% ============================================
\section{Cross Product in Three Dimensions}
% ============================================

\begin{definition}[3D Cross Product]
For $\vv = \threevec{v_1}{v_2}{v_3}$ and $\vw = \threevec{w_1}{w_2}{w_3}$:
\[
\vv \times \vw = \threevec{v_2 w_3 - v_3 w_2}{v_3 w_1 - v_1 w_3}{v_1 w_2 - v_2 w_1}
\]
\end{definition}

\begin{theorem}[Determinant Formula]
\[
\vv \times \vw = \det\threemat{\vi}{\vj}{\vk}{v_1}{v_2}{v_3}{w_1}{w_2}{w_3}
\]
(symbolic expansion along the first row)
\end{theorem}

\begin{intuition}
The cross product $\vv \times \vw$:
\begin{itemize}
    \item \textbf{Direction:} Perpendicular to both vectors (right-hand rule)
    \item \textbf{Magnitude:} Area of the parallelogram formed by $\vv$ and $\vw$
\end{itemize}
\end{intuition}

\begin{center}
\begin{tikzpicture}[scale=1]
    % 3D coordinate hint
    \draw[axis] (0,0) -- (3,0) node[right] {$x$};
    \draw[axis] (0,0) -- (0,3) node[above] {$z$};
    \draw[axis] (0,0) -- (-1,-0.7) node[below left] {$y$};

    % Vectors
    \draw[vector, red!70, very thick] (0,0) -- (2,-0.5) node[right] {$\vv$};
    \draw[vector, green!60!black, very thick] (0,0) -- (-0.5,0) node[left] {$\vw$};
    \draw[vector, blue!70, ultra thick] (0,0) -- (0,2) node[above] {$\vv \times \vw$};

    % Parallelogram base
    \fill[gray!20] (0,0) -- (2,-0.5) -- (1.5,-0.5) -- (-0.5,0) -- cycle;
\end{tikzpicture}
\end{center}

% ============================================
\section{Properties of the Cross Product}
% ============================================

\begin{theorem}[Main Properties]
\begin{enumerate}
    \item \textbf{Anti-commutativity:} $\vv \times \vw = -(\vw \times \vv)$
    \item \textbf{Distributivity:} $\vv \times (\vw + \vu) = \vv \times \vw + \vv \times \vu$
    \item \textbf{Scalar multiplication:} $(c\vv) \times \vw = c(\vv \times \vw)$
    \item \textbf{Orthogonality:} $\vv \times \vw \perp \vv$ and $\vv \times \vw \perp \vw$
    \item \textbf{Self-cross is zero:} $\vv \times \vv = \vzero$
\end{enumerate}
\end{theorem}

\begin{warning}
The cross product is \textbf{not commutative}!
\[
\vv \times \vw \neq \vw \times \vv
\]
In fact, they are opposite to each other.
\end{warning}

\begin{theorem}[Magnitude Formula]
\[
\norm{\vv \times \vw} = \norm{\vv} \norm{\vw} \sin\theta
\]
where $\theta$ is the angle between the two vectors.
\end{theorem}

% ============================================
\section{Cross Product from the Linear Transformation Viewpoint}
% ============================================

\begin{intuition}
A deeper view: the cross product can be defined through duality.

The function $f(\vx) = \det[\vv \,|\, \vw \,|\, \vx]$ is a linear function of $\vx$. By duality, there must exist a vector $\vp$ such that:
\[
f(\vx) = \vp \cdot \vx
\]
This $\vp$ is exactly $\vv \times \vw$!
\end{intuition}

\begin{theorem}
\[
\det[\vv \,|\, \vw \,|\, \vx] = (\vv \times \vw) \cdot \vx
\]
\end{theorem}

% ============================================
\section{Basis Vector Products}
% ============================================

\begin{theorem}[Cross Products of Basis Vectors]
\begin{align*}
\vi \times \vj &= \vk & \vj \times \vi &= -\vk \\
\vj \times \vk &= \vi & \vk \times \vj &= -\vi \\
\vk \times \vi &= \vj & \vi \times \vk &= -\vj
\end{align*}
\end{theorem}

\begin{example}
\[
\threevec{2}{3}{4} \times \threevec{5}{6}{7}
\]
\begin{align*}
&= (3 \times 7 - 4 \times 6)\vi - (2 \times 7 - 4 \times 5)\vj + (2 \times 6 - 3 \times 5)\vk \\
&= (21-24)\vi - (14-20)\vj + (12-15)\vk \\
&= -3\vi + 6\vj - 3\vk = \threevec{-3}{6}{-3}
\end{align*}
\end{example}

% ============================================
\section{Applications}
% ============================================

\begin{practical}
\textbf{Physics: Torque}

Torque of force $\vec{F}$ about a point at distance $\vec{r}$:
\[
\vec{\tau} = \vec{r} \times \vec{F}
\]
The direction of torque is perpendicular to the plane containing the force and the lever arm.
\end{practical}

\begin{practical}
\textbf{Physics: Lorentz Force}

Force on a charged particle moving in a magnetic field:
\[
\vec{F} = q\vec{v} \times \vec{B}
\]
\end{practical}

\begin{practical}
\textbf{Computer Graphics: Surface Normal}

To find the normal vector to a triangular surface with vertices $A$, $B$, $C$:
\[
\vec{n} = (\vec{B} - \vec{A}) \times (\vec{C} - \vec{A})
\]
\end{practical}

\begin{practical}
\textbf{Computing Triangle Area}

Area of triangle with vertices $A$, $B$, $C$:
\[
\text{Area} = \frac{1}{2}\norm{(\vec{B}-\vec{A}) \times (\vec{C}-\vec{A})}
\]
\end{practical}

% ============================================
\section{Triple Products}
% ============================================

\begin{definition}[Scalar Triple Product]
\[
\va \cdot (\vb \times \vc) = \det[\va \,|\, \vb \,|\, \vc]
\]
Result = signed volume of the parallelepiped formed by the three vectors
\end{definition}

\begin{theorem}[Cyclic Property]
\[
\va \cdot (\vb \times \vc) = \vb \cdot (\vc \times \va) = \vc \cdot (\va \times \vb)
\]
\end{theorem}

% ============================================
\section{Exercises}
% ============================================

\begin{exercise}
Compute the cross product:
\[
\va = \threevec{1}{2}{3}, \quad \vb = \threevec{4}{5}{6}
\]
\end{exercise}

\begin{exercise}
Find the area of the parallelogram with sides $\va = \threevec{1}{0}{0}$ and $\vb = \threevec{1}{1}{0}$.
\end{exercise}

\begin{exercise}
Show that $\va \times \vb$ is perpendicular to both $\va$ and $\vb$.
\end{exercise}

\begin{exercise}
Find the normal vector to the plane passing through points $(1,0,0)$, $(0,1,0)$, $(0,0,1)$.
\end{exercise}

\begin{exercise}[Challenge]
Show that:
\[
\norm{\va \times \vb}^2 + (\va \cdot \vb)^2 = \norm{\va}^2 \norm{\vb}^2
\]
(This is called Lagrange's identity)
\end{exercise}

\begin{problem}
Calculate the volume of the parallelepiped with edges:
\[
\va = \threevec{1}{1}{0}, \quad \vb = \threevec{0}{1}{1}, \quad \vc = \threevec{1}{0}{1}
\]
\end{problem}

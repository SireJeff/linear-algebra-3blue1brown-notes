% main.tex - English Linear Algebra Lecture Notes
% Master Document - Compile with pdfLaTeX
%
% Linear Algebra: Essence of Linear Algebra
% Based on 3Blue1Brown video series
%
% Compilation: pdflatex main.tex (run 2 times for references)

% preamble.tex - Persian Linear Algebra Lecture Notes
% XeLaTeX configuration for Persian RTL support

\documentclass[12pt,a4paper,openany,oneside]{book}

% === Page Geometry ===
\usepackage{geometry}
\geometry{
    top=2.5cm,
    bottom=2.5cm,
    left=2.5cm,
    right=2.5cm,
    headheight=14pt
}

% === Math Packages (MUST come before xepersian) ===
\usepackage{amsmath}
\usepackage{amssymb}
\usepackage{amsthm}
\usepackage{mathtools}
\usepackage{bm}

% === Graphics and Diagrams ===
\usepackage{graphicx}
\usepackage{tikz}
\usepackage{pgfplots}
\pgfplotsset{compat=1.18}
\usetikzlibrary{arrows.meta, calc, positioning, decorations.pathreplacing, patterns}

% === Colors and Boxes ===
\usepackage{xcolor}
\usepackage{tcolorbox}
\tcbuselibrary{skins,breakable,theorems}

% === Tables and Lists ===
\usepackage{booktabs}
\usepackage{array}
\usepackage{enumitem}

% === Hyperlinks (BEFORE xepersian) ===
\usepackage[colorlinks=true,linkcolor=blue!70!black,urlcolor=blue!70!black,citecolor=green!50!black]{hyperref}

% === Persian Support - MUST BE LAST ===
\usepackage{xepersian}

% === Font Configuration ===
% Using Tahoma as fallback (supports Persian/Arabic)
% For better typography, install XB Zar and XB Titre fonts
\settextfont[Scale=1.0]{Tahoma}
\setlatintextfont{Times New Roman}
\setdigitfont{Tahoma}
\defpersianfont\titr[Scale=1.2]{Tahoma}
\defpersianfont\nastaliq[Scale=1.0]{Tahoma}

% === Chapter and Section Styling ===
\renewcommand{\chaptername}{درس}
\renewcommand{\contentsname}{فهرست مطالب}
\renewcommand{\listfigurename}{فهرست شکل‌ها}
\renewcommand{\listtablename}{فهرست جدول‌ها}
\renewcommand{\indexname}{نمایه}
\renewcommand{\figurename}{شکل}
\renewcommand{\tablename}{جدول}
\renewcommand{\partname}{بخش}
\renewcommand{\appendixname}{پیوست}
\renewcommand{\abstractname}{چکیده}
\renewcommand{\bibname}{منابع}

% === Theorem Environments (Persian) ===
\theoremstyle{definition}
\newtheorem{definition}{تعریف}[chapter]
\newtheorem{example}{مثال}[chapter]
\newtheorem{exercise}{تمرین}[chapter]
\newtheorem{problem}{مسئله}[chapter]

\theoremstyle{plain}
\newtheorem{theorem}{قضیه}[chapter]
\newtheorem{lemma}{لم}[chapter]
\newtheorem{proposition}{گزاره}[chapter]
\newtheorem{corollary}{نتیجه}[chapter]

\theoremstyle{remark}
\newtheorem{remark}{نکته}[chapter]
\newtheorem{note}{یادداشت}[chapter]

% === Custom tcolorbox Environments ===

% Geometric Intuition Box
\newtcolorbox{intuition}{
    enhanced,
    colback=blue!5,
    colframe=blue!75!black,
    fonttitle=\titr,
    title={شهود هندسی},
    breakable,
    left=5mm,
    right=5mm,
    top=3mm,
    bottom=3mm,
    arc=2mm,
    boxrule=1pt
}

% Practical Application Box
\newtcolorbox{practical}{
    enhanced,
    colback=green!5,
    colframe=green!60!black,
    fonttitle=\titr,
    title={کاربرد عملی},
    breakable,
    left=5mm,
    right=5mm,
    top=3mm,
    bottom=3mm,
    arc=2mm,
    boxrule=1pt
}

% Warning Box
\newtcolorbox{warning}{
    enhanced,
    colback=red!5,
    colframe=red!75!black,
    fonttitle=\titr,
    title={هشدار},
    breakable,
    left=5mm,
    right=5mm,
    top=3mm,
    bottom=3mm,
    arc=2mm,
    boxrule=1pt
}

% Important Note Box
\newtcolorbox{important}{
    enhanced,
    colback=orange!5,
    colframe=orange!75!black,
    fonttitle=\titr,
    title={نکته مهم},
    breakable,
    left=5mm,
    right=5mm,
    top=3mm,
    bottom=3mm,
    arc=2mm,
    boxrule=1pt
}

% Summary Box
\newtcolorbox{summary}{
    enhanced,
    colback=purple!5,
    colframe=purple!75!black,
    fonttitle=\titr,
    title={خلاصه},
    breakable,
    left=5mm,
    right=5mm,
    top=3mm,
    bottom=3mm,
    arc=2mm,
    boxrule=1pt
}

% === Page Header/Footer ===
\usepackage{fancyhdr}
\pagestyle{fancy}
\fancyhf{}
\fancyhead[LO]{\leftmark}
\fancyhead[RE]{\rightmark}
\fancyfoot[C]{\thepage}
\renewcommand{\headrulewidth}{0.4pt}
\renewcommand{\footrulewidth}{0pt}

% === Spacing ===
\usepackage{setspace}
\onehalfspacing

% === Custom Commands for Math ===
% commands.tex - Custom Math Commands for Linear Algebra
% English Linear Algebra Lecture Notes

% === Vector Notation ===
% Bold vectors (column vectors)
\newcommand{\vect}[1]{\mathbf{#1}}
\newcommand{\vv}{\vect{v}}
\newcommand{\vw}{\vect{w}}
\newcommand{\vu}{\vect{u}}
\newcommand{\vx}{\vect{x}}
\newcommand{\vy}{\vect{y}}
\newcommand{\vz}{\vect{z}}
\newcommand{\va}{\vect{a}}
\newcommand{\vb}{\vect{b}}
\newcommand{\vc}{\vect{c}}

% Basis vectors (with hat notation)
\newcommand{\vi}{\hat{\imath}}
\newcommand{\vj}{\hat{\jmath}}
\newcommand{\vk}{\hat{k}}

% Zero vector
\newcommand{\vzero}{\vect{0}}

% === Matrix Notation ===
\newcommand{\mat}[1]{\mathbf{#1}}
\newcommand{\mA}{\mat{A}}
\newcommand{\mB}{\mat{B}}
\newcommand{\mC}{\mat{C}}
\newcommand{\mD}{\mat{D}}
\newcommand{\mI}{\mat{I}}
\newcommand{\mP}{\mat{P}}
\newcommand{\mQ}{\mat{Q}}
\newcommand{\mT}{\mat{T}}
\newcommand{\mM}{\mat{M}}
\newcommand{\mR}{\mat{R}}
\newcommand{\mS}{\mat{S}}
\newcommand{\mH}{\mat{H}}

% Matrix with brackets
\newcommand{\bmat}[1]{\begin{bmatrix}#1\end{bmatrix}}
\newcommand{\pmat}[1]{\begin{pmatrix}#1\end{pmatrix}}
\newcommand{\vmat}[1]{\begin{vmatrix}#1\end{vmatrix}}

% === Common Operations ===
% Transpose
\newcommand{\trans}{^{\mathsf{T}}}

% Inverse
\newcommand{\inv}{^{-1}}

% Norm and absolute value
\newcommand{\norm}[1]{\left\| #1 \right\|}
\newcommand{\abs}[1]{\left| #1 \right|}

% === Important Spaces ===
\newcommand{\R}{\mathbb{R}}
\newcommand{\C}{\mathbb{C}}
\newcommand{\N}{\mathbb{N}}
\newcommand{\Z}{\mathbb{Z}}
\newcommand{\Q}{\mathbb{Q}}

% Rn space
\newcommand{\Rtwo}{\mathbb{R}^2}
\newcommand{\Rthree}{\mathbb{R}^3}

% === Linear Algebra Operators ===
\DeclareMathOperator{\spn}{span}
\DeclareMathOperator{\Span}{Span}
\DeclareMathOperator{\Null}{Null}
\DeclareMathOperator{\Col}{Col}
\DeclareMathOperator{\Row}{Row}
\DeclareMathOperator{\rank}{rank}
\DeclareMathOperator{\tr}{tr}
\DeclareMathOperator{\proj}{proj}

% === Common Matrices ===
% 2x2 matrix shorthand
\newcommand{\twomat}[4]{\begin{bmatrix} #1 & #2 \\ #3 & #4 \end{bmatrix}}

% 3x3 matrix shorthand
\newcommand{\threemat}[9]{\begin{bmatrix} #1 & #2 & #3 \\ #4 & #5 & #6 \\ #7 & #8 & #9 \end{bmatrix}}

% 2D column vector
\newcommand{\twovec}[2]{\begin{bmatrix} #1 \\ #2 \end{bmatrix}}

% 3D column vector
\newcommand{\threevec}[3]{\begin{bmatrix} #1 \\ #2 \\ #3 \end{bmatrix}}

% === Change of Basis ===
\newcommand{\coords}[2]{[#1]_{#2}}

% === System of Equations ===
\newcommand{\system}[1]{\left\{\begin{aligned} #1 \end{aligned}\right.}

% === Emphasizing ===
\newcommand{\vocab}[1]{\textbf{\textcolor{blue!70!black}{#1}}}


% === TikZ styles for linear algebra diagrams ===
\tikzset{
    vector/.style={-{Stealth[length=3mm]}, thick},
    axis/.style={-{Stealth[length=2mm]}, gray},
    grid/.style={very thin, gray!30},
    transformed/.style={blue!70},
    original/.style={black},
    basis/.style={red!70!black, very thick},
    point/.style={circle, fill, inner sep=1.5pt}
}


\begin{document}

% === Title Page ===
\begin{titlepage}
\centering
\vspace*{1cm}

{\Huge\bfseries Linear Algebra}\\[0.5cm]
{\LARGE\bfseries Essence of Linear Algebra}\\[2cm]

{\Large Comprehensive Lecture Notes}\\[0.5cm]
{\large with emphasis on geometric intuition and practical applications}\\[2cm]

\begin{tikzpicture}[scale=0.6]
    % Background grid
    \draw[grid] (-3,-3) grid (3,3);

    % Axes
    \draw[axis] (-3.5,0) -- (3.5,0);
    \draw[axis] (0,-3.5) -- (0,3.5);

    % Basis vectors
    \draw[basis, -{Stealth[length=4mm]}] (0,0) -- (1,0) node[below right] {$\vi$};
    \draw[basis, -{Stealth[length=4mm]}] (0,0) -- (0,1) node[above left] {$\vj$};

    % A transformed vector
    \draw[vector, blue!70, very thick] (0,0) -- (2,1.5) node[above right] {$\vv$};

    % Another vector
    \draw[vector, green!60!black, very thick] (0,0) -- (-1,2) node[above left] {$\vw$};
\end{tikzpicture}\\[2cm]

{\large Based on the video series}\\[0.3cm]
{\Large\textbf{Essence of Linear Algebra}}\\[0.3cm]
{\large by Grant Sanderson}\\[0.3cm]
{\small YouTube Channel: 3Blue1Brown}\\[1.5cm]

{\small \today}

\end{titlepage}

% === Credits Page ===
\begin{titlepage}
\centering
\vspace*{3cm}

{\Large\bfseries Acknowledgments}

\vspace{2cm}

{\large Special thanks to:}

\vspace{1cm}

Ebrahim Khosravani\\[0.25cm]
Mohammadhasan Shiri\\[0.25cm]
Zahra Amiri\\[0.25cm]
Amin Azizi\\[0.25cm]
Bita Qorbani\\[0.25cm]
Yasamin Yazdani\\[0.25cm]
Sepehr Talaei

\vspace{2cm}

{\small Based on the ``Essence of Linear Algebra'' video series by 3Blue1Brown}

\vspace{1cm}

{\small Sharif University of Technology Physics Scientific Association}
\end{titlepage}

% === Table of Contents ===
\tableofcontents

% === Preface ===
\chapter*{Preface}
\addcontentsline{toc}{chapter}{Preface}

These notes are based on the famous ``Essence of Linear Algebra'' video series by Grant Sanderson from the 3Blue1Brown YouTube channel. The main goal of this series is to provide an intuitive and geometric understanding of linear algebra concepts.

\section*{Features of These Notes}
\begin{itemize}
    \item \textbf{Geometric Intuition:} Emphasis on visual understanding of concepts, not just calculations
    \item \textbf{Rigorous Definitions:} Standard mathematical formulations
    \item \textbf{Practical Applications:} Applications in physics, computer science, and engineering
    \item \textbf{Exercises:} Diverse problems to reinforce learning
\end{itemize}

\section*{Prerequisites}
\begin{itemize}
    \item Familiarity with basic mathematics (high school algebra)
    \item Understanding of Cartesian coordinate systems
    \item A desire for deep understanding of mathematics!
\end{itemize}

\section*{How to Use These Notes}
Each chapter contains the following sections:
\begin{description}
    \item[Definitions:] Key concepts with mathematical notation
    \item[Geometric Intuition:] Visual explanations (in blue boxes)
    \item[Practical Applications:] Real-world examples (in green boxes)
    \item[Exercises:] Problems for practice
\end{description}

\begin{flushright}
Best wishes for learning linear algebra
\end{flushright}

% === LECTURES ===

% Chapter 1: Introduction to Vectors
% lecture01.tex - Introduction to Vectors
% Chapter 1: Introduction to Vectors

\chapter{Introduction to Vectors}
\label{ch:vectors}

\begin{abstract}
Vectors are the fundamental building block of linear algebra. In this chapter, we explore three different perspectives on vectors: the physics view, the computer science view, and the mathematics view. We also learn the basic operations of vector addition and scalar multiplication.
\end{abstract}

% ============================================
\section{Three Perspectives on Vectors}
% ============================================

The concept of a vector has different meanings depending on your field of study. Understanding these three perspectives helps us form a complete picture of vectors.

% -------------------------------------------
\subsection{The Physics Perspective}
% -------------------------------------------

\begin{definition}[Vector from Physics Perspective]
A vector is an \vocab{arrow} in space defined by two properties:
\begin{enumerate}
    \item \textbf{Length} (magnitude)
    \item \textbf{Direction}
\end{enumerate}
As long as these two properties remain the same, a vector can be moved anywhere in space and still be considered the same vector.
\end{definition}

\begin{intuition}
Draw an arrow on paper. Now move it to another location without rotating it or changing its size. From a physicist's perspective, this is still the same vector!

For example, the gravitational force on an apple always points downward with a constant magnitude---regardless of where the apple is in the room.
\end{intuition}

\begin{center}
\begin{tikzpicture}[scale=1.2]
    % Draw multiple copies of same vector
    \draw[vector, blue!70, very thick] (0,0) -- (2,1) node[midway, above] {$\vv$};
    \draw[vector, blue!70, very thick] (3,1) -- (5,2) node[midway, above] {$\vv$};
    \draw[vector, blue!70, very thick] (-1,2) -- (1,3) node[midway, above] {$\vv$};

    % Note
    \node at (2,-0.5) {\small All of these are the same vector};
\end{tikzpicture}
\end{center}

% -------------------------------------------
\subsection{The Computer Science Perspective}
% -------------------------------------------

\begin{definition}[Vector from Computer Science Perspective]
A vector is an \vocab{ordered list of numbers}. For example:
\[
\vv = \twovec{3}{-2}
\]
In this view, ``vector'' is almost synonymous with ``list.''
\end{definition}

\begin{practical}
\textbf{Example: Housing Price Model}

Suppose you want to model houses based on two features:
\begin{itemize}
    \item Square footage (in square meters)
    \item Price (in thousands of dollars)
\end{itemize}

Each house is a two-dimensional vector:
\[
\text{House}_1 = \twovec{85}{250}, \quad
\text{House}_2 = \twovec{120}{420}, \quad
\text{House}_3 = \twovec{65}{180}
\]

\textbf{Important:} The order of numbers matters! $\twovec{85}{250}$ is different from $\twovec{250}{85}$.
\end{practical}

% -------------------------------------------
\subsection{The Mathematics Perspective}
% -------------------------------------------

\begin{definition}[Vector from Mathematics Perspective]
Mathematicians define vectors abstractly: a vector is anything on which you can perform these two operations:
\begin{enumerate}
    \item \textbf{Add two vectors}
    \item \textbf{Multiply a vector by a number (scalar)}
\end{enumerate}
We will explore the details of this definition in the final chapter (Abstract Vector Spaces).
\end{definition}

\begin{important}
Throughout this course, we visualize a vector as an \textbf{arrow in a coordinate system} with its \textbf{tail at the origin}. This geometric perspective helps us understand concepts more deeply.
\end{important}

% ============================================
\section{Coordinate Systems and Vector Representation}
% ============================================

\begin{definition}[Two-Dimensional Cartesian Coordinate System]
A coordinate system consists of:
\begin{itemize}
    \item Horizontal axis: \vocab{$x$-axis}
    \item Vertical axis: \vocab{$y$-axis}
    \item Point of intersection: \vocab{origin}
\end{itemize}
\end{definition}

\begin{center}
\begin{tikzpicture}[scale=1.5]
    % Grid
    \draw[grid] (-0.5,-0.5) grid (4.5,3.5);

    % Axes
    \draw[axis, thick] (-0.5,0) -- (4.5,0) node[right] {$x$};
    \draw[axis, thick] (0,-0.5) -- (0,3.5) node[above] {$y$};

    % Origin label
    \node[below left] at (0,0) {O};

    % Tick marks
    \foreach \x in {1,2,3,4} {
        \draw (\x,0.1) -- (\x,-0.1) node[below] {\small $\x$};
    }
    \foreach \y in {1,2,3} {
        \draw (0.1,\y) -- (-0.1,\y) node[left] {\small $\y$};
    }

    % Vector
    \draw[vector, blue!70, ultra thick] (0,0) -- (3,2);
    \node[above right, blue!70!black] at (3,2) {$\vv = \twovec{3}{2}$};

    % Dashed lines to show coordinates
    \draw[dashed, red!70] (3,0) -- (3,2);
    \draw[dashed, red!70] (0,2) -- (3,2);

    % Coordinate labels
    \node[below, red!70!black] at (3,0) {$3$};
    \node[left, red!70!black] at (0,2) {$2$};
\end{tikzpicture}
\end{center}

\begin{definition}[Coordinates of a Vector]
The \vocab{coordinates} of a vector are a pair of numbers describing how to get from the origin to the tip of the vector:
\begin{itemize}
    \item First number: how far to move along the $x$-axis (right is positive, left is negative)
    \item Second number: how far to move along the $y$-axis (up is positive, down is negative)
\end{itemize}
\end{definition}

\begin{intuition}
The coordinates $\twovec{3}{2}$ mean:
\begin{enumerate}
    \item From the origin, move 3 units to the right
    \item Then move 2 units up
    \item The tip of the vector is right here!
\end{enumerate}
\end{intuition}

% ============================================
\section{Three-Dimensional Space}
% ============================================

\begin{definition}[Three-Dimensional Vector]
In three-dimensional space, a third axis called the \vocab{$z$-axis} is added, perpendicular to both the $x$ and $y$ axes. Each vector is specified by three numbers:
\[
\vv = \threevec{a}{b}{c}
\]
\end{definition}

\begin{center}
\begin{tikzpicture}[scale=1.2]
    % 3D axes
    \draw[axis, thick] (0,0) -- (3,0) node[right] {$x$};
    \draw[axis, thick] (0,0) -- (0,3) node[above] {$z$};
    \draw[axis, thick] (0,0) -- (-1.5,-1) node[below left] {$y$};

    % Vector
    \draw[vector, blue!70, ultra thick] (0,0) -- (2,1.5);
    \draw[vector, green!60!black, ultra thick] (0,0) -- (-0.8,-0.5);

    % Labels
    \node[above right, blue!70!black] at (2,1.5) {$\threevec{2}{0}{1.5}$};
\end{tikzpicture}
\end{center}

% ============================================
\section{Vector Addition}
% ============================================

\begin{definition}[Adding Two Vectors - Geometric Method]
To add two vectors $\va$ and $\vb$:
\begin{enumerate}
    \item Draw vector $\va$
    \item Move vector $\vb$ so that its tail is at the tip of $\va$
    \item The sum vector is drawn from the origin (tail of $\va$) to the tip of $\vb$
\end{enumerate}
\end{definition}

\begin{center}
\begin{tikzpicture}[scale=1.3]
    % Grid
    \draw[grid] (-0.5,-0.5) grid (5.5,3.5);

    % Axes
    \draw[axis] (-0.5,0) -- (5.5,0);
    \draw[axis] (0,-0.5) -- (0,3.5);

    % Vector a
    \draw[vector, red!70, very thick] (0,0) -- (2,1) node[midway, below] {$\va$};

    % Vector b (translated)
    \draw[vector, blue!70, very thick] (2,1) -- (4,3) node[midway, right] {$\vb$};

    % Sum vector
    \draw[vector, purple!70, ultra thick] (0,0) -- (4,3) node[midway, above left] {$\va + \vb$};

    % Original b (dashed)
    \draw[vector, blue!40, dashed] (0,0) -- (2,2) node[midway, left] {\small $\vb$};
\end{tikzpicture}
\end{center}

\begin{definition}[Adding Two Vectors - Algebraic Method]
If $\va = \twovec{a_1}{a_2}$ and $\vb = \twovec{b_1}{b_2}$:
\[
\va + \vb = \twovec{a_1 + b_1}{a_2 + b_2}
\]
We add corresponding components together.
\end{definition}

\begin{example}
\[
\twovec{1}{2} + \twovec{3}{-1} = \twovec{1+3}{2+(-1)} = \twovec{4}{1}
\]
\end{example}

\begin{intuition}
Vector addition can be interpreted as \textbf{walking a path}:
\begin{itemize}
    \item First walk along vector $\va$
    \item Then walk along vector $\vb$
    \item The net effect is the same as walking directly along $\va + \vb$
\end{itemize}

Like walking: if you take 2 steps right, then 5 steps right, it's the same as taking 7 steps right.
\end{intuition}

\begin{practical}
\textbf{Example: Airplane Speed in Wind}

An airplane flies at $\twovec{500}{0}$ km/h toward the east (positive $x$ direction). Wind blows at $\twovec{0}{50}$ km/h toward the north (positive $y$ direction).

Actual speed of the airplane relative to ground:
\[
\vv_{\text{actual}} = \twovec{500}{0} + \twovec{0}{50} = \twovec{500}{50}
\]

The airplane moves both east and slightly north.
\end{practical}

% ============================================
\section{Scalar Multiplication}
% ============================================

\begin{definition}[Scalar Times Vector]
Multiplying a number (scalar) $c$ by a vector $\vv$ \vocab{scales} the vector by factor $c$:
\[
c \cdot \twovec{v_1}{v_2} = \twovec{c \cdot v_1}{c \cdot v_2}
\]
\end{definition}

\begin{center}
\begin{tikzpicture}[scale=1]
    % Original vector
    \draw[vector, blue!70, very thick] (0,0) -- (1.5,1) node[right] {$\vv$};

    % Scaled vectors
    \draw[vector, green!60!black, very thick] (3,0) -- (6,2) node[right] {$2\vv$};
    \draw[vector, orange!70, very thick] (8,0) -- (8.75,0.5) node[right] {$\frac{1}{2}\vv$};
    \draw[vector, red!70, very thick] (11,0) -- (8.3,-1.8) node[below] {$-1.8\vv$};

    % Labels below
    \node at (0.75,-0.5) {\small original};
    \node at (4.5,-0.5) {\small stretched};
    \node at (8.4,-0.5) {\small compressed};
    \node at (9.7,-2.3) {\small reversed \& stretched};
\end{tikzpicture}
\end{center}

\begin{important}
Effects of scalar multiplication:
\begin{itemize}
    \item $c > 1$: vector is \textbf{stretched}
    \item $0 < c < 1$: vector is \textbf{compressed}
    \item $c < 0$: vector is \textbf{reversed} and then scaled
    \item $c = 0$: vector becomes zero
    \item $c = 1$: vector is unchanged
\end{itemize}
\end{important}

\begin{warning}
If $c < 0$, the vector not only changes in magnitude but also \textbf{reverses direction}!
\end{warning}

\begin{example}
If $\vv = \twovec{4}{-2}$:
\begin{align*}
2\vv &= \twovec{8}{-4} & &\text{(doubled)} \\
-\vv &= \twovec{-4}{2} & &\text{(reversed)} \\
\frac{1}{2}\vv &= \twovec{2}{-1} & &\text{(halved)}
\end{align*}
\end{example}

\begin{definition}[Scalar]
In linear algebra, numbers that multiply vectors are called \vocab{scalars}. This name comes from the verb ``to scale.'' The word ``scalar'' is essentially synonymous with ``number.''
\end{definition}

% ============================================
\section{Basis Vectors}
% ============================================

\begin{definition}[Standard Basis Vectors in $\Rtwo$]
The two standard basis vectors are:
\[
\vi = \twovec{1}{0} \quad \text{and} \quad \vj = \twovec{0}{1}
\]
\end{definition}

\begin{center}
\begin{tikzpicture}[scale=2]
    % Grid
    \draw[grid] (-0.5,-0.5) grid (2.5,2.5);

    % Axes
    \draw[axis] (-0.5,0) -- (2.5,0) node[right] {$x$};
    \draw[axis] (0,-0.5) -- (0,2.5) node[above] {$y$};

    % Basis vectors
    \draw[basis, -{Stealth[length=4mm]}] (0,0) -- (1,0) node[below right] {$\vi$};
    \draw[basis, -{Stealth[length=4mm]}] (0,0) -- (0,1) node[above left] {$\vj$};

    % A general vector
    \draw[vector, blue!70, very thick] (0,0) -- (2,1.5);
    \node[above right, blue!70!black] at (2,1.5) {$\vv = 2\vi + 1.5\vj$};

    % Components
    \draw[dashed, orange] (0,0) -- (2,0) node[midway, below] {$2\vi$};
    \draw[dashed, orange] (2,0) -- (2,1.5) node[midway, right] {$1.5\vj$};
\end{tikzpicture}
\end{center}

\begin{theorem}
Any vector in $\Rtwo$ can be written as a linear combination of basis vectors:
\[
\twovec{a}{b} = a\vi + b\vj = a\twovec{1}{0} + b\twovec{0}{1}
\]
\end{theorem}

\begin{intuition}
When we write $\twovec{3}{2}$, we're really saying:
\begin{center}
``Take 3 copies of $\vi$ and 2 copies of $\vj$, then add them together''
\end{center}
That is:
\[
\twovec{3}{2} = 3\vi + 2\vj = 3\twovec{1}{0} + 2\twovec{0}{1}
\]
\end{intuition}

% ============================================
\section{Connecting the Perspectives}
% ============================================

\begin{summary}
\textbf{The power of linear algebra} lies in translating between different perspectives:
\begin{itemize}
    \item \textbf{Data analyst:} can visualize long lists of numbers as vectors in space
    \item \textbf{Physicist:} can describe motion and forces using numbers
    \item \textbf{Graphics programmer:} can implement geometric transformations using matrices
\end{itemize}
\end{summary}

% ============================================
\section{Exercises}
% ============================================

\begin{exercise}
Draw the following vectors in a coordinate system:
\[
\va = \twovec{3}{1}, \quad \vb = \twovec{-2}{4}, \quad \vc = \twovec{0}{-3}
\]
\end{exercise}

\begin{exercise}
Calculate the sum and difference of the following vectors:
\[
\va = \twovec{2}{5}, \quad \vb = \twovec{-1}{3}
\]
\begin{enumerate}[label=(\alph*)]
    \item $\va + \vb$
    \item $\va - \vb$
    \item $2\va + 3\vb$
\end{enumerate}
\end{exercise}

\begin{exercise}
If $\vv = \twovec{4}{-2}$, calculate and sketch the following vectors:
\begin{enumerate}[label=(\alph*)]
    \item $2\vv$
    \item $-\vv$
    \item $\frac{1}{2}\vv$
    \item $-2.5\vv$
\end{enumerate}
\end{exercise}

\begin{exercise}[Applied]
A ship moves at 30 km/h toward the north. The water current flows at 10 km/h toward the east. What is the ship's actual velocity relative to the shore?
\end{exercise}

\begin{exercise}
Show that for any vector $\vv$:
\[
\vv + (-\vv) = \vzero
\]
where $\vzero = \twovec{0}{0}$ is the zero vector.
\end{exercise}

\begin{exercise}
Prove that vector addition is commutative:
\[
\va + \vb = \vb + \va
\]
\end{exercise}

\begin{exercise}[Challenge]
Three points $A(1,2)$, $B(4,6)$, and $C(7,2)$ are given. Show that these three points form an isosceles triangle.

\textit{Hint: The length of vector $\twovec{a}{b}$ equals $\sqrt{a^2 + b^2}$.}
\end{exercise}

\begin{problem}
In three-dimensional space, vector $\vv = \threevec{2}{3}{-1}$ is given. Calculate:
\begin{enumerate}[label=(\alph*)]
    \item $3\vv$
    \item $\vv + \threevec{1}{-1}{2}$
    \item $-\frac{1}{2}\vv$
\end{enumerate}
\end{problem}


% Chapter 2: Linear Combinations, Span, and Basis
% lecture02.tex - ترکیبات خطی، فضای پوشش و بردارهای پایه
% Chapter 2: Linear Combinations, Span, and Basis Vectors

\chapter{ترکیبات خطی، فضای پوشش و پایه}
\label{ch:span-basis}

\begin{abstract}
در این درس با مفاهیم کلیدی «ترکیب خطی»، «فضای پوشش» \lr{(span)}، و «پایه» آشنا می‌شویم. این مفاهیم پایه و اساس درک عمیق جبر خطی هستند و به ما امکان می‌دهند بفهمیم چگونه بردارها فضا را «پر می‌کنند».
\end{abstract}

% ============================================
\section{نگاه جدید به مختصات}
% ============================================

در درس قبل، مختصات بردار را به عنوان «دستورالعمل حرکت» معرفی کردیم. اما نگاه دیگری وجود دارد که بسیار مهم است.

\begin{intuition}
وقتی می‌نویسیم $\vv = \twovec{3}{-2}$، به جای فکر کردن به «۳ واحد راست، ۲ واحد پایین»، اینطور فکر کنید:
\begin{itemize}
    \item عدد ۳ یک \textbf{اسکالر} است که بردار $\vi$ را مقیاس می‌کند
    \item عدد $-2$ یک \textbf{اسکالر} است که بردار $\vj$ را مقیاس می‌کند
    \item بردار نهایی، \textbf{مجموع} این دو بردار مقیاس‌شده است
\end{itemize}
\end{intuition}

\begin{center}
\begin{tikzpicture}[scale=1.3]
    % Grid
    \draw[grid] (-1.5,-2.5) grid (4.5,1.5);

    % Axes
    \draw[axis] (-1.5,0) -- (4.5,0) node[right] {$x$};
    \draw[axis] (0,-2.5) -- (0,1.5) node[above] {$y$};

    % Basis vectors (unit)
    \draw[basis, -{Stealth[length=3mm]}] (0,0) -- (1,0) node[below right] {$\vi$};
    \draw[basis, -{Stealth[length=3mm]}] (0,0) -- (0,1) node[above left] {$\vj$};

    % Scaled vectors
    \draw[vector, orange!70, very thick] (0,0) -- (3,0) node[midway, below] {$3\vi$};
    \draw[vector, green!60!black, very thick] (3,0) -- (3,-2) node[midway, right] {$-2\vj$};

    % Result vector
    \draw[vector, blue!70, ultra thick] (0,0) -- (3,-2) node[below right] {$\vv = 3\vi - 2\vj$};
\end{tikzpicture}
\end{center}

% ============================================
\section{ترکیب خطی}
% ============================================

\begin{definition}[ترکیب خطی]
\vocab{ترکیب خطی} دو بردار $\vv$ و $\vw$ عبارت است از:
\[
a\vv + b\vw
\]
که در آن $a$ و $b$ اسکالرهای دلخواه هستند.

به طور کلی، ترکیب خطی $n$ بردار $\vv_1, \vv_2, \ldots, \vv_n$ برابر است با:
\[
c_1\vv_1 + c_2\vv_2 + \cdots + c_n\vv_n = \sum_{i=1}^{n} c_i\vv_i
\]
\end{definition}

\begin{intuition}
نام «خطی» از کجا می‌آید؟ اگر یکی از اسکالرها را ثابت نگه دارید و دیگری را تغییر دهید، نوک بردار حاصل روی یک \textbf{خط راست} حرکت می‌کند.

برای مثال، اگر $b = 1$ ثابت باشد و $a$ تغییر کند، بردار $a\vv + \vw$ روی خطی موازی با $\vv$ حرکت می‌کند.
\end{intuition}

\begin{example}
فرض کنید $\vv = \twovec{1}{2}$ و $\vw = \twovec{3}{-1}$. چند ترکیب خطی:
\begin{align*}
2\vv + 1\vw &= 2\twovec{1}{2} + 1\twovec{3}{-1} = \twovec{2}{4} + \twovec{3}{-1} = \twovec{5}{3} \\
-1\vv + 3\vw &= -\twovec{1}{2} + 3\twovec{3}{-1} = \twovec{-1}{-2} + \twovec{9}{-3} = \twovec{8}{-5} \\
0\vv + 0\vw &= \twovec{0}{0} = \vzero
\end{align*}
\end{example}

% ============================================
\section{فضای پوشش \lr{(Span)}}
% ============================================

\begin{definition}[فضای پوشش]
\vocab{فضای پوشش} \lr{(Span)} مجموعه‌ای از بردارها، مجموعه تمام ترکیبات خطی ممکن آن بردارهاست:
\[
\spn\{\vv_1, \vv_2, \ldots, \vv_n\} = \{c_1\vv_1 + c_2\vv_2 + \cdots + c_n\vv_n \mid c_i \in \R\}
\]
\end{definition}

\begin{intuition}
سؤال کلیدی: با داشتن چند بردار و استفاده از دو عملیات اصلی (جمع و ضرب اسکالری)، به چه بردارهایی می‌توان رسید؟

پاسخ: \textbf{فضای پوشش} آن بردارها.
\end{intuition}

% -------------------------------------------
\subsection{فضای پوشش در دو بعد}
% -------------------------------------------

\begin{theorem}
برای دو بردار $\vv$ و $\vw$ در $\Rtwo$، سه حالت ممکن است:
\begin{enumerate}
    \item اگر $\vv$ و $\vw$ در یک راستا نباشند: $\spn\{\vv, \vw\} = \Rtwo$ (کل صفحه)
    \item اگر $\vv$ و $\vw$ در یک راستا باشند (ولی غیرصفر): $\spn\{\vv, \vw\}$ یک خط است
    \item اگر هر دو صفر باشند: $\spn\{\vv, \vw\} = \{\vzero\}$ (فقط مبدأ)
\end{enumerate}
\end{theorem}

\begin{center}
\begin{tikzpicture}[scale=0.9]
    % Case 1: Full plane
    \begin{scope}[shift={(0,0)}]
        \fill[blue!10] (-2,-2) rectangle (2,2);
        \draw[axis] (-2,0) -- (2,0);
        \draw[axis] (0,-2) -- (0,2);
        \draw[vector, red!70, very thick] (0,0) -- (1,0.5) node[right] {$\vv$};
        \draw[vector, green!60!black, very thick] (0,0) -- (-0.3,1) node[above] {$\vw$};
        \node at (0,-2.5) {حالت ۱: کل صفحه};
    \end{scope}

    % Case 2: Line
    \begin{scope}[shift={(6,0)}]
        \draw[blue!30, very thick] (-2,-1) -- (2,1);
        \draw[axis] (-2,0) -- (2,0);
        \draw[axis] (0,-2) -- (0,2);
        \draw[vector, red!70, very thick] (0,0) -- (1,0.5) node[right] {$\vv$};
        \draw[vector, green!60!black, very thick] (0,0) -- (1.6,0.8) node[above] {$\vw$};
        \node at (0,-2.5) {حالت ۲: یک خط};
    \end{scope}

    % Case 3: Origin
    \begin{scope}[shift={(12,0)}]
        \draw[axis] (-2,0) -- (2,0);
        \draw[axis] (0,-2) -- (0,2);
        \fill[blue!70] (0,0) circle (3pt);
        \node at (0,-2.5) {حالت ۳: فقط مبدأ};
    \end{scope}
\end{tikzpicture}
\end{center}

% -------------------------------------------
\subsection{فضای پوشش در سه بعد}
% -------------------------------------------

\begin{theorem}
برای بردارها در $\Rthree$:
\begin{itemize}
    \item \textbf{یک بردار غیرصفر:} فضای پوشش یک \textbf{خط} است
    \item \textbf{دو بردار غیر هم‌راستا:} فضای پوشش یک \textbf{صفحه} است
    \item \textbf{سه بردار که در یک صفحه نباشند:} فضای پوشش کل $\Rthree$ است
\end{itemize}
\end{theorem}

\begin{intuition}
دو بردار در فضای سه‌بعدی را تصور کنید. ترکیبات خطی آنها یک صفحه تخت از مبدأ می‌سازد. حالا اگر بردار سومی اضافه کنید که روی این صفحه نباشد، مثل این است که صفحه را در فضا «جارو» می‌کنید و کل فضا را پوشش می‌دهید.
\end{intuition}

% ============================================
\section{استقلال و وابستگی خطی}
% ============================================

\begin{definition}[وابستگی خطی]
مجموعه‌ای از بردارها \vocab{وابسته خطی} است اگر بتوان یکی از آنها را حذف کرد بدون اینکه فضای پوشش تغییر کند. به عبارت دیگر، حداقل یکی از بردارها «اضافی» است.

به بیان ریاضی: بردارهای $\vv_1, \ldots, \vv_n$ وابسته خطی هستند اگر و تنها اگر اسکالرهای غیرهمه‌صفر $c_1, \ldots, c_n$ وجود داشته باشند که:
\[
c_1\vv_1 + c_2\vv_2 + \cdots + c_n\vv_n = \vzero
\]
\end{definition}

\begin{definition}[استقلال خطی]
مجموعه‌ای از بردارها \vocab{مستقل خطی} است اگر هیچ‌کدام از آنها اضافی نباشد - یعنی هر بردار بعدی جدیدی به فضای پوشش اضافه می‌کند.

به بیان ریاضی: بردارهای $\vv_1, \ldots, \vv_n$ مستقل خطی هستند اگر و تنها اگر:
\[
c_1\vv_1 + c_2\vv_2 + \cdots + c_n\vv_n = \vzero \implies c_1 = c_2 = \cdots = c_n = 0
\]
\end{definition}

\begin{example}
بردارهای $\vv = \twovec{2}{1}$ و $\vw = \twovec{4}{2}$ را در نظر بگیرید.

این دو بردار \textbf{وابسته خطی} هستند زیرا $\vw = 2\vv$. می‌توان نوشت:
\[
2\vv - \vw = \vzero \quad \text{یعنی} \quad 2\twovec{2}{1} - \twovec{4}{2} = \twovec{0}{0}
\]
\end{example}

\begin{example}
بردارهای $\vv = \twovec{1}{0}$ و $\vw = \twovec{0}{1}$ مستقل خطی هستند.

اگر $a\vv + b\vw = \vzero$:
\[
a\twovec{1}{0} + b\twovec{0}{1} = \twovec{a}{b} = \twovec{0}{0}
\]
پس $a = 0$ و $b = 0$. تنها راه رسیدن به بردار صفر، صفر بودن همه ضرایب است.
\end{example}

\begin{warning}
در فضای $n$-بعدی، حداکثر $n$ بردار می‌توانند مستقل خطی باشند. اگر بیش از $n$ بردار داشته باشید، حتماً وابسته خطی هستند.
\end{warning}

% ============================================
\section{پایه \lr{(Basis)}}
% ============================================

\begin{definition}[پایه]
\vocab{پایه} یک فضای برداری، مجموعه‌ای از بردارها است که:
\begin{enumerate}
    \item \textbf{مستقل خطی} باشند
    \item کل فضا را \textbf{پوشش دهند} (span کنند)
\end{enumerate}
\end{definition}

\begin{theorem}
پایه استاندارد $\Rtwo$ عبارت است از:
\[
\left\{ \vi = \twovec{1}{0}, \; \vj = \twovec{0}{1} \right\}
\]
و پایه استاندارد $\Rthree$:
\[
\left\{ \vi = \threevec{1}{0}{0}, \; \vj = \threevec{0}{1}{0}, \; \vk = \threevec{0}{0}{1} \right\}
\]
\end{theorem}

\begin{intuition}
پایه مثل یک «زبان» برای توصیف بردارهاست. وقتی می‌گوییم $\twovec{3}{2}$، در واقع داریم می‌گوییم «۳ تا از اولین بردار پایه و ۲ تا از دومی».

اگر پایه متفاوتی انتخاب کنیم، همان بردار مختصات متفاوتی خواهد داشت - مثل ترجمه یک جمله به زبان دیگر.
\end{intuition}

% -------------------------------------------
\subsection{پایه‌های غیراستاندارد}
% -------------------------------------------

\begin{example}
مجموعه زیر نیز یک پایه برای $\Rtwo$ است:
\[
\mathcal{B} = \left\{ \vb_1 = \twovec{1}{1}, \; \vb_2 = \twovec{1}{-1} \right\}
\]

\textbf{اثبات استقلال خطی:} اگر $a\vb_1 + b\vb_2 = \vzero$:
\[
a\twovec{1}{1} + b\twovec{1}{-1} = \twovec{a+b}{a-b} = \twovec{0}{0}
\]
پس $a + b = 0$ و $a - b = 0$، که نتیجه می‌دهد $a = b = 0$.

\textbf{پوشش کل صفحه:} چون دو بردار مستقل خطی داریم و در $\Rtwo$ هستیم، کل صفحه پوشش داده می‌شود.
\end{example}

\begin{center}
\begin{tikzpicture}[scale=1.5]
    % Grid
    \draw[grid] (-2,-2) grid (2,2);

    % Axes
    \draw[axis] (-2,0) -- (2,0) node[right] {$x$};
    \draw[axis] (0,-2) -- (0,2) node[above] {$y$};

    % Standard basis
    \draw[basis, -{Stealth[length=3mm]}] (0,0) -- (1,0) node[below right] {$\vi$};
    \draw[basis, -{Stealth[length=3mm]}] (0,0) -- (0,1) node[above left] {$\vj$};

    % Non-standard basis
    \draw[vector, blue!70, very thick] (0,0) -- (1,1) node[above right] {$\vb_1$};
    \draw[vector, green!60!black, very thick] (0,0) -- (1,-1) node[below right] {$\vb_2$};
\end{tikzpicture}
\end{center}

% ============================================
\section{بردارها به عنوان نقاط}
% ============================================

\begin{remark}
گاهی به جای فکر کردن به بردار به عنوان پیکان، راحت‌تر است آن را به عنوان \textbf{نقطه} در نظر بگیریم - نقطه‌ای که نوک بردار در آن قرار دارد.

\begin{itemize}
    \item وقتی به یک بردار خاص فکر می‌کنید: آن را \textbf{پیکان} تصور کنید
    \item وقتی به مجموعه‌ای از بردارها فکر می‌کنید: آنها را \textbf{نقاط} تصور کنید
\end{itemize}
\end{remark}

\begin{center}
\begin{tikzpicture}[scale=1]
    % Left: Vectors as arrows
    \begin{scope}[shift={(0,0)}]
        \draw[axis] (-0.5,0) -- (3,0);
        \draw[axis] (0,-0.5) -- (0,3);
        \draw[vector, blue!70] (0,0) -- (1,2);
        \draw[vector, red!70] (0,0) -- (2,1);
        \draw[vector, green!60!black] (0,0) -- (2.5,2.5);
        \node at (1.5,-1) {بردار به عنوان پیکان};
    \end{scope}

    % Right: Vectors as points
    \begin{scope}[shift={(6,0)}]
        \draw[axis] (-0.5,0) -- (3,0);
        \draw[axis] (0,-0.5) -- (0,3);
        \fill[blue!70] (1,2) circle (3pt);
        \fill[red!70] (2,1) circle (3pt);
        \fill[green!60!black] (2.5,2.5) circle (3pt);
        \node at (1.5,-1) {بردار به عنوان نقطه};
    \end{scope}
\end{tikzpicture}
\end{center}

% ============================================
\section{خلاصه مفاهیم کلیدی}
% ============================================

\begin{summary}
\begin{description}
    \item[ترکیب خطی:] $c_1\vv_1 + c_2\vv_2 + \cdots + c_n\vv_n$ با اسکالرهای دلخواه
    \item[فضای پوشش:] مجموعه همه ترکیبات خطی ممکن
    \item[استقلال خطی:] هیچ برداری اضافی نیست
    \item[وابستگی خطی:] حداقل یک بردار می‌تواند حذف شود
    \item[پایه:] مجموعه مستقل خطی که کل فضا را پوشش می‌دهد
\end{description}
\end{summary}

% ============================================
\section{تمرین‌ها}
% ============================================

\begin{exercise}
بردار $\vv = \twovec{4}{6}$ را به صورت ترکیب خطی از $\vi$ و $\vj$ بنویسید.
\end{exercise}

\begin{exercise}
آیا بردارهای زیر مستقل خطی هستند؟ توضیح دهید.
\[
\va = \twovec{1}{3}, \quad \vb = \twovec{2}{6}
\]
\end{exercise}

\begin{exercise}
آیا بردارهای زیر مستقل خطی هستند؟
\[
\vv_1 = \twovec{1}{0}, \quad \vv_2 = \twovec{1}{1}, \quad \vv_3 = \twovec{0}{1}
\]
\textit{راهنمایی: در $\Rtwo$ حداکثر چند بردار می‌توانند مستقل خطی باشند؟}
\end{exercise}

\begin{exercise}
فضای پوشش بردارهای زیر را توصیف کنید:
\[
\vv = \threevec{1}{0}{0}, \quad \vw = \threevec{0}{1}{0}
\]
\end{exercise}

\begin{exercise}[چالشی]
نشان دهید که مجموعه زیر یک پایه برای $\Rtwo$ است:
\[
\left\{ \twovec{2}{1}, \twovec{1}{3} \right\}
\]
سپس مختصات بردار $\twovec{5}{7}$ را در این پایه جدید پیدا کنید.
\end{exercise}

\begin{exercise}
سه بردار در $\Rthree$ داده شده است:
\[
\vv_1 = \threevec{1}{0}{1}, \quad \vv_2 = \threevec{0}{1}{1}, \quad \vv_3 = \threevec{1}{1}{2}
\]
آیا این بردارها مستقل خطی هستند؟ فضای پوشش آنها چیست؟
\end{exercise}

\begin{problem}
ثابت کنید که اگر $\{\vv_1, \vv_2\}$ پایه‌ای برای $\Rtwo$ باشد، آنگاه $\{2\vv_1, 3\vv_2\}$ نیز یک پایه است.
\end{problem}

\begin{problem}
آیا می‌توان سه بردار در $\Rtwo$ یافت که مستقل خطی باشند؟ چرا؟
\end{problem}


% Chapter 3: Linear Transformations and Matrices
% lecture03.tex - تبدیلات خطی و ماتریس‌ها
% Chapter 3: Linear Transformations and Matrices

\chapter{تبدیلات خطی و ماتریس‌ها}
\label{ch:transformations}

\begin{abstract}
تبدیلات خطی قلب جبر خطی هستند. در این درس می‌آموزیم که چگونه هر تبدیل خطی را می‌توان با یک ماتریس نمایش داد، و ضرب ماتریسی چگونه با ترکیب تبدیلات مرتبط است.
\end{abstract}

% ============================================
\section{تبدیل چیست؟}
% ============================================

\begin{definition}[تبدیل]
\vocab{تبدیل} \lr{(Transformation)} تابعی است که بردارها را به بردارهای دیگر می‌برد:
\[
T: \Rtwo \to \Rtwo
\]
یعنی هر بردار ورودی را به یک بردار خروجی نگاشت می‌کند.
\end{definition}

\begin{intuition}
به جای «تابع» از واژه «تبدیل» استفاده می‌کنیم چون می‌خواهیم به \textbf{حرکت} فکر کنیم. تصور کنید هر بردار در فضا به جایی جدید «منتقل» می‌شود.

شبکه خطوط مختصات را تصور کنید. یک تبدیل این شبکه را می‌کشد، می‌فشارد، می‌چرخاند، یا به شکل دیگری تغییر می‌دهد.
\end{intuition}

% ============================================
\section{تبدیل خطی}
% ============================================

\begin{definition}[تبدیل خطی]
تبدیل $T$ یک \vocab{تبدیل خطی} است اگر و تنها اگر دو شرط زیر برقرار باشد:
\begin{enumerate}
    \item \textbf{جمع‌پذیری:} $T(\vv + \vw) = T(\vv) + T(\vw)$
    \item \textbf{همگنی:} $T(c\vv) = c \cdot T(\vv)$
\end{enumerate}
برای هر بردارهای $\vv, \vw$ و هر اسکالر $c$.
\end{definition}

\begin{intuition}
یک تبدیل خطی را می‌توان با دو ویژگی هندسی شناخت:
\begin{enumerate}
    \item \textbf{خطوط راست، راست می‌مانند} (خم نمی‌شوند)
    \item \textbf{مبدأ در جای خود می‌ماند}
\end{enumerate}

اگر خطوط شبکه بعد از تبدیل همچنان موازی و با فاصله یکنواخت باقی بمانند، تبدیل خطی است.
\end{intuition}

\begin{center}
\begin{tikzpicture}[scale=0.8]
    % Original grid
    \begin{scope}[shift={(0,0)}]
        \draw[grid] (-2,-2) grid (2,2);
        \draw[axis] (-2,0) -- (2,0);
        \draw[axis] (0,-2) -- (0,2);
        \draw[basis, -{Stealth[length=3mm]}] (0,0) -- (1,0) node[below] {$\vi$};
        \draw[basis, -{Stealth[length=3mm]}] (0,0) -- (0,1) node[left] {$\vj$};
        \node at (0,-3) {قبل از تبدیل};
    \end{scope}

    % Arrow
    \draw[-{Stealth[length=5mm]}, very thick] (3,0) -- (5,0) node[midway, above] {$T$};

    % Transformed grid (shear)
    \begin{scope}[shift={(8,0)}]
        \draw[blue!30, thin] (-2,-2) -- (0,2) -- (4,2) -- (2,-2) -- cycle;
        \foreach \y in {-1,0,1} {
            \draw[blue!30, thin] ({-2+\y},-2) -- ({\y+2},2);
        }
        \foreach \x in {-2,-1,0,1,2} {
            \draw[blue!30, thin] ({\x-1},-2) -- ({\x+1},2);
        }
        \draw[axis] (-2,0) -- (4,0);
        \draw[axis] (0,-2) -- (2,2);
        \draw[transformed, -{Stealth[length=3mm]}, very thick] (0,0) -- (1,0) node[below] {$T(\vi)$};
        \draw[transformed, -{Stealth[length=3mm]}, very thick] (0,0) -- (1,1) node[above left] {$T(\vj)$};
        \node at (1,-3) {بعد از تبدیل};
    \end{scope}
\end{tikzpicture}
\end{center}

% ============================================
\section{ماتریس یک تبدیل خطی}
% ============================================

\begin{theorem}[قضیه اساسی]
هر تبدیل خطی $T: \Rtwo \to \Rtwo$ کاملاً با دانستن اینکه $T$ بردارهای پایه $\vi$ و $\vj$ را به کجا می‌برد، مشخص می‌شود.
\end{theorem}

\begin{proof}
هر بردار $\vv = \twovec{x}{y}$ را می‌توان نوشت: $\vv = x\vi + y\vj$

با استفاده از خطی بودن:
\[
T(\vv) = T(x\vi + y\vj) = xT(\vi) + yT(\vj)
\]
پس اگر $T(\vi)$ و $T(\vj)$ را بدانیم، می‌توانیم $T(\vv)$ را برای هر $\vv$ محاسبه کنیم.
\end{proof}

\begin{definition}[ماتریس تبدیل]
اگر $T(\vi) = \twovec{a}{c}$ و $T(\vj) = \twovec{b}{d}$ باشد، \vocab{ماتریس تبدیل} $T$ برابر است با:
\[
\mA = \twomat{a}{b}{c}{d}
\]
ستون اول محل فرود $\vi$ و ستون دوم محل فرود $\vj$ است.
\end{definition}

\begin{important}
\textbf{قانون طلایی:} ستون‌های ماتریس = محل فرود بردارهای پایه
\end{important}

% ============================================
\section{ضرب ماتریس در بردار}
% ============================================

\begin{definition}[ضرب ماتریس در بردار]
\[
\twomat{a}{b}{c}{d} \twovec{x}{y} = x\twovec{a}{c} + y\twovec{b}{d} = \twovec{ax + by}{cx + dy}
\]
\end{definition}

\begin{intuition}
ضرب ماتریس در بردار یعنی:
\begin{enumerate}
    \item مؤلفه $x$ بردار ورودی را در ستون اول ضرب کن
    \item مؤلفه $y$ بردار ورودی را در ستون دوم ضرب کن
    \item نتایج را جمع کن
\end{enumerate}
این دقیقاً همان ترکیب خطی بردارهای پایه جدید است!
\end{intuition}

\begin{example}
تبدیل چرخش $90°$ پادساعتگرد:
\[
T(\vi) = \twovec{0}{1}, \quad T(\vj) = \twovec{-1}{0}
\]
پس ماتریس چرخش:
\[
\mR = \twomat{0}{-1}{1}{0}
\]
بررسی: $\mR\twovec{1}{0} = \twovec{0}{1}$ ✓
\end{example}

% ============================================
\section{مثال‌های تبدیلات مهم}
% ============================================

\subsection{چرخش \lr{(Rotation)}}

\begin{definition}[ماتریس چرخش]
چرخش به اندازه زاویه $\theta$ پادساعتگرد:
\[
\mR_\theta = \twomat{\cos\theta}{-\sin\theta}{\sin\theta}{\cos\theta}
\]
\end{definition}

\begin{example}
چرخش $45°$:
\[
\mR_{45°} = \twomat{\frac{\sqrt{2}}{2}}{-\frac{\sqrt{2}}{2}}{\frac{\sqrt{2}}{2}}{\frac{\sqrt{2}}{2}}
\]
\end{example}

\subsection{مقیاس‌گذاری \lr{(Scaling)}}

\begin{definition}[ماتریس مقیاس]
مقیاس‌گذاری با ضرایب $s_x$ در راستای $x$ و $s_y$ در راستای $y$:
\[
\mS = \twomat{s_x}{0}{0}{s_y}
\]
\end{definition}

\begin{practical}
\textbf{کاربرد در گرافیک کامپیوتری:} برای بزرگ یا کوچک کردن تصویر:
\begin{itemize}
    \item $s_x = s_y = 2$: تصویر دو برابر بزرگ می‌شود
    \item $s_x = 1, s_y = 0.5$: تصویر در راستای عمودی فشرده می‌شود
\end{itemize}
\end{practical}

\subsection{برش \lr{(Shear)}}

\begin{definition}[ماتریس برش افقی]
\[
\mH = \twomat{1}{k}{0}{1}
\]
که در آن $k$ مقدار برش است.
\end{definition}

\begin{center}
\begin{tikzpicture}[scale=1]
    % Original square
    \begin{scope}[shift={(0,0)}]
        \draw[fill=blue!20] (0,0) -- (1,0) -- (1,1) -- (0,1) -- cycle;
        \draw[basis] (0,0) -- (1,0) node[midway, below] {$\vi$};
        \draw[basis] (0,0) -- (0,1) node[midway, left] {$\vj$};
        \node at (0.5,-0.7) {اصلی};
    \end{scope}

    % Sheared
    \begin{scope}[shift={(4,0)}]
        \draw[fill=blue!20] (0,0) -- (1,0) -- (2,1) -- (1,1) -- cycle;
        \draw[transformed, very thick] (0,0) -- (1,0) node[midway, below] {$T(\vi)$};
        \draw[transformed, very thick] (0,0) -- (1,1) node[midway, left] {$T(\vj)$};
        \node at (1,-0.7) {برش با $k=1$};
    \end{scope}
\end{tikzpicture}
\end{center}

\subsection{انعکاس \lr{(Reflection)}}

\begin{example}
انعکاس نسبت به محور $x$:
\[
\twomat{1}{0}{0}{-1}
\]

انعکاس نسبت به خط $y = x$:
\[
\twomat{0}{1}{1}{0}
\]
\end{example}

% ============================================
\section{ضرب ماتریس‌ها = ترکیب تبدیلات}
% ============================================

\begin{theorem}[ترکیب تبدیلات]
اگر $\mA$ ماتریس تبدیل $T_1$ و $\mB$ ماتریس تبدیل $T_2$ باشد، آنگاه:
\[
\mB \cdot \mA = \text{ماتریس تبدیل } (T_2 \circ T_1)
\]
یعنی اول $T_1$ و سپس $T_2$ اعمال شود.
\end{theorem}

\begin{warning}
ترتیب مهم است! $\mA\mB \neq \mB\mA$ در حالت کلی.

اول چرخش و بعد برش $\neq$ اول برش و بعد چرخش
\end{warning}

\begin{definition}[ضرب دو ماتریس]
\[
\twomat{a}{b}{c}{d} \twomat{e}{f}{g}{h} = \twomat{ae+bg}{af+bh}{ce+dg}{cf+dh}
\]
\end{definition}

\begin{intuition}
برای محاسبه ستون‌های $\mB\mA$:
\begin{itemize}
    \item ستون اول: $\mB$ ضرب در ستون اول $\mA$ = محل فرود $\vi$ پس از هر دو تبدیل
    \item ستون دوم: $\mB$ ضرب در ستون دوم $\mA$ = محل فرود $\vj$ پس از هر دو تبدیل
\end{itemize}
\end{intuition}

\begin{example}
چرخش $90°$ و سپس برش:
\[
\underbrace{\twomat{1}{1}{0}{1}}_{\text{برش}} \underbrace{\twomat{0}{-1}{1}{0}}_{\text{چرخش}} = \twomat{1}{-1}{1}{0}
\]

بررسی: $\vi \to \twovec{0}{1} \to \twovec{1}{1}$ ✓
\end{example}

% ============================================
\section{کاربردهای عملی}
% ============================================

\begin{practical}
\textbf{گرافیک کامپیوتری و بازی‌های ویدیویی}

هر شیء در بازی با مجموعه‌ای از بردارها (رأس‌ها) توصیف می‌شود. برای حرکت، چرخش، یا تغییر اندازه شیء، کافی است ماتریس مناسب را در همه رأس‌ها ضرب کنیم.

یک انیمیشن = دنباله‌ای از ضرب ماتریس‌ها
\end{practical}

\begin{practical}
\textbf{پردازش تصویر}

فیلترهای تصویر مثل تار کردن، تیز کردن، و تشخیص لبه با ضرب ماتریسی پیاده‌سازی می‌شوند.
\end{practical}

% ============================================
\section{تمرین‌ها}
% ============================================

\begin{exercise}
ماتریس تبدیلی که $\vi$ را به $\twovec{2}{1}$ و $\vj$ را به $\twovec{-1}{3}$ می‌برد، بنویسید.
\end{exercise}

\begin{exercise}
حاصل ضرب زیر را محاسبه کنید:
\[
\twomat{1}{2}{3}{4} \twovec{5}{6}
\]
\end{exercise}

\begin{exercise}
ماتریس چرخش $180°$ را بنویسید و نشان دهید که برابر است با $-\mI$.
\end{exercise}

\begin{exercise}
دو ماتریس زیر را در هم ضرب کنید (به هر دو ترتیب) و نشان دهید که $\mA\mB \neq \mB\mA$:
\[
\mA = \twomat{1}{2}{0}{1}, \quad \mB = \twomat{0}{1}{1}{0}
\]
\end{exercise}

\begin{exercise}[چالشی]
ماتریسی پیدا کنید که انعکاس نسبت به خط $y = 2x$ باشد.

\textit{راهنمایی: این خط با محور $x$ زاویه $\arctan(2)$ می‌سازد.}
\end{exercise}

\begin{problem}
نشان دهید که ترکیب دو چرخش با زاویه‌های $\alpha$ و $\beta$ برابر است با چرخش به زاویه $\alpha + \beta$.
\end{problem}

\begin{problem}
اگر $\mA^2 = \mA$ (یعنی $\mA$ یک ماتریس تصویر باشد)، چه تفسیر هندسی‌ای دارد؟
\end{problem}


% Chapter 4: 3D Transformations and Determinants
% lecture04.tex - تبدیلات در سه بعد و دترمینان
% Chapter 4: 3D Transformations and Determinants

\chapter{تبدیلات سه‌بعدی و دترمینان}
\label{ch:determinant}

\begin{abstract}
دترمینان عددی است که میزان «کشیدگی» یا «فشردگی» فضا توسط یک تبدیل خطی را اندازه می‌گیرد. در این درس معنای هندسی دترمینان و نحوه محاسبه آن را می‌آموزیم.
\end{abstract}

% ============================================
\section{تبدیلات در فضای سه‌بعدی}
% ============================================

\begin{definition}[ماتریس تبدیل سه‌بعدی]
یک تبدیل خطی در $\Rthree$ با ماتریس $3 \times 3$ نمایش داده می‌شود:
\[
\mA = \threemat{a_{11}}{a_{12}}{a_{13}}{a_{21}}{a_{22}}{a_{23}}{a_{31}}{a_{32}}{a_{33}}
\]
ستون‌ها محل فرود $\vi$، $\vj$، و $\vk$ هستند.
\end{definition}

\begin{intuition}
همان منطق دوبعدی در سه بعد هم کار می‌کند. شبکه سه‌بعدی را تصور کنید که کشیده، فشرده، چرخیده، یا برش می‌خورد. خطوط همچنان راست می‌مانند و مبدأ ثابت است.
\end{intuition}

% ============================================
\section{دترمینان: معنای هندسی}
% ============================================

\begin{definition}[دترمینان - تعریف هندسی]
\vocab{دترمینان} یک ماتریس، ضریبی است که نشان می‌دهد تبدیل متناظر، مساحت (در دو بعد) یا حجم (در سه بعد) را چند برابر می‌کند.

اگر $\mA$ ماتریس تبدیل $T$ باشد:
\[
\det(\mA) = \frac{\text{مساحت/حجم بعد از تبدیل}}{\text{مساحت/حجم قبل از تبدیل}}
\]
\end{definition}

\begin{center}
\begin{tikzpicture}[scale=1.2]
    % Original unit square
    \begin{scope}[shift={(0,0)}]
        \fill[blue!20] (0,0) -- (1,0) -- (1,1) -- (0,1) -- cycle;
        \draw[thick] (0,0) -- (1,0) -- (1,1) -- (0,1) -- cycle;
        \draw[basis] (0,0) -- (1,0);
        \draw[basis] (0,0) -- (0,1);
        \node at (0.5,0.5) {$1$};
        \node at (0.5,-0.5) {مساحت = $1$};
    \end{scope}

    \draw[-{Stealth}, very thick] (1.5,0.5) -- (2.5,0.5) node[midway, above] {$T$};

    % Transformed parallelogram
    \begin{scope}[shift={(3,0)}]
        \fill[green!20] (0,0) -- (2,0) -- (3,1.5) -- (1,1.5) -- cycle;
        \draw[thick] (0,0) -- (2,0) -- (3,1.5) -- (1,1.5) -- cycle;
        \draw[transformed, very thick] (0,0) -- (2,0);
        \draw[transformed, very thick] (0,0) -- (1,1.5);
        \node at (1.5,0.75) {$3$};
        \node at (1.5,-0.5) {مساحت = $3$};
    \end{scope}
\end{tikzpicture}
\end{center}

\begin{important}
دترمینان = ضریب تغییر مساحت/حجم

اگر $\det(\mA) = 3$، هر شکل پس از تبدیل، $3$ برابر بزرگ‌تر می‌شود.
\end{important}

% ============================================
\section{علامت دترمینان}
% ============================================

\begin{theorem}[معنای علامت دترمینان]
\begin{itemize}
    \item $\det(\mA) > 0$: جهت‌گیری فضا حفظ می‌شود
    \item $\det(\mA) < 0$: جهت‌گیری فضا معکوس می‌شود (مثل آینه)
    \item $\det(\mA) = 0$: فضا به بعد پایین‌تر فشرده می‌شود
\end{itemize}
\end{theorem}

\begin{intuition}
در دو بعد، اگر $\vj$ نسبت به $\vi$ در سمت چپ باشد، جهت‌گیری «مثبت» است. اگر تبدیل این رابطه را عوض کند (مثلاً انعکاس)، دترمینان منفی می‌شود.

در سه بعد، قاعده دست راست: اگر انگشتان از $\vi$ به $\vj$ بچرخند و شست به $\vk$ اشاره کند، جهت‌گیری مثبت است.
\end{intuition}

% ============================================
\section{محاسبه دترمینان}
% ============================================

\subsection{دترمینان ماتریس $2 \times 2$}

\begin{definition}
\[
\det\twomat{a}{b}{c}{d} = ad - bc
\]
\end{definition}

\begin{example}
\[
\det\twomat{3}{1}{0}{2} = 3 \times 2 - 1 \times 0 = 6
\]
مساحت هر شکل ۶ برابر می‌شود.
\end{example}

\subsection{دترمینان ماتریس $3 \times 3$}

\begin{definition}[قاعده ساروس یا بسط]
\[
\det\threemat{a}{b}{c}{d}{e}{f}{g}{h}{i} = a(ei-fh) - b(di-fg) + c(dh-eg)
\]
\end{definition}

\begin{intuition}
دترمینان $3 \times 3$ برابر است با حجم متوازی‌السطوح ساخته شده از سه بردار ستونی ماتریس (با علامت).
\end{intuition}

\begin{example}
\[
\det\threemat{1}{2}{3}{4}{5}{6}{7}{8}{9}
\]
\begin{align*}
&= 1(5 \times 9 - 6 \times 8) - 2(4 \times 9 - 6 \times 7) + 3(4 \times 8 - 5 \times 7) \\
&= 1(45-48) - 2(36-42) + 3(32-35) \\
&= 1(-3) - 2(-6) + 3(-3) \\
&= -3 + 12 - 9 = 0
\end{align*}
دترمینان صفر یعنی سه ستون در یک صفحه هستند!
\end{example}

% ============================================
\section{خواص دترمینان}
% ============================================

\begin{theorem}[خواص اصلی دترمینان]
\begin{enumerate}
    \item $\det(\mI) = 1$
    \item $\det(\mA\mB) = \det(\mA) \cdot \det(\mB)$
    \item $\det(\mA\trans) = \det(\mA)$
    \item $\det(c\mA) = c^n \det(\mA)$ برای ماتریس $n \times n$
    \item اگر یک سطر/ستون صفر باشد: $\det(\mA) = 0$
    \item اگر دو سطر/ستون برابر باشند: $\det(\mA) = 0$
\end{enumerate}
\end{theorem}

\begin{intuition}
خاصیت $\det(\mA\mB) = \det(\mA) \cdot \det(\mB)$ بسیار مهم است:

اگر $\mA$ مساحت را ۳ برابر کند و $\mB$ مساحت را ۲ برابر کند، ترکیب آنها مساحت را $3 \times 2 = 6$ برابر می‌کند.
\end{intuition}

% ============================================
\section{دترمینان صفر: فشردگی فضا}
% ============================================

\begin{theorem}
$\det(\mA) = 0$ اگر و تنها اگر:
\begin{itemize}
    \item ستون‌های $\mA$ وابسته خطی باشند
    \item تبدیل متناظر، فضا را به بعد پایین‌تر ببرد
\end{itemize}
\end{theorem}

\begin{center}
\begin{tikzpicture}[scale=1]
    % 2D to 1D
    \begin{scope}[shift={(0,0)}]
        \fill[blue!20] (0,0) -- (1,0) -- (1,1) -- (0,1) -- cycle;
        \node at (0.5,-0.5) {$\Rtwo$};
    \end{scope}

    \draw[-{Stealth}, very thick] (1.5,0.5) -- (3,0.5) node[midway, above] {$\det = 0$};

    \begin{scope}[shift={(3.5,0)}]
        \draw[blue!70, ultra thick] (-0.5,0.5) -- (2,0.5);
        \node at (0.75,-0.5) {یک خط};
    \end{scope}
\end{tikzpicture}
\end{center}

\begin{practical}
\textbf{تشخیص وابستگی خطی:}

می‌خواهید بدانید آیا سه بردار در فضا مستقل خطی هستند؟ آنها را ستون‌های یک ماتریس قرار دهید. اگر $\det \neq 0$، مستقل هستند.
\end{practical}

% ============================================
\section{مثال‌های کاربردی}
% ============================================

\begin{practical}
\textbf{محاسبه مساحت مثلث با رأس‌های مشخص}

اگر رأس‌های مثلث $(x_1, y_1)$، $(x_2, y_2)$، $(x_3, y_3)$ باشند:
\[
\text{مساحت} = \frac{1}{2} \left| \det\threemat{x_1}{y_1}{1}{x_2}{y_2}{1}{x_3}{y_3}{1} \right|
\]
\end{practical}

\begin{practical}
\textbf{فیزیک: گشتاور و نیرو}

دترمینان در محاسبه ضرب خارجی بردارها استفاده می‌شود که در محاسبه گشتاور، تکانه زاویه‌ای، و میدان‌های الکترومغناطیسی کاربرد دارد.
\end{practical}

% ============================================
\section{تمرین‌ها}
% ============================================

\begin{exercise}
دترمینان ماتریس‌های زیر را محاسبه کنید:
\begin{enumerate}[label=(\alph*)]
    \item $\twomat{3}{7}{1}{4}$
    \item $\twomat{2}{6}{1}{3}$
    \item $\twomat{-1}{2}{3}{-6}$
\end{enumerate}
\end{exercise}

\begin{exercise}
اگر $\det(\mA) = 5$ و $\det(\mB) = -2$، مقدار $\det(\mA\mB)$ چیست؟
\end{exercise}

\begin{exercise}
مساحت متوازی‌الاضلاع با رأس‌های $(0,0)$، $(3,1)$، $(1,4)$، $(4,5)$ را محاسبه کنید.
\end{exercise}

\begin{exercise}
دترمینان ماتریس چرخش $\theta$ را محاسبه کنید و نتیجه را تفسیر کنید.
\end{exercise}

\begin{exercise}[چالشی]
ثابت کنید که برای هر ماتریس $\mA$:
\[
\det(\mA\inv) = \frac{1}{\det(\mA)}
\]
(فرض کنید $\mA$ معکوس‌پذیر است)
\end{exercise}

\begin{problem}
حجم متوازی‌السطوح ساخته شده از بردارهای زیر را محاسبه کنید:
\[
\va = \threevec{1}{0}{2}, \quad \vb = \threevec{0}{3}{1}, \quad \vc = \threevec{2}{1}{0}
\]
\end{problem}


% Chapter 5: Inverse Matrices, Column Space, and Null Space
% lecture05.tex - Inverse Matrices, Column Space, and Null Space
% Chapter 5: Inverse Matrices, Column Space, and Null Space

\chapter{Inverse Matrices, Column Space, and Null Space}
\label{ch:inverse}

\begin{abstract}
In this chapter, we learn about inverse matrices, solving systems of linear equations, and important spaces associated with matrices (column space, null space, and rank).
\end{abstract}

% ============================================
\section{Systems of Linear Equations}
% ============================================

\begin{definition}[System of Equations in Matrix Form]
A system of linear equations:
\[
\system{
a_{11}x_1 + a_{12}x_2 + \cdots + a_{1n}x_n &= b_1 \\
a_{21}x_1 + a_{22}x_2 + \cdots + a_{2n}x_n &= b_2 \\
&\vdots \\
a_{m1}x_1 + a_{m2}x_2 + \cdots + a_{mn}x_n &= b_m
}
\]
can be written as $\mA\vx = \vb$.
\end{definition}

\begin{intuition}
Interpret the equation $\mA\vx = \vb$ as:
\begin{center}
``What vector $\vx$ lands on $\vb$ after applying transformation $\mA$?''
\end{center}
In other words: ``reverse'' the transformation $\mA$ to get from $\vb$ to $\vx$.
\end{intuition}

% ============================================
\section{Inverse Matrix}
% ============================================

\begin{definition}[Inverse Matrix]
The matrix $\mA\inv$ is the \vocab{inverse} of matrix $\mA$ if:
\[
\mA\inv \mA = \mA \mA\inv = \mI
\]
where $\mI$ is the identity matrix.
\end{definition}

\begin{intuition}
If $\mA$ is a transformation, $\mA\inv$ is the transformation that undoes the effect of $\mA$:
\begin{itemize}
    \item If $\mA$ is rotation by $90°$, then $\mA\inv$ is rotation by $-90°$
    \item If $\mA$ scales by 2, then $\mA\inv$ scales by $\frac{1}{2}$
\end{itemize}
\end{intuition}

\begin{theorem}[Solving Systems with the Inverse]
If $\mA$ is invertible, the unique solution to $\mA\vx = \vb$ is:
\[
\vx = \mA\inv \vb
\]
\end{theorem}

\subsection{Formula for $2 \times 2$ Inverse}

\begin{theorem}
If $\mA = \twomat{a}{b}{c}{d}$ and $\det(\mA) \neq 0$:
\[
\mA\inv = \frac{1}{ad-bc} \twomat{d}{-b}{-c}{a}
\]
\end{theorem}

\begin{example}
\[
\mA = \twomat{3}{1}{2}{4}, \quad \det(\mA) = 12 - 2 = 10
\]
\[
\mA\inv = \frac{1}{10} \twomat{4}{-1}{-2}{3} = \twomat{0.4}{-0.1}{-0.2}{0.3}
\]
\end{example}

% ============================================
\section{When Does the Inverse Exist?}
% ============================================

\begin{theorem}
Matrix $\mA$ is invertible if and only if $\det(\mA) \neq 0$.
\end{theorem}

\begin{intuition}
If $\det(\mA) = 0$, the transformation $\mA$ collapses space (e.g., plane to line). Such a transformation is not reversible---information is lost.

It's like adding several numbers together: you can't recover the original numbers from just the sum.
\end{intuition}

% ============================================
\section{Column Space}
% ============================================

\begin{definition}[Column Space]
The \vocab{column space} of matrix $\mA$, denoted $\Col(\mA)$, is the span of the columns of $\mA$:
\[
\Col(\mA) = \spn\{\text{columns of } \mA\}
\]
\end{definition}

\begin{intuition}
Column space = set of all possible outputs of transformation $\mA$

Question: ``Does $\mA\vx = \vb$ have a solution?'' is equivalent to ``Is $\vb$ in the column space of $\mA$?''
\end{intuition}

\begin{example}
\[
\mA = \twomat{1}{3}{2}{6}
\]
The second column = 3 times the first column, so:
\[
\Col(\mA) = \spn\left\{\twovec{1}{2}\right\} = \text{a line through the origin}
\]
\end{example}

% ============================================
\section{Null Space}
% ============================================

\begin{definition}[Null Space]
The \vocab{null space} (or kernel) of matrix $\mA$ is the set of all vectors that $\mA$ sends to zero:
\[
\Null(\mA) = \{\vx \mid \mA\vx = \vzero\}
\]
\end{definition}

\begin{intuition}
Null space = vectors that transformation $\mA$ ``squishes'' to zero

If $\det(\mA) \neq 0$: only the zero vector gets squished, so $\Null(\mA) = \{\vzero\}$

If $\det(\mA) = 0$: an entire line or plane gets compressed to the origin
\end{intuition}

\begin{example}
For matrix $\mA = \twomat{1}{2}{2}{4}$:

Solve $\mA\vx = \vzero$:
\[
\twomat{1}{2}{2}{4}\twovec{x}{y} = \twovec{0}{0}
\]
Equation: $x + 2y = 0$, so $x = -2y$

Null space: $\Null(\mA) = \left\{ t\twovec{-2}{1} \mid t \in \R \right\}$ (a line)
\end{example}

% ============================================
\section{Rank}
% ============================================

\begin{definition}[Rank]
The \vocab{rank} of matrix $\mA$ equals:
\begin{itemize}
    \item The dimension of the column space
    \item The number of linearly independent columns
    \item The number of linearly independent rows
\end{itemize}
\[
\rank(\mA) = \dim(\Col(\mA))
\]
\end{definition}

\begin{theorem}[Rank-Nullity Theorem]
For an $m \times n$ matrix:
\[
\rank(\mA) + \dim(\Null(\mA)) = n
\]
(the number of columns)
\end{theorem}

\begin{intuition}
Rank = dimensions that the transformation preserves

$\dim(\Null(\mA))$ = dimensions that are lost

Their sum = dimension of the input space
\end{intuition}

% ============================================
\section{Non-Square Matrices}
% ============================================

\begin{definition}[Transformations Between Dimensions]
An $m \times n$ matrix represents a transformation from $\R^n$ to $\R^m$:
\begin{itemize}
    \item $m > n$: transformation from lower to higher dimension
    \item $m < n$: transformation from higher to lower dimension
\end{itemize}
\end{definition}

\begin{example}
A $2 \times 3$ matrix:
\[
\mA = \twomat{1}{0}{2}{0}{1}{-1}
\]
is a transformation from $\Rthree$ to $\Rtwo$. It ``projects'' 3D space onto a plane.
\end{example}

% ============================================
\section{Different Cases for Systems of Equations}
% ============================================

\begin{theorem}[Analysis of $\mA\vx = \vb$]
\begin{enumerate}
    \item \textbf{Unique solution:} If $\det(\mA) \neq 0$
    \item \textbf{Infinitely many solutions:} If $\det(\mA) = 0$ and $\vb \in \Col(\mA)$
    \item \textbf{No solution:} If $\vb \notin \Col(\mA)$
\end{enumerate}
\end{theorem}

\begin{center}
\begin{tikzpicture}[scale=0.9]
    % Case 1: Unique solution
    \begin{scope}[shift={(0,0)}]
        \fill[blue!10] (-1.5,-1.5) rectangle (1.5,1.5);
        \draw[axis] (-1.5,0) -- (1.5,0);
        \draw[axis] (0,-1.5) -- (0,1.5);
        \fill[red] (0.8,0.6) circle (2pt) node[above right] {$\vx$};
        \node at (0,-2) {Unique solution};
        \node at (0,-2.5) {\small $\det \neq 0$};
    \end{scope}

    % Case 2: Infinite solutions
    \begin{scope}[shift={(5,0)}]
        \draw[blue!30, very thick] (-1.5,-0.75) -- (1.5,0.75);
        \draw[axis] (-1.5,0) -- (1.5,0);
        \draw[axis] (0,-1.5) -- (0,1.5);
        \node at (0,-2) {Infinite solutions};
        \node at (0,-2.5) {\small $\det = 0$, $\vb \in \Col$};
    \end{scope}

    % Case 3: No solution
    \begin{scope}[shift={(10,0)}]
        \draw[blue!30, very thick] (-1.5,-0.75) -- (1.5,0.75);
        \draw[axis] (-1.5,0) -- (1.5,0);
        \draw[axis] (0,-1.5) -- (0,1.5);
        \fill[red] (0.5,1.2) circle (2pt) node[right] {$\vb$};
        \node[red] at (0.5,0.8) {$\times$};
        \node at (0,-2) {No solution};
        \node at (0,-2.5) {\small $\vb \notin \Col$};
    \end{scope}
\end{tikzpicture}
\end{center}

% ============================================
\section{Exercises}
% ============================================

\begin{exercise}
Compute the inverse of:
\[
\mA = \twomat{2}{5}{1}{3}
\]
\end{exercise}

\begin{exercise}
Solve the following system using the inverse matrix:
\[
\system{
2x + 3y &= 7 \\
x + 2y &= 4
}
\]
\end{exercise}

\begin{exercise}
Find the null space of:
\[
\mA = \twomat{1}{-2}{-3}{6}
\]
\end{exercise}

\begin{exercise}
Determine the rank of:
\[
\mA = \threemat{1}{2}{3}{2}{4}{6}{1}{1}{1}
\]
\end{exercise}

\begin{exercise}[Challenge]
Show that $(\mA\mB)\inv = \mB\inv\mA\inv$ (if the inverses exist).
\end{exercise}

\begin{problem}
For what values of $k$ is the following matrix not invertible?
\[
\mA = \twomat{k}{2}{3}{k}
\]
\end{problem}


% Chapter 6: Dot Products and Duality
% lecture06.tex - Dot Products and Duality
% Chapter 6: Dot Products and Duality

\chapter{Dot Products and Duality}
\label{ch:dotproduct}

\begin{abstract}
The dot product is one of the fundamental operations in linear algebra. In this chapter, we learn its definition, properties, and the deep concept of duality that connects vectors to linear functions.
\end{abstract}

% ============================================
\section{Definition of the Dot Product}
% ============================================

\begin{definition}[Dot Product]
The \vocab{dot product} of two vectors $\vv = \twovec{v_1}{v_2}$ and $\vw = \twovec{w_1}{w_2}$:
\[
\vv \cdot \vw = v_1 w_1 + v_2 w_2
\]
In $n$-dimensional space:
\[
\vv \cdot \vw = \sum_{i=1}^{n} v_i w_i
\]
\end{definition}

\begin{important}
The dot product of two vectors is a \textbf{number} (scalar), not a vector!
\end{important}

\begin{example}
\[
\twovec{3}{2} \cdot \twovec{1}{4} = 3 \times 1 + 2 \times 4 = 3 + 8 = 11
\]
\end{example}

% ============================================
\section{Geometric Interpretation}
% ============================================

\begin{theorem}[Geometric Formula for Dot Product]
\[
\vv \cdot \vw = \norm{\vv} \norm{\vw} \cos\theta
\]
where $\theta$ is the angle between the two vectors.
\end{theorem}

\begin{intuition}
The dot product can be interpreted two ways:
\begin{enumerate}
    \item Length of $\vv$ times the length of the projection of $\vw$ onto $\vv$
    \item Length of $\vw$ times the length of the projection of $\vv$ onto $\vw$
\end{enumerate}
\end{intuition}

\begin{center}
\begin{tikzpicture}[scale=1.5]
    % Vectors
    \draw[vector, blue!70, very thick] (0,0) -- (3,0) node[right] {$\vv$};
    \draw[vector, green!60!black, very thick] (0,0) -- (2,1.5) node[above] {$\vw$};

    % Projection
    \draw[dashed, red] (2,1.5) -- (2,0);
    \draw[red, thick] (0,0) -- (2,0) node[midway, below] {projection of $\vw$};

    % Angle
    \draw (0.7,0) arc (0:36.87:0.7) node[midway, right] {$\theta$};
\end{tikzpicture}
\end{center}

% ============================================
\section{Properties of the Dot Product}
% ============================================

\begin{theorem}[Main Properties]
\begin{enumerate}
    \item \textbf{Commutative:} $\vv \cdot \vw = \vw \cdot \vv$
    \item \textbf{Distributive:} $\vv \cdot (\vw + \vu) = \vv \cdot \vw + \vv \cdot \vu$
    \item \textbf{Scalar multiplication:} $(c\vv) \cdot \vw = c(\vv \cdot \vw)$
    \item \textbf{Positive definiteness:} $\vv \cdot \vv \geq 0$ and $\vv \cdot \vv = 0 \Leftrightarrow \vv = \vzero$
\end{enumerate}
\end{theorem}

\begin{definition}[Length of a Vector]
\[
\norm{\vv} = \sqrt{\vv \cdot \vv} = \sqrt{v_1^2 + v_2^2 + \cdots + v_n^2}
\]
\end{definition}

% ============================================
\section{Orthogonality}
% ============================================

\begin{theorem}[Condition for Orthogonality]
Two vectors $\vv$ and $\vw$ are \vocab{orthogonal} (perpendicular) if and only if:
\[
\vv \cdot \vw = 0
\]
\end{theorem}

\begin{intuition}
If $\vv \cdot \vw = 0$, then $\cos\theta = 0$, so $\theta = 90°$.

The projection of any vector onto a perpendicular vector is zero.
\end{intuition}

\begin{example}
Vectors $\vv = \twovec{3}{2}$ and $\vw = \twovec{2}{-3}$ are orthogonal because:
\[
\vv \cdot \vw = 3 \times 2 + 2 \times (-3) = 6 - 6 = 0
\]
\end{example}

% ============================================
\section{Vector Projection}
% ============================================

\begin{definition}[Projection of a Vector]
The projection of vector $\vv$ onto vector $\vw$:
\[
\proj_{\vw}(\vv) = \frac{\vv \cdot \vw}{\vw \cdot \vw} \vw = \frac{\vv \cdot \vw}{\norm{\vw}^2} \vw
\]
\end{definition}

\begin{intuition}
The projection of $\vv$ onto $\vw$ is a vector in the direction of $\vw$ that represents the ``shadow'' of $\vv$ on the line of $\vw$.
\end{intuition}

\begin{center}
\begin{tikzpicture}[scale=1.5]
    \draw[vector, blue!70, very thick] (0,0) -- (3,0) node[right] {$\vw$};
    \draw[vector, green!60!black, very thick] (0,0) -- (2,2) node[above] {$\vv$};
    \draw[vector, red!70, very thick] (0,0) -- (2,0) node[below] {$\proj_{\vw}(\vv)$};
    \draw[dashed, gray] (2,2) -- (2,0);
\end{tikzpicture}
\end{center}

% ============================================
\section{Duality}
% ============================================

\begin{intuition}
A deep insight: every $1 \times n$ row vector can be thought of as a \textbf{linear function} that takes $n$-dimensional vectors to numbers.

For example, $\bmat{2 & 1}$ is a linear function:
\[
\bmat{2 & 1} \twovec{x}{y} = 2x + y
\]
\end{intuition}

\begin{theorem}[Duality]
Every linear function $f: \R^n \to \R$ can be written as a dot product with a fixed vector:
\[
f(\vx) = \vv \cdot \vx
\]
for a unique vector $\vv$.
\end{theorem}

\begin{definition}[Dual Vector]
The vector $\vv$ that represents a linear function $f$ is called the \vocab{dual vector} of that function.
\end{definition}

\begin{practical}
\textbf{Application in Machine Learning:}

In neural networks, each neuron computes a linear function of its inputs:
\[
\text{output} = w_1 x_1 + w_2 x_2 + \cdots + w_n x_n = \vw \cdot \vx
\]
The weights $\vw$ are the dual vector of that neuron.
\end{practical}

% ============================================
\section{Practical Applications}
% ============================================

\begin{practical}
\textbf{Physics: Mechanical Work}

Work done by force $\vec{F}$ over displacement $\vec{d}$:
\[
W = \vec{F} \cdot \vec{d} = \norm{\vec{F}} \norm{\vec{d}} \cos\theta
\]
If the force is perpendicular to the direction of motion, no work is done!
\end{practical}

\begin{practical}
\textbf{Computer Graphics: Lighting Calculations}

The intensity of light reflected from a surface:
\[
I = \max(0, \vec{n} \cdot \vec{l})
\]
where $\vec{n}$ is the surface normal and $\vec{l}$ is the light direction.
\end{practical}

\begin{practical}
\textbf{Text Similarity}

To compare two text documents, convert each to a vector (e.g., TF-IDF) and compute cosine similarity:
\[
\text{similarity} = \frac{\vv \cdot \vw}{\norm{\vv}\norm{\vw}} = \cos\theta
\]
\end{practical}

% ============================================
\section{Exercises}
% ============================================

\begin{exercise}
Compute the dot product of:
\begin{enumerate}[label=(\alph*)]
    \item $\twovec{1}{2} \cdot \twovec{3}{4}$
    \item $\threevec{1}{-1}{2} \cdot \threevec{2}{3}{-1}$
\end{enumerate}
\end{exercise}

\begin{exercise}
Are vectors $\va = \twovec{4}{3}$ and $\vb = \twovec{-3}{4}$ orthogonal?
\end{exercise}

\begin{exercise}
Find the projection of $\vv = \twovec{3}{4}$ onto $\vw = \twovec{1}{0}$.
\end{exercise}

\begin{exercise}
Calculate the angle between vectors $\va = \twovec{1}{1}$ and $\vb = \twovec{1}{0}$.
\end{exercise}

\begin{exercise}[Challenge]
Show that for any two vectors $\vv$ and $\vw$:
\[
\norm{\vv + \vw}^2 = \norm{\vv}^2 + 2(\vv \cdot \vw) + \norm{\vw}^2
\]
\end{exercise}

\begin{problem}
A force of magnitude $10$ Newtons acts at an angle of $60°$ to the horizontal on an object. If the object moves $5$ meters horizontally, how much work is done?
\end{problem}


% Chapter 7: Cross Products and Applications
% lecture07.tex - Cross Products and Applications
% Chapter 7: Cross Products and Applications

\chapter{Cross Products and Applications}
\label{ch:crossproduct}

\begin{abstract}
The cross product (vector product) is an operation that creates a vector perpendicular to two vectors in three-dimensional space. In this chapter, we study its definition, properties, and deep connection to determinants.
\end{abstract}

% ============================================
\section{Cross Product in Two Dimensions}
% ============================================

\begin{definition}[2D Cross Product]
For two vectors in $\Rtwo$:
\[
\vv \times \vw = v_1 w_2 - v_2 w_1 = \det\twomat{v_1}{w_1}{v_2}{w_2}
\]
The result is a \textbf{number} (not a vector).
\end{definition}

\begin{intuition}
2D cross product = signed area of the parallelogram formed by the two vectors

Positive sign: $\vw$ is to the left of $\vv$\\
Negative sign: $\vw$ is to the right of $\vv$
\end{intuition}

\begin{center}
\begin{tikzpicture}[scale=1.2]
    % Parallelogram
    \fill[blue!20] (0,0) -- (2,0.5) -- (3,2) -- (1,1.5) -- cycle;
    \draw[vector, red!70, very thick] (0,0) -- (2,0.5) node[midway, below] {$\vv$};
    \draw[vector, green!60!black, very thick] (0,0) -- (1,1.5) node[midway, left] {$\vw$};
    \draw[dashed] (2,0.5) -- (3,2);
    \draw[dashed] (1,1.5) -- (3,2);
    \node at (1.5,1) {Area $= |\vv \times \vw|$};
\end{tikzpicture}
\end{center}

% ============================================
\section{Cross Product in Three Dimensions}
% ============================================

\begin{definition}[3D Cross Product]
For $\vv = \threevec{v_1}{v_2}{v_3}$ and $\vw = \threevec{w_1}{w_2}{w_3}$:
\[
\vv \times \vw = \threevec{v_2 w_3 - v_3 w_2}{v_3 w_1 - v_1 w_3}{v_1 w_2 - v_2 w_1}
\]
\end{definition}

\begin{theorem}[Determinant Formula]
\[
\vv \times \vw = \det\threemat{\vi}{\vj}{\vk}{v_1}{v_2}{v_3}{w_1}{w_2}{w_3}
\]
(symbolic expansion along the first row)
\end{theorem}

\begin{intuition}
The cross product $\vv \times \vw$:
\begin{itemize}
    \item \textbf{Direction:} Perpendicular to both vectors (right-hand rule)
    \item \textbf{Magnitude:} Area of the parallelogram formed by $\vv$ and $\vw$
\end{itemize}
\end{intuition}

\begin{center}
\begin{tikzpicture}[scale=1]
    % 3D coordinate hint
    \draw[axis] (0,0) -- (3,0) node[right] {$x$};
    \draw[axis] (0,0) -- (0,3) node[above] {$z$};
    \draw[axis] (0,0) -- (-1,-0.7) node[below left] {$y$};

    % Vectors
    \draw[vector, red!70, very thick] (0,0) -- (2,-0.5) node[right] {$\vv$};
    \draw[vector, green!60!black, very thick] (0,0) -- (-0.5,0) node[left] {$\vw$};
    \draw[vector, blue!70, ultra thick] (0,0) -- (0,2) node[above] {$\vv \times \vw$};

    % Parallelogram base
    \fill[gray!20] (0,0) -- (2,-0.5) -- (1.5,-0.5) -- (-0.5,0) -- cycle;
\end{tikzpicture}
\end{center}

% ============================================
\section{Properties of the Cross Product}
% ============================================

\begin{theorem}[Main Properties]
\begin{enumerate}
    \item \textbf{Anti-commutativity:} $\vv \times \vw = -(\vw \times \vv)$
    \item \textbf{Distributivity:} $\vv \times (\vw + \vu) = \vv \times \vw + \vv \times \vu$
    \item \textbf{Scalar multiplication:} $(c\vv) \times \vw = c(\vv \times \vw)$
    \item \textbf{Orthogonality:} $\vv \times \vw \perp \vv$ and $\vv \times \vw \perp \vw$
    \item \textbf{Self-cross is zero:} $\vv \times \vv = \vzero$
\end{enumerate}
\end{theorem}

\begin{warning}
The cross product is \textbf{not commutative}!
\[
\vv \times \vw \neq \vw \times \vv
\]
In fact, they are opposite to each other.
\end{warning}

\begin{theorem}[Magnitude Formula]
\[
\norm{\vv \times \vw} = \norm{\vv} \norm{\vw} \sin\theta
\]
where $\theta$ is the angle between the two vectors.
\end{theorem}

% ============================================
\section{Cross Product from the Linear Transformation Viewpoint}
% ============================================

\begin{intuition}
A deeper view: the cross product can be defined through duality.

The function $f(\vx) = \det[\vv \,|\, \vw \,|\, \vx]$ is a linear function of $\vx$. By duality, there must exist a vector $\vp$ such that:
\[
f(\vx) = \vp \cdot \vx
\]
This $\vp$ is exactly $\vv \times \vw$!
\end{intuition}

\begin{theorem}
\[
\det[\vv \,|\, \vw \,|\, \vx] = (\vv \times \vw) \cdot \vx
\]
\end{theorem}

% ============================================
\section{Basis Vector Products}
% ============================================

\begin{theorem}[Cross Products of Basis Vectors]
\begin{align*}
\vi \times \vj &= \vk & \vj \times \vi &= -\vk \\
\vj \times \vk &= \vi & \vk \times \vj &= -\vi \\
\vk \times \vi &= \vj & \vi \times \vk &= -\vj
\end{align*}
\end{theorem}

\begin{example}
\[
\threevec{2}{3}{4} \times \threevec{5}{6}{7}
\]
\begin{align*}
&= (3 \times 7 - 4 \times 6)\vi - (2 \times 7 - 4 \times 5)\vj + (2 \times 6 - 3 \times 5)\vk \\
&= (21-24)\vi - (14-20)\vj + (12-15)\vk \\
&= -3\vi + 6\vj - 3\vk = \threevec{-3}{6}{-3}
\end{align*}
\end{example}

% ============================================
\section{Applications}
% ============================================

\begin{practical}
\textbf{Physics: Torque}

Torque of force $\vec{F}$ about a point at distance $\vec{r}$:
\[
\vec{\tau} = \vec{r} \times \vec{F}
\]
The direction of torque is perpendicular to the plane containing the force and the lever arm.
\end{practical}

\begin{practical}
\textbf{Physics: Lorentz Force}

Force on a charged particle moving in a magnetic field:
\[
\vec{F} = q\vec{v} \times \vec{B}
\]
\end{practical}

\begin{practical}
\textbf{Computer Graphics: Surface Normal}

To find the normal vector to a triangular surface with vertices $A$, $B$, $C$:
\[
\vec{n} = (\vec{B} - \vec{A}) \times (\vec{C} - \vec{A})
\]
\end{practical}

\begin{practical}
\textbf{Computing Triangle Area}

Area of triangle with vertices $A$, $B$, $C$:
\[
\text{Area} = \frac{1}{2}\norm{(\vec{B}-\vec{A}) \times (\vec{C}-\vec{A})}
\]
\end{practical}

% ============================================
\section{Triple Products}
% ============================================

\begin{definition}[Scalar Triple Product]
\[
\va \cdot (\vb \times \vc) = \det[\va \,|\, \vb \,|\, \vc]
\]
Result = signed volume of the parallelepiped formed by the three vectors
\end{definition}

\begin{theorem}[Cyclic Property]
\[
\va \cdot (\vb \times \vc) = \vb \cdot (\vc \times \va) = \vc \cdot (\va \times \vb)
\]
\end{theorem}

% ============================================
\section{Exercises}
% ============================================

\begin{exercise}
Compute the cross product:
\[
\va = \threevec{1}{2}{3}, \quad \vb = \threevec{4}{5}{6}
\]
\end{exercise}

\begin{exercise}
Find the area of the parallelogram with sides $\va = \threevec{1}{0}{0}$ and $\vb = \threevec{1}{1}{0}$.
\end{exercise}

\begin{exercise}
Show that $\va \times \vb$ is perpendicular to both $\va$ and $\vb$.
\end{exercise}

\begin{exercise}
Find the normal vector to the plane passing through points $(1,0,0)$, $(0,1,0)$, $(0,0,1)$.
\end{exercise}

\begin{exercise}[Challenge]
Show that:
\[
\norm{\va \times \vb}^2 + (\va \cdot \vb)^2 = \norm{\va}^2 \norm{\vb}^2
\]
(This is called Lagrange's identity)
\end{exercise}

\begin{problem}
Calculate the volume of the parallelepiped with edges:
\[
\va = \threevec{1}{1}{0}, \quad \vb = \threevec{0}{1}{1}, \quad \vc = \threevec{1}{0}{1}
\]
\end{problem}


% Chapter 8: Cramer's Rule
% lecture08.tex - قاعده کرامر
% Chapter 8: Cramer's Rule

\chapter{قاعده کرامر}
\label{ch:cramer}

\begin{abstract}
قاعده کرامر روشی زیبا برای حل دستگاه معادلات خطی با استفاده از دترمینان است. در این درس با تفسیر هندسی این قاعده و شرایط کاربرد آن آشنا می‌شویم.
\end{abstract}

% ============================================
\section{بیان قاعده کرامر}
% ============================================

\begin{theorem}[قاعده کرامر]
برای دستگاه $\mA\vx = \vb$ با ماتریس $n \times n$ و $\det(\mA) \neq 0$:
\[
x_i = \frac{\det(\mA_i)}{\det(\mA)}
\]
که $\mA_i$ ماتریسی است که ستون $i$ام $\mA$ با بردار $\vb$ جایگزین شده.
\end{theorem}

\begin{example}
دستگاه $2 \times 2$:
\[
\system{
ax + by &= e \\
cx + dy &= f
}
\]
جواب:
\[
x = \frac{\det\twomat{e}{b}{f}{d}}{\det\twomat{a}{b}{c}{d}} = \frac{ed - bf}{ad - bc}
\]
\[
y = \frac{\det\twomat{a}{e}{c}{f}}{\det\twomat{a}{b}{c}{d}} = \frac{af - ec}{ad - bc}
\]
\end{example}

% ============================================
\section{تفسیر هندسی}
% ============================================

\begin{intuition}
در دو بعد، معادله $\mA\vx = \vb$ می‌پرسد:

«چه ترکیب خطی از ستون‌های $\mA$ برابر $\vb$ است؟»

اگر $\mA = [\va_1 \,|\, \va_2]$ و $\vx = \twovec{x}{y}$:
\[
x\va_1 + y\va_2 = \vb
\]
\end{intuition}

\begin{center}
\begin{tikzpicture}[scale=1.3]
    % Parallelogram
    \draw[vector, red!70, very thick] (0,0) -- (2,0.5) node[right] {$\va_1$};
    \draw[vector, blue!70, very thick] (0,0) -- (0.5,2) node[above] {$\va_2$};
    \draw[vector, green!60!black, ultra thick] (0,0) -- (1.5,1.5) node[above right] {$\vb$};

    % Parallelogram formed by b and a2
    \fill[orange!20] (0,0) -- (1.5,1.5) -- (2,3.5) -- (0.5,2) -- cycle;

    % Parallelogram formed by a1 and a2
    \fill[gray!20] (0,0) -- (2,0.5) -- (2.5,2.5) -- (0.5,2) -- cycle;

    \node at (1.2,2.5) {\small مساحت نارنجی};
    \node at (1.8,1) {\small مساحت خاکستری};
\end{tikzpicture}
\end{center}

\begin{theorem}[تفسیر مساحتی]
\[
x = \frac{\text{مساحت متوازی‌الاضلاع}(\vb, \va_2)}{\text{مساحت متوازی‌الاضلاع}(\va_1, \va_2)}
\]
\[
y = \frac{\text{مساحت متوازی‌الاضلاع}(\va_1, \vb)}{\text{مساحت متوازی‌الاضلاع}(\va_1, \va_2)}
\]
\end{theorem}

\begin{intuition}
چرا این کار می‌کند؟

متوازی‌الاضلاع $(\vb, \va_2)$ را در نظر بگیرید. چون $\vb = x\va_1 + y\va_2$:
\[
\text{مساحت}(\vb, \va_2) = \text{مساحت}(x\va_1 + y\va_2, \va_2)
\]
چون $\va_2 \times \va_2 = 0$:
\[
= \text{مساحت}(x\va_1, \va_2) = x \cdot \text{مساحت}(\va_1, \va_2)
\]
پس:
\[
x = \frac{\text{مساحت}(\vb, \va_2)}{\text{مساحت}(\va_1, \va_2)}
\]
\end{intuition}

% ============================================
\section{مثال محاسباتی}
% ============================================

\begin{example}
حل دستگاه:
\[
\system{
3x + 2y &= 7 \\
x + 4y &= 9
}
\]

\textbf{مرحله ۱:} محاسبه $\det(\mA)$:
\[
\det\twomat{3}{2}{1}{4} = 12 - 2 = 10
\]

\textbf{مرحله ۲:} محاسبه $x$:
\[
x = \frac{\det\twomat{7}{2}{9}{4}}{\det(\mA)} = \frac{28 - 18}{10} = \frac{10}{10} = 1
\]

\textbf{مرحله ۳:} محاسبه $y$:
\[
y = \frac{\det\twomat{3}{7}{1}{9}}{\det(\mA)} = \frac{27 - 7}{10} = \frac{20}{10} = 2
\]

\textbf{بررسی:} $3(1) + 2(2) = 7$ ✓ و $1(1) + 4(2) = 9$ ✓
\end{example}

% ============================================
\section{سه بعد و بالاتر}
% ============================================

\begin{example}
دستگاه $3 \times 3$:
\[
\system{
2x + y - z &= 3 \\
x - y + 2z &= 1 \\
3x + 2y + z &= 4
}
\]

ماتریس ضرایب:
\[
\mA = \threemat{2}{1}{-1}{1}{-1}{2}{3}{2}{1}
\]

\[
x = \frac{\det\threemat{3}{1}{-1}{1}{-1}{2}{4}{2}{1}}{\det(\mA)}, \quad
y = \frac{\det\threemat{2}{3}{-1}{1}{1}{2}{3}{4}{1}}{\det(\mA)}, \quad
z = \frac{\det\threemat{2}{1}{3}{1}{-1}{1}{3}{2}{4}}{\det(\mA)}
\]
\end{example}

\begin{intuition}
در سه بعد، به جای نسبت مساحت‌ها، نسبت \textbf{حجم‌ها} را داریم.

$x$ = نسبت حجم متوازی‌السطوح $(\vb, \va_2, \va_3)$ به حجم $(\va_1, \va_2, \va_3)$
\end{intuition}

% ============================================
\section{محدودیت‌ها و کاربردها}
% ============================================

\begin{warning}
قاعده کرامر:
\begin{itemize}
    \item فقط برای دستگاه‌های مربعی ($n$ معادله، $n$ مجهول)
    \item فقط وقتی $\det(\mA) \neq 0$ (جواب یکتا)
    \item برای $n$ بزرگ، محاسباتی \textbf{بسیار پرهزینه} است
\end{itemize}
\end{warning}

\begin{remark}
در عمل برای حل دستگاه‌های بزرگ از روش‌هایی مثل حذف گاوسی استفاده می‌شود که کارآمدتر هستند. اما قاعده کرامر:
\begin{itemize}
    \item درک نظری عمیق‌تری می‌دهد
    \item برای فرمول‌های تحلیلی مفید است
    \item در اثبات قضایا کاربرد دارد
\end{itemize}
\end{remark}

\begin{practical}
\textbf{کاربرد: یافتن تقاطع خطوط}

دو خط $a_1x + b_1y = c_1$ و $a_2x + b_2y = c_2$ در نقطه زیر تقاطع دارند:
\[
x = \frac{c_1 b_2 - c_2 b_1}{a_1 b_2 - a_2 b_1}, \quad y = \frac{a_1 c_2 - a_2 c_1}{a_1 b_2 - a_2 b_1}
\]
این فرمول مستقیم از قاعده کرامر می‌آید.
\end{practical}

% ============================================
\section{تمرین‌ها}
% ============================================

\begin{exercise}
دستگاه زیر را با قاعده کرامر حل کنید:
\[
\system{
2x + 3y &= 8 \\
4x - y &= 2
}
\]
\end{exercise}

\begin{exercise}
دستگاه زیر را با قاعده کرامر حل کنید:
\[
\system{
x + y + z &= 6 \\
2x - y + z &= 3 \\
x + 2y - z &= 2
}
\]
\end{exercise}

\begin{exercise}
نقطه تقاطع دو خط $2x + 3y = 7$ و $x - y = 1$ را با قاعده کرامر پیدا کنید.
\end{exercise}

\begin{exercise}
تفسیر هندسی قاعده کرامر را برای حالتی که $\det(\mA) = 0$ توضیح دهید.
\end{exercise}

\begin{exercise}[چالشی]
نشان دهید که فرمول کرامر با ضرب $\mA\inv \vb$ سازگار است.
\end{exercise}

\begin{problem}
یک مثلث با رأس‌های $A(1,1)$، $B(4,2)$، $C(2,5)$ داده شده. با استفاده از قاعده کرامر، مختصات مرکز ثقل را پیدا کنید.
\end{problem}


% Chapter 9: Change of Basis
% lecture09.tex - تغییر پایه
% Chapter 9: Change of Basis

\chapter{تغییر پایه}
\label{ch:changebasis}

\begin{abstract}
یک بردار یکسان در پایه‌های مختلف، مختصات متفاوتی دارد. در این درس یاد می‌گیریم چگونه بین دستگاه‌های مختصات مختلف ترجمه کنیم و این مفهوم چه ارتباطی با ماتریس‌ها دارد.
\end{abstract}

% ============================================
\section{مختصات نسبت به پایه‌های مختلف}
% ============================================

\begin{intuition}
مختصات یک بردار به پایه انتخابی بستگی دارد. اگر زبان متفاوتی صحبت کنید، همان مفهوم را متفاوت بیان می‌کنید.

مثال: بردار $\vv$ که در پایه استاندارد $\twovec{3}{2}$ است، اگر پایه‌ای متفاوت داشته باشیم، ممکن است $\twovec{1}{1}$ باشد!
\end{intuition}

\begin{definition}[مختصات در پایه]
اگر $\mathcal{B} = \{\vb_1, \vb_2\}$ یک پایه باشد، مختصات بردار $\vv$ در این پایه، ضرایب $c_1, c_2$ هستند که:
\[
\vv = c_1 \vb_1 + c_2 \vb_2
\]
نماد: $[\vv]_{\mathcal{B}} = \twovec{c_1}{c_2}$
\end{definition}

\begin{example}
پایه $\mathcal{B} = \left\{ \vb_1 = \twovec{2}{1}, \vb_2 = \twovec{-1}{1} \right\}$

بردار $\vv = \twovec{3}{2}$ را در این پایه بیان کنید.

باید $c_1, c_2$ را پیدا کنیم که:
\[
c_1 \twovec{2}{1} + c_2 \twovec{-1}{1} = \twovec{3}{2}
\]
\[
\system{
2c_1 - c_2 &= 3 \\
c_1 + c_2 &= 2
}
\]
حل: $c_1 = \frac{5}{3}$, $c_2 = \frac{1}{3}$

پس $[\vv]_{\mathcal{B}} = \twovec{5/3}{1/3}$
\end{example}

% ============================================
\section{ماتریس تغییر پایه}
% ============================================

\begin{definition}[ماتریس تغییر پایه]
\vocab{ماتریس تغییر پایه} از پایه $\mathcal{B}$ به پایه استاندارد، ماتریسی است که ستون‌هایش بردارهای پایه $\mathcal{B}$ هستند:
\[
\mP = [\vb_1 \,|\, \vb_2 \,|\, \cdots \,|\, \vb_n]
\]
\end{definition}

\begin{theorem}
\[
\vv = \mP \, [\vv]_{\mathcal{B}}
\]
یعنی: مختصات در پایه $\mathcal{B}$ $\times$ ماتریس تغییر پایه = بردار در پایه استاندارد
\end{theorem}

\begin{theorem}
\[
[\vv]_{\mathcal{B}} = \mP\inv \vv
\]
یعنی: برای تبدیل از پایه استاندارد به پایه $\mathcal{B}$، از معکوس استفاده می‌کنیم.
\end{theorem}

% ============================================
\section{تبدیل تغییر پایه}
% ============================================

\begin{intuition}
ماتریس $\mP$ یک تبدیل هویت است که فقط زبان را عوض می‌کند:
\begin{itemize}
    \item $\mP$: از زبان $\mathcal{B}$ به زبان استاندارد ترجمه می‌کند
    \item $\mP\inv$: از زبان استاندارد به زبان $\mathcal{B}$ ترجمه می‌کند
\end{itemize}
\end{intuition}

\begin{center}
\begin{tikzpicture}[scale=1.2]
    % Standard basis
    \begin{scope}[shift={(0,0)}]
        \draw[grid] (-0.5,-0.5) grid (3.5,2.5);
        \draw[axis] (-0.5,0) -- (3.5,0);
        \draw[axis] (0,-0.5) -- (0,2.5);
        \draw[basis] (0,0) -- (1,0) node[below] {$\vi$};
        \draw[basis] (0,0) -- (0,1) node[left] {$\vj$};
        \draw[vector, blue!70, very thick] (0,0) -- (3,2) node[above] {$\vv$};
        \node at (1.5,-1) {پایه استاندارد};
        \node at (1.5,-1.5) {$\vv = \twovec{3}{2}$};
    \end{scope}

    % Alternative basis
    \begin{scope}[shift={(6,0)}]
        \draw[gray!30] (-0.5,-0.5) -- (1,0.5) -- (3,1.5) -- (1.5,1) -- cycle;
        \draw[axis] (-0.5,0) -- (3.5,0);
        \draw[axis] (0,-0.5) -- (0,2.5);
        \draw[red!70, very thick, -{Stealth}] (0,0) -- (2,1) node[below right] {$\vb_1$};
        \draw[red!70, very thick, -{Stealth}] (0,0) -- (-1,1) node[above left] {$\vb_2$};
        \draw[vector, blue!70, very thick] (0,0) -- (3,2) node[above] {$\vv$};
        \node at (1.5,-1) {پایه $\mathcal{B}$};
        \node at (1.5,-1.5) {$[\vv]_{\mathcal{B}} = \twovec{?}{?}$};
    \end{scope}
\end{tikzpicture}
\end{center}

% ============================================
\section{نمایش تبدیل در پایه‌های مختلف}
% ============================================

\begin{theorem}[تبدیل در پایه جدید]
اگر $\mA$ ماتریس تبدیل $T$ در پایه استاندارد باشد، ماتریس همان تبدیل در پایه $\mathcal{B}$:
\[
[\mA]_{\mathcal{B}} = \mP\inv \mA \mP
\]
\end{theorem}

\begin{intuition}
این فرمول سه مرحله دارد:
\begin{enumerate}
    \item $\mP$: از پایه $\mathcal{B}$ به پایه استاندارد ترجمه کن
    \item $\mA$: تبدیل را در پایه استاندارد اعمال کن
    \item $\mP\inv$: نتیجه را به پایه $\mathcal{B}$ برگردان
\end{enumerate}
\end{intuition}

\begin{center}
\begin{tikzpicture}[node distance=3cm]
    \node (B1) {$[\vv]_{\mathcal{B}}$};
    \node (S1) [right of=B1] {$\vv$};
    \node (S2) [right of=S1] {$T(\vv)$};
    \node (B2) [right of=S2] {$[T(\vv)]_{\mathcal{B}}$};

    \draw[-{Stealth}, thick] (B1) -- (S1) node[midway, above] {$\mP$};
    \draw[-{Stealth}, thick] (S1) -- (S2) node[midway, above] {$\mA$};
    \draw[-{Stealth}, thick] (S2) -- (B2) node[midway, above] {$\mP\inv$};
    \draw[-{Stealth}, thick, blue] (B1) to[bend right=30] node[midway, below] {$\mP\inv\mA\mP$} (B2);
\end{tikzpicture}
\end{center}

% ============================================
\section{اهمیت تغییر پایه}
% ============================================

\begin{intuition}
چرا تغییر پایه مهم است؟

بعضی تبدیلات در پایه‌های خاص \textbf{ساده‌تر} به نظر می‌رسند. مثلاً:
\begin{itemize}
    \item چرخش در پایه استاندارد پیچیده است
    \item اما در پایه‌ای که یک محور روی محور چرخش باشد، ساده می‌شود
\end{itemize}

بهترین پایه برای یک تبدیل؟ \textbf{پایه ویژه} (eigenbasis) - درس بعدی!
\end{intuition}

\begin{practical}
\textbf{کاربرد: ساده‌سازی محاسبات}

فرض کنید می‌خواهید $\mA^{100}$ را محاسبه کنید. اگر در پایه‌ای مناسب، $\mA$ قطری شود:
\[
\mP\inv \mA \mP = \mD \quad \text{(قطری)}
\]
آنگاه:
\[
\mA^{100} = \mP \mD^{100} \mP\inv
\]
و $\mD^{100}$ بسیار ساده محاسبه می‌شود!
\end{practical}

% ============================================
\section{تمرین‌ها}
% ============================================

\begin{exercise}
پایه $\mathcal{B} = \left\{ \twovec{1}{1}, \twovec{1}{-1} \right\}$ داده شده. مختصات بردار $\vv = \twovec{3}{1}$ را در این پایه پیدا کنید.
\end{exercise}

\begin{exercise}
ماتریس تغییر پایه از $\mathcal{B} = \left\{ \twovec{2}{1}, \twovec{1}{1} \right\}$ به پایه استاندارد را بنویسید.
\end{exercise}

\begin{exercise}
اگر $\mA = \twomat{2}{1}{0}{3}$ در پایه استاندارد، ماتریس این تبدیل را در پایه $\mathcal{B} = \left\{ \twovec{1}{0}, \twovec{1}{1} \right\}$ پیدا کنید.
\end{exercise}

\begin{exercise}[چالشی]
نشان دهید که $\det(\mP\inv\mA\mP) = \det(\mA)$.
\end{exercise}

\begin{problem}
دو پایه $\mathcal{B}_1$ و $\mathcal{B}_2$ داده شده. ماتریس تغییر پایه مستقیم از $\mathcal{B}_1$ به $\mathcal{B}_2$ (بدون گذر از پایه استاندارد) را چگونه محاسبه می‌کنید؟
\end{problem}


% Chapter 10: Eigenvalues and Eigenvectors
% lecture10.tex - بردارها و مقادیر ویژه
% Chapter 10: Eigenvectors and Eigenvalues

\chapter{بردارها و مقادیر ویژه}
\label{ch:eigen}

\begin{abstract}
بردارهای ویژه جهت‌های خاصی هستند که تبدیل خطی آنها را فقط مقیاس می‌کند بدون اینکه جهتشان را تغییر دهد. این مفهوم یکی از مهم‌ترین ایده‌های جبر خطی است با کاربردهای گسترده در فیزیک، مهندسی، و علوم داده.
\end{abstract}

% ============================================
\section{انگیزه: بردارهای خاص}
% ============================================

\begin{intuition}
وقتی یک تبدیل خطی اعمال می‌کنید، اکثر بردارها از جهت اصلی‌شان منحرف می‌شوند. اما برخی بردارهای خاص فقط کشیده یا فشرده می‌شوند و \textbf{روی همان خط} باقی می‌مانند.

این بردارهای خاص، \textbf{بردارهای ویژه} نامیده می‌شوند.
\end{intuition}

\begin{center}
\begin{tikzpicture}[scale=1.2]
    % Before transformation
    \begin{scope}[shift={(0,0)}]
        \draw[grid] (-2,-2) grid (2,2);
        \draw[axis] (-2,0) -- (2,0);
        \draw[axis] (0,-2) -- (0,2);
        \draw[vector, blue!70, very thick] (0,0) -- (1,0.5) node[right] {$\vv_1$};
        \draw[vector, red!70, very thick] (0,0) -- (0.5,1) node[above] {$\vv_2$};
        \draw[vector, green!60!black, very thick] (0,0) -- (-1,1) node[above left] {ویژه!};
        \node at (0,-2.5) {قبل از تبدیل};
    \end{scope}

    \draw[-{Stealth}, very thick] (2.5,0) -- (4,0) node[midway, above] {$T$};

    % After transformation
    \begin{scope}[shift={(6.5,0)}]
        \draw[blue!20] (-2,-1) -- (0,2) -- (4,2) -- (2,-1) -- cycle;
        \draw[axis] (-2,0) -- (4,0);
        \draw[axis] (0,-2) -- (0,2);
        \draw[vector, blue!70, very thick] (0,0) -- (1.5,1) node[right] {$T(\vv_1)$};
        \draw[vector, red!70, very thick] (0,0) -- (1.5,1.5) node[above] {$T(\vv_2)$};
        \draw[vector, green!60!black, very thick] (0,0) -- (-2,2) node[above left] {$2 \times$ ویژه!};
        \node at (1,-2.5) {بعد از تبدیل};
    \end{scope}
\end{tikzpicture}
\end{center}

% ============================================
\section{تعریف رسمی}
% ============================================

\begin{definition}[بردار ویژه و مقدار ویژه]
بردار غیرصفر $\vv$ یک \vocab{بردار ویژه} ماتریس $\mA$ است اگر:
\[
\mA\vv = \lambda\vv
\]
برای یک عدد $\lambda$. این عدد \vocab{مقدار ویژه} متناظر نامیده می‌شود.
\end{definition}

\begin{intuition}
$\mA\vv = \lambda\vv$ یعنی:
\begin{center}
«تبدیل $\mA$ روی بردار $\vv$ فقط اثر یک ضرب اسکالری دارد»
\end{center}
بردار $\vv$ روی همان خط می‌ماند، فقط $\lambda$ برابر می‌شود.
\end{intuition}

\begin{example}
برای ماتریس $\mA = \twomat{3}{1}{0}{2}$:

بردار $\vv = \twovec{1}{0}$ یک بردار ویژه است:
\[
\mA\vv = \twomat{3}{1}{0}{2}\twovec{1}{0} = \twovec{3}{0} = 3\twovec{1}{0} = 3\vv
\]
مقدار ویژه متناظر: $\lambda = 3$
\end{example}

% ============================================
\section{یافتن مقادیر ویژه}
% ============================================

\begin{theorem}[معادله مشخصه]
$\lambda$ مقدار ویژه $\mA$ است اگر و تنها اگر:
\[
\det(\mA - \lambda\mI) = 0
\]
\end{theorem}

\begin{proof}
$\mA\vv = \lambda\vv$ را می‌توان نوشت:
\[
\mA\vv - \lambda\vv = \vzero \implies (\mA - \lambda\mI)\vv = \vzero
\]
این معادله جواب غیرصفر دارد اگر و تنها اگر $\mA - \lambda\mI$ معکوس‌پذیر نباشد، یعنی $\det(\mA - \lambda\mI) = 0$.
\end{proof}

\begin{definition}[چندجمله‌ای مشخصه]
$\det(\mA - \lambda\mI)$ یک چندجمله‌ای در $\lambda$ است که \vocab{چندجمله‌ای مشخصه} نامیده می‌شود. ریشه‌های آن مقادیر ویژه هستند.
\end{definition}

\begin{example}
برای $\mA = \twomat{3}{1}{0}{2}$:
\[
\det(\mA - \lambda\mI) = \det\twomat{3-\lambda}{1}{0}{2-\lambda} = (3-\lambda)(2-\lambda) - 0 = 0
\]
ریشه‌ها: $\lambda_1 = 3$, $\lambda_2 = 2$
\end{example}

% ============================================
\section{یافتن بردارهای ویژه}
% ============================================

\begin{theorem}
برای هر مقدار ویژه $\lambda$، بردارهای ویژه متناظر از حل دستگاه همگن زیر به دست می‌آیند:
\[
(\mA - \lambda\mI)\vv = \vzero
\]
\end{theorem}

\begin{example}
ادامه مثال قبل با $\lambda = 2$:
\[
(\mA - 2\mI)\vv = \twomat{1}{1}{0}{0}\twovec{v_1}{v_2} = \vzero
\]
معادله: $v_1 + v_2 = 0$، پس $v_1 = -v_2$

بردار ویژه: $\vv = t\twovec{-1}{1}$ برای هر $t \neq 0$
\end{example}

% ============================================
\section{فضای ویژه}
% ============================================

\begin{definition}[فضای ویژه]
\vocab{فضای ویژه} متناظر با مقدار ویژه $\lambda$:
\[
E_\lambda = \Null(\mA - \lambda\mI) = \{\vv \mid \mA\vv = \lambda\vv\}
\]
\end{definition}

\begin{intuition}
فضای ویژه شامل همه بردارهایی است که تبدیل $\mA$ آنها را فقط با ضریب $\lambda$ مقیاس می‌کند. این فضا همیشه یک زیرفضای برداری است.
\end{intuition}

% ============================================
\section{تفسیر هندسی}
% ============================================

\begin{practical}
\textbf{چرخش سه‌بعدی}

برای یک چرخش در $\Rthree$، بردار ویژه با $\lambda = 1$ \textbf{محور چرخش} است! این بردار ثابت می‌ماند.

توصیف چرخش با محور و زاویه بسیار ساده‌تر از ماتریس $3 \times 3$ است.
\end{practical}

\begin{example}
ماتریس برش: $\mA = \twomat{1}{1}{0}{1}$

مقدار ویژه: $\lambda = 1$ (مضاعف)

بردار ویژه: $\twovec{1}{0}$ (فقط یک جهت ویژه)

تفسیر: برش، محور $x$ را ثابت نگه می‌دارد.
\end{example}

% ============================================
\section{حالات خاص}
% ============================================

\begin{theorem}[چرخش دوبعدی]
ماتریس چرخش $\mR_\theta = \twomat{\cos\theta}{-\sin\theta}{\sin\theta}{\cos\theta}$ برای $\theta \neq 0, \pi$:

مقادیر ویژه: $\lambda = \cos\theta \pm i\sin\theta$ (مختلط!)

هیچ بردار حقیقی روی جای خود نمی‌ماند - همه بردارها می‌چرخند.
\end{theorem}

\begin{warning}
مقادیر ویژه می‌توانند \textbf{مختلط} باشند حتی برای ماتریس‌های حقیقی! این در چرخش‌ها اتفاق می‌افتد.
\end{warning}

% ============================================
\section{کاربردها}
% ============================================

\begin{practical}
\textbf{Google PageRank}

صفحات وب را بردار، لینک‌ها را ماتریس در نظر بگیرید. بردار ویژه غالب (با بزرگ‌ترین مقدار ویژه) اهمیت نسبی صفحات را نشان می‌دهد.
\end{practical}

\begin{practical}
\textbf{تحلیل مؤلفه‌های اصلی (PCA)}

در یادگیری ماشین، بردارهای ویژه ماتریس کوواریانس، جهت‌های اصلی تغییرات داده را نشان می‌دهند.
\end{practical}

\begin{practical}
\textbf{مکانیک کوانتومی}

مقادیر ویژه عملگرها = نتایج ممکن اندازه‌گیری

بردارهای ویژه = حالت‌های پایدار سیستم
\end{practical}

% ============================================
\section{تمرین‌ها}
% ============================================

\begin{exercise}
مقادیر ویژه و بردارهای ویژه ماتریس زیر را پیدا کنید:
\[
\mA = \twomat{4}{1}{2}{3}
\]
\end{exercise}

\begin{exercise}
نشان دهید که مقادیر ویژه ماتریس قطری، درایه‌های قطری آن هستند.
\end{exercise}

\begin{exercise}
مقادیر ویژه ماتریس چرخش $90°$ را پیدا کنید.
\end{exercise}

\begin{exercise}
اگر $\lambda$ مقدار ویژه $\mA$ باشد، نشان دهید $\lambda^2$ مقدار ویژه $\mA^2$ است.
\end{exercise}

\begin{exercise}[چالشی]
ثابت کنید که اثر ماتریس (مجموع درایه‌های قطری) برابر مجموع مقادیر ویژه است.
\end{exercise}

\begin{problem}
ماتریس پوپولاسیون:
\[
\mL = \twomat{0}{4}{0.5}{0}
\]
مقدار ویژه غالب را پیدا کنید و تفسیر کنید.
\end{problem}


% Chapter 11: Computing Eigenvalues and Eigenbasis
% lecture11.tex - محاسبه مقادیر ویژه و پایه ویژه
% Chapter 11: Computing Eigenvalues and Eigenbasis

\chapter{محاسبه مقادیر ویژه و پایه ویژه}
\label{ch:eigenbasis}

\begin{abstract}
در این درس روش‌های سریع‌تر برای محاسبه مقادیر ویژه ماتریس‌های $2 \times 2$ را یاد می‌گیریم و با مفهوم پایه ویژه و قطری‌سازی آشنا می‌شویم.
\end{abstract}

% ============================================
\section{ترفند سریع برای ماتریس $2 \times 2$}
% ============================================

\begin{theorem}[فرمول سریع]
برای ماتریس $\mA = \twomat{a}{b}{c}{d}$:

\textbf{حاصل‌جمع مقادیر ویژه:}
\[
\lambda_1 + \lambda_2 = a + d = \tr(\mA)
\]

\textbf{حاصل‌ضرب مقادیر ویژه:}
\[
\lambda_1 \cdot \lambda_2 = ad - bc = \det(\mA)
\]
\end{theorem}

\begin{intuition}
با دانستن $m = \lambda_1 + \lambda_2$ و $p = \lambda_1 \lambda_2$، می‌توانید $\lambda_{1,2}$ را پیدا کنید:
\[
\lambda = \frac{m}{2} \pm \sqrt{\left(\frac{m}{2}\right)^2 - p}
\]
یا: دو عدد پیدا کنید که مجموعشان $m$ و حاصل‌ضربشان $p$ باشد.
\end{intuition}

\begin{example}
$\mA = \twomat{3}{1}{4}{1}$

اثر: $m = 3 + 1 = 4$

دترمینان: $p = 3 - 4 = -1$

مقادیر ویژه: دو عددی که $x + y = 4$ و $xy = -1$:
\[
\lambda = 2 \pm \sqrt{4 - (-1)} = 2 \pm \sqrt{5}
\]
\end{example}

% ============================================
\section{پایه ویژه \lr{(Eigenbasis)}}
% ============================================

\begin{definition}[پایه ویژه]
اگر بردارهای ویژه یک ماتریس بتوانند یک \vocab{پایه} تشکیل دهند (یعنی $n$ بردار ویژه مستقل خطی داشته باشیم)، این پایه را \vocab{پایه ویژه} می‌نامیم.
\end{definition}

\begin{theorem}
در پایه ویژه، ماتریس تبدیل \vocab{قطری} می‌شود:
\[
[\mA]_{\text{پایه ویژه}} = \mD = \threemat{\lambda_1}{0}{0}{0}{\lambda_2}{0}{0}{0}{\ddots}
\]
\end{theorem}

\begin{intuition}
چرا قطری؟

در پایه ویژه، هر بردار پایه فقط مقیاس می‌شود:
\begin{align*}
\mA \vv_1 &= \lambda_1 \vv_1 \to \text{ستون اول: } \threevec{\lambda_1}{0}{0} \\
\mA \vv_2 &= \lambda_2 \vv_2 \to \text{ستون دوم: } \threevec{0}{\lambda_2}{0}
\end{align*}
\end{intuition}

% ============================================
\section{قطری‌سازی}
% ============================================

\begin{definition}[قطری‌سازی]
ماتریس $\mA$ \vocab{قطری‌پذیر} است اگر:
\[
\mA = \mP \mD \mP\inv
\]
که $\mD$ قطری و $\mP$ ماتریس بردارهای ویژه است.
\end{definition}

\begin{theorem}[شرط قطری‌پذیری]
ماتریس $n \times n$ قطری‌پذیر است اگر و تنها اگر $n$ بردار ویژه مستقل خطی داشته باشد.
\end{theorem}

\begin{example}
$\mA = \twomat{3}{1}{0}{2}$ با مقادیر ویژه $\lambda_1 = 3$, $\lambda_2 = 2$

بردارهای ویژه: $\vv_1 = \twovec{1}{0}$, $\vv_2 = \twovec{-1}{1}$

\[
\mP = \twomat{1}{-1}{0}{1}, \quad \mD = \twomat{3}{0}{0}{2}
\]

بررسی: $\mP\mD\mP\inv = \mA$ ✓
\end{example}

% ============================================
\section{توان ماتریس با قطری‌سازی}
% ============================================

\begin{theorem}
اگر $\mA = \mP\mD\mP\inv$:
\[
\mA^n = \mP \mD^n \mP\inv
\]
و $\mD^n$ بسیار ساده است:
\[
\mD^n = \threemat{\lambda_1^n}{0}{0}{0}{\lambda_2^n}{0}{0}{0}{\ddots}
\]
\end{theorem}

\begin{example}
محاسبه $\mA^{100}$ برای مثال قبل:
\[
\mA^{100} = \mP \twomat{3^{100}}{0}{0}{2^{100}} \mP\inv
\]
بدون قطری‌سازی، باید ۱۰۰ ضرب ماتریسی انجام می‌دادیم!
\end{example}

\begin{practical}
\textbf{کاربرد: اعداد فیبوناچی}

دنباله فیبوناچی: $F_{n+1} = F_n + F_{n-1}$

ماتریس:
\[
\twomat{1}{1}{1}{0}^n = \twomat{F_{n+1}}{F_n}{F_n}{F_{n-1}}
\]

با قطری‌سازی، فرمول بسته پیدا می‌کنیم:
\[
F_n = \frac{1}{\sqrt{5}}\left[\left(\frac{1+\sqrt{5}}{2}\right)^n - \left(\frac{1-\sqrt{5}}{2}\right)^n\right]
\]
\end{practical}

% ============================================
\section{ماتریس‌های غیرقطری‌پذیر}
% ============================================

\begin{example}
ماتریس برش $\mA = \twomat{1}{1}{0}{1}$:

چندجمله‌ای مشخصه: $(1-\lambda)^2 = 0$

مقدار ویژه: $\lambda = 1$ (مضاعف)

بردار ویژه: فقط $\twovec{1}{0}$ (یک بعدی)

این ماتریس قطری‌پذیر نیست!
\end{example}

\begin{warning}
مقدار ویژه تکراری لزوماً مشکل‌ساز نیست. مشکل زمانی است که تعداد بردارهای ویژه مستقل کمتر از تعداد تکرار مقدار ویژه باشد.
\end{warning}

% ============================================
\section{ماتریس‌های متقارن}
% ============================================

\begin{theorem}[قضیه طیفی]
ماتریس متقارن ($\mA = \mA\trans$):
\begin{enumerate}
    \item مقادیر ویژه حقیقی دارد
    \item بردارهای ویژه متناظر با مقادیر ویژه متمایز، عمود هستند
    \item همیشه قطری‌پذیر است (با پایه متعامد)
\end{enumerate}
\end{theorem}

\begin{intuition}
ماتریس‌های متقارن «خوش‌رفتار» هستند. آنها همیشه قطری می‌شوند و بردارهای ویژه‌شان عمود هستند - مثل محورهای اصلی یک بیضی.
\end{intuition}

% ============================================
\section{تمرین‌ها}
% ============================================

\begin{exercise}
با ترفند سریع، مقادیر ویژه ماتریس زیر را پیدا کنید:
\[
\mA = \twomat{5}{2}{2}{2}
\]
\end{exercise}

\begin{exercise}
ماتریس $\mA = \twomat{2}{1}{1}{2}$ را قطری کنید.
\end{exercise}

\begin{exercise}
با استفاده از قطری‌سازی، $\mA^{10}$ را محاسبه کنید:
\[
\mA = \twomat{1}{1}{0}{2}
\]
\end{exercise}

\begin{exercise}
آیا ماتریس زیر قطری‌پذیر است؟
\[
\mA = \twomat{2}{1}{0}{2}
\]
\end{exercise}

\begin{exercise}[چالشی]
نشان دهید که $\mA$ و $\mA\trans$ مقادیر ویژه یکسانی دارند.
\end{exercise}

\begin{problem}
جمعیت خرگوش‌ها و روباه‌ها با مدل زیر توصیف می‌شود:
\[
\twovec{R_{n+1}}{F_{n+1}} = \twomat{1.1}{-0.4}{0.2}{0.8}\twovec{R_n}{F_n}
\]
رفتار بلندمدت جمعیت را تحلیل کنید.
\end{problem}


% Chapter 12: Abstract Vector Spaces
% lecture12.tex - Abstract Vector Spaces
% Chapter 12: Abstract Vector Spaces

\chapter{Abstract Vector Spaces}
\label{ch:abstract}

\begin{abstract}
In this final chapter, we extend the concept of vectors beyond geometric arrows. We will see that functions, polynomials, and even music can be ``vectors'' - as long as they follow the rules of linear algebra.
\end{abstract}

% ============================================
\section{Motivation: What is a Vector?}
% ============================================

\begin{intuition}
So far we have known vectors as arrows or lists of numbers. But mathematicians ask a deeper question:

\textbf{``What things behave like vectors?''}

Answer: Anything that can be added and multiplied by scalars!
\end{intuition}

\begin{example}
Functions can be added: $(f + g)(x) = f(x) + g(x)$

Functions can be multiplied by numbers: $(cf)(x) = c \cdot f(x)$

So functions can be ``vectors''!
\end{example}

% ============================================
\section{Formal Definition of Vector Space}
% ============================================

\begin{definition}[Vector Space]
A \vocab{vector space} over the field $\R$ is a set $V$ with two operations:
\begin{itemize}
    \item \textbf{Addition:} $+ : V \times V \to V$
    \item \textbf{Scalar multiplication:} $\cdot : \R \times V \to V$
\end{itemize}
that satisfy the following axioms.
\end{definition}

\begin{theorem}[Vector Space Axioms]
For all $\vu, \vv, \vw \in V$ and $a, b \in \R$:

\textbf{Addition axioms:}
\begin{enumerate}
    \item $\vu + \vv = \vv + \vu$ (commutativity)
    \item $(\vu + \vv) + \vw = \vu + (\vv + \vw)$ (associativity)
    \item Existence of identity: $\exists \vzero : \vv + \vzero = \vv$
    \item Existence of inverse: $\forall \vv, \exists (-\vv) : \vv + (-\vv) = \vzero$
\end{enumerate}

\textbf{Scalar multiplication axioms:}
\begin{enumerate}[resume]
    \item $a(b\vv) = (ab)\vv$
    \item $1 \cdot \vv = \vv$
\end{enumerate}

\textbf{Distributive axioms:}
\begin{enumerate}[resume]
    \item $a(\vu + \vv) = a\vu + a\vv$
    \item $(a + b)\vv = a\vv + b\vv$
\end{enumerate}
\end{theorem}

% ============================================
\section{Examples of Vector Spaces}
% ============================================

\subsection{$\R^n$ - The Standard Space}

\begin{example}
$\R^n = \{(x_1, x_2, \ldots, x_n) \mid x_i \in \R\}$ with usual addition and multiplication.

This is the same space we have been working with until now.
\end{example}

\subsection{The Space of Polynomials}

\begin{example}
$\mathcal{P}_n$ = the set of polynomials of degree at most $n$:
\[
p(x) = a_0 + a_1 x + a_2 x^2 + \cdots + a_n x^n
\]

Addition: $(p + q)(x) = p(x) + q(x)$

Scalar multiplication: $(cp)(x) = c \cdot p(x)$

Zero vector: $p(x) = 0$

\textbf{Basis:} $\{1, x, x^2, \ldots, x^n\}$ - dimension of space: $n + 1$
\end{example}

\subsection{The Space of Functions}

\begin{example}
$C[a,b]$ = the set of continuous functions on interval $[a,b]$

This space is \textbf{infinite-dimensional}!
\end{example}

\subsection{The Space of Matrices}

\begin{example}
$M_{m \times n}$ = the set of $m \times n$ matrices

Addition: element-wise addition

Scalar multiplication: multiply all entries by the scalar

Dimension: $m \times n$
\end{example}

% ============================================
\section{Abstract Linear Transformations}
% ============================================

\begin{definition}[Linear Transformation Between Spaces]
A function $T: V \to W$ is a \vocab{linear transformation} if:
\begin{enumerate}
    \item $T(\vu + \vv) = T(\vu) + T(\vv)$
    \item $T(c\vv) = c \cdot T(\vv)$
\end{enumerate}
\end{definition}

\begin{example}
\textbf{Differentiation:} $D: \mathcal{P}_n \to \mathcal{P}_{n-1}$ with $D(p) = p'$

This is linear because $(f + g)' = f' + g'$ and $(cf)' = cf'$.
\end{example}

\begin{example}
\textbf{Definite integral:} $I: C[0,1] \to \R$ with $I(f) = \int_0^1 f(x) dx$

This is linear because $\int(f+g) = \int f + \int g$ and $\int cf = c\int f$.
\end{example}

% ============================================
\section{Functions as Infinite-Dimensional Vectors}
% ============================================

\begin{intuition}
A function $f: [0, 2\pi] \to \R$ can be thought of as an infinite-dimensional vector:
\begin{itemize}
    \item Each point $x$ is a ``component''
    \item The value $f(x)$ is the value of that component
\end{itemize}

Inner product of functions:
\[
\langle f, g \rangle = \int_0^{2\pi} f(x) g(x) dx
\]
\end{intuition}

\begin{practical}
\textbf{Fourier Series}

Sine and cosine functions form a ``basis'' for periodic functions:
\[
f(x) = \frac{a_0}{2} + \sum_{n=1}^{\infty} (a_n \cos nx + b_n \sin nx)
\]

The coefficients $a_n, b_n$ are like ``coordinates'' of the function in this basis!
\end{practical}

% ============================================
\section{Why Does Abstraction Matter?}
% ============================================

\begin{summary}
\textbf{The Power of Abstraction:}

When something is a vector space, \textbf{all the tools of linear algebra} are applicable:
\begin{itemize}
    \item Linear independence and dependence
    \item Basis and dimension
    \item Linear transformations and matrices
    \item Eigenvalues and eigenvectors
    \item Image and null space
\end{itemize}

One theorem in linear algebra = a theorem for all these spaces!
\end{summary}

\begin{practical}
\textbf{Quantum Mechanics}

Quantum states form a vector space (Hilbert space). Physical operators are linear transformations. Measurement values = eigenvalues!
\end{practical}

\begin{practical}
\textbf{Signal Processing}

Audio signals are vectors in function space. Fourier transform is a change of basis. Filters are linear transformations.
\end{practical}

\begin{practical}
\textbf{Machine Learning}

Data are vectors in feature space. Linear models are linear transformations. Dimensionality reduction = finding a better basis.
\end{practical}

% ============================================
\section{Looking Ahead}
% ============================================

\begin{remark}
This course was an introduction to linear algebra. More advanced topics include:
\begin{itemize}
    \item \textbf{Quadratic forms} and classification of conic sections
    \item \textbf{Singular Value Decomposition (SVD)} - powerful tool in data science
    \item \textbf{Numerical linear algebra} - efficient algorithms
    \item \textbf{Hilbert spaces} - infinite-dimensional linear algebra
    \item \textbf{Representation theory} - groups and linear algebra
\end{itemize}
\end{remark}

% ============================================
\section{Exercises}
% ============================================

\begin{exercise}
Show that the set of $2 \times 2$ symmetric matrices is a vector space. What is its dimension?
\end{exercise}

\begin{exercise}
Is the set of polynomials with $p(0) = 1$ a vector space? Why or why not?
\end{exercise}

\begin{exercise}
Show that differentiation $D: \mathcal{P}_3 \to \mathcal{P}_2$ is a linear transformation. Write its matrix in the standard basis.
\end{exercise}

\begin{exercise}
Prove that in any vector space: $0 \cdot \vv = \vzero$
\end{exercise}

\begin{exercise}[Challenge]
Consider the solution space of the differential equation $y'' + y = 0$. Show that this is a vector space and find a basis for it.
\end{exercise}

\begin{problem}
For the linear transformation $T: \mathcal{P}_2 \to \mathcal{P}_2$ with $T(p) = p + p'$:
\begin{enumerate}[label=(\alph*)]
    \item Write the matrix of $T$ in the basis $\{1, x, x^2\}$
    \item Find the eigenvalues
    \item Find the eigenpolynomials (eigenvectors)
\end{enumerate}
\end{problem}

% ============================================
\section{Course Summary}
% ============================================

\begin{summary}
\textbf{The Essence of Linear Algebra}

We started from vectors and matrices and arrived at abstract spaces. Key ideas:

\begin{enumerate}
    \item \textbf{Vector:} Something that has addition and scalar multiplication
    \item \textbf{Linear transformation:} A function that preserves lines
    \item \textbf{Matrix:} The numerical representation of a linear transformation
    \item \textbf{Determinant:} The volume scaling factor
    \item \textbf{Eigenvalues:} Special directions that only get scaled
    \item \textbf{Abstraction:} All these concepts work beyond arrows
\end{enumerate}

Linear algebra is the common language of mathematics, physics, engineering, and computer science.
\end{summary}



% === Appendix: Glossary ===
\appendix
\chapter{Glossary}

\begin{center}
\begin{tabular}{l|l}
\toprule
\textbf{Term} & \textbf{Symbol/Notation} \\
\midrule
Vector & $\vv, \vw, \vu$ \\
Matrix & $\mA, \mB, \mC$ \\
Linear Transformation & $T: \Rtwo \to \Rtwo$ \\
Determinant & $\det(\mA)$ \\
Eigenvalue & $\lambda$ \\
Eigenvector & $\vv$ \\
Basis & $\{\vi, \vj\}$ \\
Span & $\spn\{\vv_1, \vv_2, \ldots\}$ \\
Null Space & $\Null(\mA)$ \\
Column Space & $\Col(\mA)$ \\
Row Space & $\Row(\mA)$ \\
Rank & $\rank(\mA)$ \\
Linear Independence & vectors with no redundancy \\
Linear Dependence & at least one redundant vector \\
Dot Product & $\vv \cdot \vw$ \\
Cross Product & $\vv \times \vw$ \\
Identity Matrix & $\mI$ \\
Inverse Matrix & $\mA\inv$ \\
Transpose & $\mA\trans$ \\
Change of Basis & $\mP\inv \mA \mP$ \\
Eigenbasis & basis of eigenvectors \\
Diagonal Matrix & non-zero only on diagonal \\
Scalar & a number (real or complex) \\
Linear Combination & $c_1\vv_1 + c_2\vv_2 + \cdots$ \\
Vector Space & set closed under $+$ and scalar $\times$ \\
\bottomrule
\end{tabular}
\end{center}

\end{document}

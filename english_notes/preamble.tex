% preamble.tex - English Linear Algebra Lecture Notes
% Standard LaTeX configuration (no RTL support needed)

\documentclass[12pt,a4paper,openany,oneside]{book}

% === Page Geometry ===
\usepackage{geometry}
\geometry{
    top=2.5cm,
    bottom=2.5cm,
    left=2.5cm,
    right=2.5cm,
    headheight=14pt
}

% === Math Packages ===
\usepackage{amsmath}
\usepackage{amssymb}
\usepackage{amsthm}
\usepackage{mathtools}
\usepackage{bm}

% === Graphics and Diagrams ===
\usepackage{graphicx}
\usepackage{tikz}
\usepackage{pgfplots}
\pgfplotsset{compat=1.18}
\usetikzlibrary{arrows.meta, calc, positioning, decorations.pathreplacing, patterns}

% === Colors and Boxes ===
\usepackage{xcolor}
\usepackage{tcolorbox}
\tcbuselibrary{skins,breakable,theorems}

% === Tables and Lists ===
\usepackage{booktabs}
\usepackage{array}
\usepackage{enumitem}

% === Hyperlinks ===
\usepackage[colorlinks=true,linkcolor=blue!70!black,urlcolor=blue!70!black,citecolor=green!50!black]{hyperref}

% === Theorem Environments (English) ===
\theoremstyle{definition}
\newtheorem{definition}{Definition}[chapter]
\newtheorem{example}{Example}[chapter]
\newtheorem{exercise}{Exercise}[chapter]
\newtheorem{problem}{Problem}[chapter]

\theoremstyle{plain}
\newtheorem{theorem}{Theorem}[chapter]
\newtheorem{lemma}{Lemma}[chapter]
\newtheorem{proposition}{Proposition}[chapter]
\newtheorem{corollary}{Corollary}[chapter]

\theoremstyle{remark}
\newtheorem{remark}{Remark}[chapter]
\newtheorem{note}{Note}[chapter]

% === Custom tcolorbox Environments ===

% Geometric Intuition Box
\newtcolorbox{intuition}{
    enhanced,
    colback=blue!5,
    colframe=blue!75!black,
    fonttitle=\bfseries,
    title={Geometric Intuition},
    breakable,
    left=5mm,
    right=5mm,
    top=3mm,
    bottom=3mm,
    arc=2mm,
    boxrule=1pt
}

% Practical Application Box
\newtcolorbox{practical}{
    enhanced,
    colback=green!5,
    colframe=green!60!black,
    fonttitle=\bfseries,
    title={Practical Application},
    breakable,
    left=5mm,
    right=5mm,
    top=3mm,
    bottom=3mm,
    arc=2mm,
    boxrule=1pt
}

% Warning Box
\newtcolorbox{warning}{
    enhanced,
    colback=red!5,
    colframe=red!75!black,
    fonttitle=\bfseries,
    title={Warning},
    breakable,
    left=5mm,
    right=5mm,
    top=3mm,
    bottom=3mm,
    arc=2mm,
    boxrule=1pt
}

% Important Note Box
\newtcolorbox{important}{
    enhanced,
    colback=orange!5,
    colframe=orange!75!black,
    fonttitle=\bfseries,
    title={Important},
    breakable,
    left=5mm,
    right=5mm,
    top=3mm,
    bottom=3mm,
    arc=2mm,
    boxrule=1pt
}

% Summary Box
\newtcolorbox{summary}{
    enhanced,
    colback=purple!5,
    colframe=purple!75!black,
    fonttitle=\bfseries,
    title={Summary},
    breakable,
    left=5mm,
    right=5mm,
    top=3mm,
    bottom=3mm,
    arc=2mm,
    boxrule=1pt
}

% === Page Header/Footer ===
\usepackage{fancyhdr}
\pagestyle{fancy}
\fancyhf{}
\fancyhead[L]{\leftmark}
\fancyhead[R]{\rightmark}
\fancyfoot[C]{\thepage}
\renewcommand{\headrulewidth}{0.4pt}
\renewcommand{\footrulewidth}{0pt}

% === Spacing ===
\usepackage{setspace}
\onehalfspacing

% === Custom Commands for Math ===
% commands.tex - Custom Math Commands for Linear Algebra
% English Linear Algebra Lecture Notes

% === Vector Notation ===
% Bold vectors (column vectors)
\newcommand{\vect}[1]{\mathbf{#1}}
\newcommand{\vv}{\vect{v}}
\newcommand{\vw}{\vect{w}}
\newcommand{\vu}{\vect{u}}
\newcommand{\vx}{\vect{x}}
\newcommand{\vy}{\vect{y}}
\newcommand{\vz}{\vect{z}}
\newcommand{\va}{\vect{a}}
\newcommand{\vb}{\vect{b}}
\newcommand{\vc}{\vect{c}}

% Basis vectors (with hat notation)
\newcommand{\vi}{\hat{\imath}}
\newcommand{\vj}{\hat{\jmath}}
\newcommand{\vk}{\hat{k}}

% Zero vector
\newcommand{\vzero}{\vect{0}}

% === Matrix Notation ===
\newcommand{\mat}[1]{\mathbf{#1}}
\newcommand{\mA}{\mat{A}}
\newcommand{\mB}{\mat{B}}
\newcommand{\mC}{\mat{C}}
\newcommand{\mD}{\mat{D}}
\newcommand{\mI}{\mat{I}}
\newcommand{\mP}{\mat{P}}
\newcommand{\mQ}{\mat{Q}}
\newcommand{\mT}{\mat{T}}
\newcommand{\mM}{\mat{M}}
\newcommand{\mR}{\mat{R}}
\newcommand{\mS}{\mat{S}}
\newcommand{\mH}{\mat{H}}

% Matrix with brackets
\newcommand{\bmat}[1]{\begin{bmatrix}#1\end{bmatrix}}
\newcommand{\pmat}[1]{\begin{pmatrix}#1\end{pmatrix}}
\newcommand{\vmat}[1]{\begin{vmatrix}#1\end{vmatrix}}

% === Common Operations ===
% Transpose
\newcommand{\trans}{^{\mathsf{T}}}

% Inverse
\newcommand{\inv}{^{-1}}

% Norm and absolute value
\newcommand{\norm}[1]{\left\| #1 \right\|}
\newcommand{\abs}[1]{\left| #1 \right|}

% === Important Spaces ===
\newcommand{\R}{\mathbb{R}}
\newcommand{\C}{\mathbb{C}}
\newcommand{\N}{\mathbb{N}}
\newcommand{\Z}{\mathbb{Z}}
\newcommand{\Q}{\mathbb{Q}}

% Rn space
\newcommand{\Rtwo}{\mathbb{R}^2}
\newcommand{\Rthree}{\mathbb{R}^3}

% === Linear Algebra Operators ===
\DeclareMathOperator{\spn}{span}
\DeclareMathOperator{\Span}{Span}
\DeclareMathOperator{\Null}{Null}
\DeclareMathOperator{\Col}{Col}
\DeclareMathOperator{\Row}{Row}
\DeclareMathOperator{\rank}{rank}
\DeclareMathOperator{\tr}{tr}
\DeclareMathOperator{\proj}{proj}

% === Common Matrices ===
% 2x2 matrix shorthand
\newcommand{\twomat}[4]{\begin{bmatrix} #1 & #2 \\ #3 & #4 \end{bmatrix}}

% 3x3 matrix shorthand
\newcommand{\threemat}[9]{\begin{bmatrix} #1 & #2 & #3 \\ #4 & #5 & #6 \\ #7 & #8 & #9 \end{bmatrix}}

% 2D column vector
\newcommand{\twovec}[2]{\begin{bmatrix} #1 \\ #2 \end{bmatrix}}

% 3D column vector
\newcommand{\threevec}[3]{\begin{bmatrix} #1 \\ #2 \\ #3 \end{bmatrix}}

% === Change of Basis ===
\newcommand{\coords}[2]{[#1]_{#2}}

% === System of Equations ===
\newcommand{\system}[1]{\left\{\begin{aligned} #1 \end{aligned}\right.}

% === Emphasizing ===
\newcommand{\vocab}[1]{\textbf{\textcolor{blue!70!black}{#1}}}


% === TikZ styles for linear algebra diagrams ===
\tikzset{
    vector/.style={-{Stealth[length=3mm]}, thick},
    axis/.style={-{Stealth[length=2mm]}, gray},
    grid/.style={very thin, gray!30},
    transformed/.style={blue!70},
    original/.style={black},
    basis/.style={red!70!black, very thick},
    point/.style={circle, fill, inner sep=1.5pt}
}
